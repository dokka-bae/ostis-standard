\scsectionfamily{Часть 2 Стандарта OSTIS. Смысловое представление и онтологическая систематизация знаний в интеллектуальных компьютерных системах нового поколения}
\label{part_representation}

\scsection[
    \protect\scneditors{Никифоров С.А.;Бобёр  Е.С.}
    \protect\scnmonographychapter{Глава 2.6. Языковые средства формального описания синтаксиса и денотационной семантики различных языков в интеллектуальных компьютерных системах нового поколения}
    ]{Предметная область и онтология информационных конструкций и языков}
\label{intro_lang}
\input{Contents/part_kb/intro_lang.tex}

\scsubsection[
    \protect\scnidtf{Предметная область и онтология языка внутреннего представления информационных конструкций в памяти ostis-систем}
    \protect\scnmonographychapter{Глава 2.1. Универсальный язык смыслового представления знаний и смысловое пространство}
    ]{Предметная область и онтология внутреннего языка ostis-систем}
    \label{intro_sc_code}
\input{Contents/part_kb/intro_lang/intro_sc_code.tex}

\scsubsection[
    \protect\scnmonographychapter{Глава 2.2. Семейство внешних языков интеллектуальных компьютерных систем нового поколения, близких языку внутреннего смыслового представления знаний (SCg, SCs, SCn)}
    ]{Предметная область и онтология внешних идентификаторов знаков, входящих в информационные конструкции внутреннего языка ostis-систем}
\label{intro_idtf}
\begin{SCn}
\scnsectionheader{\currentname}
\begin{scnsubstruct}
\begin{scnreltovector}{конкатенация сегментов}
\scnitem{Понятие внешнего идентификатора sc-элемента}
\scnitem{Понятие простого идентификатора sc-элемента}
\scnitem{Понятие сложного идентификатора sc-элемента}
\end{scnreltovector}
\scnheader{Предметная область внешних идентификаторов знаков, входящих в информационные конструкции внутреннего языка ostis-систем}
\scniselement{предметная область}
\begin{scnhaselementrole}{класс объектов исследования}
sc-идентификатор\end{scnhaselementrole}
\begin{scnhaselementrolelist}{класс объектов исследования}
основной sc-идентификатор;строковый sc-идентификатор;системный sc-идентификатор;нетранслируемый sc-идентификатор;простой sc-идентификатор;простой строковый sc-идентификатор;имя нарицательное;имя собственное;sc-выражение;ограничитель sc-выражений
\end{scnhaselementrolelist}
\begin{scnhaselementrolelist}{исследуемое отношение}
sc-идентификатор*
\end{scnhaselementrolelist}
\scnsegmentheader{Понятие внешнего идентификатора sc-элемента}
\begin{scnsubstruct}
\scnheader{sc-идентификатор}
\scnidtf{строка символов или пиктограмма, взаимно однозначно представляющая соответствующий sc-элемент, хранимый в sc-памяти}
\scnidtf{внешний идентификатор sc-элемента}
\scntext{пояснение}{Внешние идентификаторы \textit{sc-элементов} (или, сокращенно \scnkeyword{sc-идентификаторы}
) необходимы \mbox{\textit{ostis-системе}} исключительно для того, чтобы осуществлять обмен информацией с другими \textit{ostis-системами} или со своими пользователями. Для того чтобы представить свою \textit{базу знаний}, решать самые различные \textit{задачи}, связанные с анализом текущего состояния и эволюцией своей \textit{базы знаний}, задачи, связанные с анализом текущего состояния (текущих ситуаций) окружающей среды, принятием соответствующих решений (целей) и организацией соответствующих \textit{действий}, направленных на выполнение принятых решений (на достижение поставленных целей), \textit{ostis-системе} не нужны никакие внешние идентификаторы (в частности, имена) соответствующие \textit{sc-элементам}. Но для \uline{понимания} сообщений, принимаемых от других субъектов (что для \textit{ostis-системы} означает построение \textit{sc-текста},~~ \textit{семантически эквивалентного} принятому сообщению) и для анализа сообщений, передаваемых другим субъектам (что для \textit{ostis-системы} означает синтез \textit{внешнего текста},~~\textit{семантически эквивалентного} заданному \textit{sc-тексту} и удовлетворяющего некоторым дополнительным требованиям, например, эмоционального характера) \textit{ostis-системе} необходимо знать, как в принимаемом или передаваемом сообщении изображаются (представляются) \textit{знаки}, \uline{синонимичные sc-элементам}, которые уже хранятся или могут храниться в составе \textit{базы знаний}~~\textit{ostis-системы}. В качестве внешних идентификаторов \textit{sc-элементов} чаще всего используются имена (термины) соответствующих (обозначаемых) сущностей, представленные отдельными словами или словосочетаниями на различных естественных языках, но также могут использоваться иероглифы, условные обозначения, пиктограммы.В общем случае \textit{sc-элементу} может соответствовать несколько синонимичных ему имен на разных \textit{естественных языках}. Более того, \textit{sc-элементу} может соответствовать несколько синонимичных ему имен на одном и том же \textit{естественном языке}. В этом случае одно из этих имен объявляется как основной внешний идентификатор для соответствующего \textit{sc-элемента} и соответствующего \textit{естественного языка}. Основное требование, предъявляемое к таким внешним идентификаторам это отсутствие как синонимов, так и омонимов в рамках множества основных внешних идентификаторов sc-элементов для каждого естественного языка. Каждый внешний идентификатор \textit{sc-элемента}, используемый ostis-системой, может быть описан (представлен) в её памяти в виде \textit{внутреннего файла ostis-системы}, т.е. в виде электронного образа всевозможных вхождений данного внешнего идентификатора во всевозможные внешние тексты соответствующего внешнего языка. В некоторых случаях явное представление в памяти не требуется, например, в случае \textit{нетранслируемых sc-идентификаторов}.}
\begin{scnsubdividing}
\scnitem{простой sc-идентификатор\\\scnidtf{простой внешний идентификатор sc-элемента}
}
\scnitem{sc-выражение\\\scnidtf{сложный внешний идентификатор sc-элемента, в состав которого входит один или несколько идентификаторов других sc-элементов}
}
\end{scnsubdividing}
\begin{scnsubdividing}
\scnitem{основной sc-идентификатор\\\scnidtf{основной sc-идентификатор для носителей дополнительно указываемого языка общения (например, соответствующего естественного языка)}
\scntext{примечание}{\textit{основной sc-идентификатор} является уникальным (не имеет синонимов и омонимов) в рамках соответствующего естественного языка}
\scnsuperset{основной международный sc-идентификатор}
\scntext{примечание}{В качестве \textit{основных sc-идентификаторов} могут использоваться также общепринятые международные условные обозначения некоторых сущностей, например, обозначения часто используемых функций (sin, cos, tg, log, и т.д.), единиц измерения, денежных единиц и многое другое. Формально каждый основной международный sc-идентификатор считается основным sc-идентификатором также и для каждого естественного языка, несмотря на то, что символы, используемые в основных международных sc-идентификаторах, могут не соответствовать алфавиту некоторых или даже всех естественных языков.}
}
\scnitem{неосновной sc-идентификатор\\\scntext{примечание}{С помощью неосновных sc-идентификаторов указываются возможные \textit{синонимы*} соответствующего \textit{основного sc-идентификатора}, которые в частности, могут пояснять или даже определять обозначаемую сущность, указывает на важные свойства этой сущности.}
\scnsuperset{\normalfont(}
неосновной sc-идентификатор $\cap$ пояснение\normalfont)}
\end{scnsubdividing}

\end{scnsubstruct}

\end{scnsubstruct}

\end{SCn}

\scsubsection[
    \protect\scnmonographychapter{Глава 2.2. Семейство внешних языков интеллектуальных компьютерных систем нового поколения, близких языку внутреннего смыслового представления знаний (SCg, SCs, SCn)}
    ]{Предметная область и онтология языка внешнего графического представления информационных конструкций внутреннего языка ostis-систем}
\label{intro_scg}
\begin{SCn}
\scnsectionheader{\currentname}
\begin{scnsubstruct}
\scnidtf{Описание SCg-кода}
\begin{scnreltovector}{конкатенация сегментов}
\scnitem{Основные положения языка графического представления знаний ostis-систем}
\scnitem{Описание Ядра SCg-кода}
\scnitem{Описание Первого расширения Ядра SCg-кода}
\scnitem{Описание Второго расширения Ядра SCg-кода}
\scnitem{Описание Третьего расширения Ядра SCg-кода}
\scnitem{Описание Четвертого расширения Ядра SCg-кода}
\scnitem{Описание Пятого расширения Ядра SCg-кода}
\scnitem{Описание Шестого расширения Ядра SCg-кода}
\scnitem{Описание Седьмого расширения Ядра SCg-кода}
\end{scnreltovector}
\scnsegmentheader{Основные положения языка графического представления знаний ostis-систем}
\begin{scnsubstruct}
\scnheader{SCg-код}
\scnidtf{Semantic Code graphical}
\scnidtf{Язык визуального (графического) представления баз знаний ostis-систем}
\scniselement{графовый язык}
\scntext{explanation}{\textit{SCg-код} представляет собой способ визуализации \textit{sc-текстов} (информационных конструкций SC-кода) в виде рисунков этих абстрактных конструкций. Подчеркнем, что абстрактная \textit{графовая структура} и её рисунок (графическое изображение) -- это не одно и то же даже если они \textit{изоморфны} друг другу. \mbox{\textit{SCg-код}} рассматривается нами как объединение \textit{Ядра SCg-кода}, обеспечивающего изоморфное графическое изображение любого \textit{sc-текста}, а также нескольких направлений расширения этого ядра, обеспечивающих повышение компактности и читабельности}
\end{scnsubstruct}
\end{scnsubstruct}
\end{SCn}

\scsubsubsection[
    \protect\scnmonographychapter{Глава 2.2. Семейство внешних языков интеллектуальных компьютерных систем нового поколения, близких языку внутреннего смыслового представления знаний (SCg, SCs, SCn)}
    ]{Предметная область и онтология синтаксиса языка внешнего графического представления информационных конструкций внутреннего языка ostis-систем}
\label{intro_scg_syntax}

\scsubsubsection[
    \protect\scnmonographychapter{Глава 2.2. Семейство внешних языков интеллектуальных компьютерных систем нового поколения, близких языку внутреннего смыслового представления знаний (SCg, SCs, SCn)}
    ]{Предметная область и онтология денотационной семантики языка внешнего графического представления информационных конструкций внутреннего языка ostis-систем}
\label{intro_scg_semantic}

\scsubsubsection[
    \protect\scnmonographychapter{Глава 2.2. Семейство внешних языков интеллектуальных компьютерных систем нового поколения, близких языку внутреннего смыслового представления знаний (SCg, SCs, SCn)}
    ]{Предметная область и онтология иерархического семейства подъязыков, семантически эквивалентных языку внешнего графического представления информационных конструкций внутреннего языка ostis-систем}
\label{intro_scg_sublang}

\scsubsection[
    \protect\scnmonographychapter{Глава 2.2. Семейство внешних языков интеллектуальных компьютерных систем нового поколения, близких языку внутреннего смыслового представления знаний (SCg, SCs, SCn)}
    ]{Предметная область и онтология языка внешнего линейного представления информационных конструкций внутреннего языка ostis-систем}
\label{intro_scs}
\begin{SCn}
\scnsectionheader{\currentname}
\begin{scnsubstruct}
\scnidtf{Описание \textit{SCs-кода}}
\begin{scnreltovector}{конкатенация сегментов}
\scnitem{Описание Алфавита SCs-кода}
\scnitem{Описание sc.s-разделителей и sc.s-ограничителей}
\scnitem{Описание sc.s-предложений}
\scnitem{Описание Ядра SCs-кода и различных направлений его расширения}
\end{scnreltovector}
\scnheader{SCs-код}
\scnidtf{Semantic Code string}
\scnidtf{Язык линейного представления знаний ostis-систем}
\scnidtf{Множество всевозможных текстов \textit{SCs-кода}}
\scnidtf{Тексты \textit{SCs-кода}}
\scniselement{имя собственное}
\scnidtf{текст \textit{SCs-кода}}
\scniselement{имя нарицательное}
\scnidtf{sc.s-текст}
\scniselement{линейный язык}
\scnrelfrom{алфавит}{Алфавит SCs-кода}
\scnrelfrom{разделители}{sc.s-разделитель}
\scnrelfrom{ограничители}{sc.s-ограничитель}
\scnrelfrom{предложения}{sc.s-предложение}
\scnrelfrom{неоднозначные обозначения описываемых сущностей}{неоднозначное sc.s-изображение sc-элемента}
\scnidtftext{explanation}{Множество линейных текстов (\textit{sc.s-текстов}), каждый из которых состоит из предложений (\textit{sc.s-предложений}), разделенных друг от друга двойной \textit{точкой с запятой} (разделителем \textit{sc.s-предложений}). При этом \mbox{\textit{sc.s-предложение}} представляет собой последовательность \textit{sc-идентификаторов}, являющихся именами описываемых \textit{сущностей} и разделяемых между собой различными \textit{sc.s-разделителями} и \textit{sc.s-ограничителями}}
\scnheader{неоднозначное sc.s-изображение sc-элемента}
\scnrelboth{пара пересекающихся множеств}{sc-выражение}
\scnidtf{условное обозначение неименуемой (неидентифицируемой) сущности}
\scnsuperset{sc.s-коннектор}
\scnidtf{неоднозначное sc.s-изображение \textit{sc-коннектора}, являющееся также одновременно одним из видов \textit{sc.s-разделителей}}
\scnsubset{sc.s-разделитель}
\scnsuperset{неоднозначное sc.s-изображение sc-узла}
\scnsuperset{условное обозначение неименуемого множества sc-элементов}
\scntext{explanation}{условное обозначение неименуемого множества sc-элементов в \textit{SCs-коде} представляется строкой из двух символов -- \textit{левой фигурной скобки} и \textit{правой фигурной скобки}.}\scnsuperset{условное обозначение неименуемого кортежа sc-элементов}
\scntext{explanation}{В \textit{SCs-коде} такое обозначение представляется двух-символьной \textit{строкой}, состоящей из \textit{левой угловой скобки} и \textit{правой угловой скобки}}\newpage\scnsuperset{условное обозначение неименуемого файла-экземпляра ostis-системы}
\scntext{explanation}{В \textit{SCs-коде} такое обозначение представляется двух-символьной \textit{строкой}, состоящей из \textit{левой квадратной скобки} и \textit{правой квадратной скобки}}\scnsuperset{условное обозначение неименуемого файла-образца ostis-системы}
\scntext{explanation}{В \textit{SCs-коде} такое обозначение представляется \textit{строкой}, состоящей из \textit{восклицательного знака}, \textit{левой квадратной скобки}, \textit{правой квадратной скобки} и еще одного \textit{восклицательного знака}}\scnsegmentheader{Описание Алфавита SCs-кода}
\begin{scnsubstruct}
\scnheader{Алфавит SCs-кода}
\scnidtf{Алфавит символов SCs-кода}
\scnidtf{множество символов SCs-кода}
\scnidtf{символ, используемый в текстах SCs-кода}
\begin{scnreltoset}{объединение}
\scnitem{Алфавит символов, используемых в sc.s-разделителях}
\scnitem{Алфавит символов, используемых в sc.s-ограничителях}
\scnitem{Алфавит символов, используемых в sc-идентификаторах\\\begin{scnreltoset}{объединение}
\scnitem{Алфавит символов, используемых в простых строковых sc-идентификаторах}
\scnitem{Алфавит символов, используемых в sc-выражениях}
\end{scnreltoset}
}
\scnitem{Алфавит символов, используемых в неоднозначных sc.s-изображениях sc-узлов}
\end{scnreltoset}
\begin{scnrelfromlist}{принципы}
\scnitem{\scnfileitem{Алфавит SCs-кода строится на основе современных общепринятых наборов символов, что позволяет упростить разработку средств для работы с sc.s-текстами с использованием современных технологий.}
}
\scnitem{\scnfileitem{В состав sc.s-текстов, как и в состав текстов любых других языков, являющихся вариантами внешнего отображения текстов SC-кода, могут входить различные файлы, в том числе естественно-языковые или даже файлы, содержащие другие sc.s-тексты. В общем случае в таких файлах могут использоваться самые разные символы, в связи с чем будем считать, что в Алфавит SCs-кода эти символы не включаются.}
}
\end{scnrelfromlist}
\scnheader{Алфавит символов, используемых в sc.s-разделителях}
\begin{scnhaselementrolelist}
\scnitem{\textit{пробел}}
\scnitem{ \textit{точка с запятой}}
\scnitem{ \textit{двоеточие}}
\scnitem{ \textit{круглый маркер}}
\scnitem{ \textit{знак равенства}}
\end{scnhaselementrolelist}
\scnsuperset{Алфавит символов, используемых в sc.s-разделителях, изображающих связь инцидентности sc-элементов}
\begin{scnhaselementrolelist}
\scnitem{\scnfileclass{<}
}
\scnitem{~\scnfileclass{>}
}
\scnitem{~\scnfileclass{|}
}
\scnitem{~\scnfileclass{-}
}
\end{scnhaselementrolelist}
\scnsuperset{Алфавит символов, используемых в sc.s-коннекторах}
\scnsuperset{Расширенный алфавит символов, используемых в sc.s-коннекторах}
\scnidtf{Расширенный алфавит sc.s-коннекторов}
\begin{scnhaselementrolelist}
\scnitem{\scnfileclass{$\in$}
}
\scnitem{~\scnfileclass{$\ni$}
}
\scnitem{~\scnfileclass{$\notin$}
}
\scnitem{~\scnfileclass{$\not \ni$}
}
\scnitem{~\scnfileclass{$\subseteq$}
}
\scnitem{~\scnfileclass{$\supseteq$}
}
\scnitem{~\scnfileclass{$\subset$}
}
\scnitem{~\scnfileclass{$\supset$}
}
\scnitem{~\scnfileclass{$\leq$}
}
\scnitem{~\scnfileclass{$\geq$}
}
\scnitem{~\scnfileclass{$\Leftarrow$}
}
\scnitem{~\scnfileclass{$\Rightarrow$}
}
\scnitem{~\scnfileclass{$\Leftrightarrow$}
}
\scnitem{~\scnfileclass{$\leftarrow$}
}
\scnitem{~\scnfileclass{$\rightarrow$}
}
\scnitem{~\scnfileclass{$\leftrightarrow$}
}
\end{scnhaselementrolelist}
\scnsuperset{Базовый алфавит символов, используемых в sc.s-коннекторах}
\scnidtf{Базовый алфавит sc.s-коннекторов}
\begin{scnhaselementrolelist}
\scnitem{\scnfileclass{$\sim$}
}
\scnitem{~\textit{знак подчеркивания}}
\scnitem{~ \textit{знак равенства}}
\scnitem{~\scnfileclass{>}
}
\scnitem{~ \scnfileclass{<}
}
\scnitem{~\textit{двоеточие}}
\scnitem{~\scnfileclass{-}
}
\scnitem{~\scnfileclass{|}
}
\scnitem{~\scnfileclass{/}
}
\end{scnhaselementrolelist}
\scntext{note}{Как в Базовом, так и в Расширенном Алфавитах sc.s-коннекторов используются следующие общие признаки, характеризующие тип изображаемого sc-коннектора:\begin{scnitemize}
\item \textit{знак подчеркивания} как признак изображений переменных sc-коннекторов (один  \textit{знак подчеркивания} для sc-коннекторов, являющихся первичными sc-переменными, два  \textit{знака подчеркивания} для sc-коннекторов, являющихся вторичными sc-переменными (sc-метапеременными));\item \textit{вертикальная черта} (|) как признак изображений негативных sc-дуг принадлежности;\item \textit{косая черта} (/) как признак изображений нечетких sc-дуг принадлежности;\item \textit{тильда} ($\sim$) как признак изображений временных sc-дуг принадлежности\end{scnitemize}
При необходимости комбинации указанных признаков перечисленные символы комбинируются так, как показано в сегменте \textit{Описание sc.s-разделителей и sc.s-ограничителей}.}\bigskip\begin{scnset}
\scnitem{Расширенный алфавит символов, используемых в sc.s-коннекторах}
\scnitem{ Базовый алфавит символов, используемых в sc.s-коннекторах}
\end{scnset}
\scntext{explanation}{Для упрощения процесса разработки исходных текстов баз знаний с использованием SCs-кода и создания соответствующих средств вводятся два алфавита символов. \textit{Базовый алфавит символов, используемых в sc.s-коннекторах} включает только символы, входящие в переносимый набор символов (portable character set) и имеющиеся на стандартной современной клавиатуре. Таким образом, для разработки исходных текстов баз знаний, использующих только \textit{Базовый алфавит символов, используемых в sc.s-коннекторах} достаточно обычного текстового редактора. \textit{Расширенный алфавит символов, используемых в sc.s-коннекторах} включает также дополнительные символы, которые позволяют сделать sc.s-тексты (и sc.n-тексты) более читабельными и наглядными. Для визуализации и разработки sc.s-текстов с использованием расширенного алфавита требуется наличие специализированных средств.}\scnheader{Алфавит символов, используемых в sc.s-ограничителях}
\begin{scnhaselementrolelist}
\scnitem{\scnfileclass{(}
}
\scnitem{~\scnfileclass{)}
}
\scnitem{~\scnfileclass{*}
}
\end{scnhaselementrolelist}
\scnheader{Алфавит символов, используемых в неоднозначных sc.s-изображениях sc-узлов}
\begin{scnhaselementrolelist}
\scnitem{\scnfileclass{\}
\scnitem{~\scnfileclass{\}
}
}
\scnitem{~\scnfileclass{-}
}
\scnitem{~\scnfileclass{!}
}
\scnitem{~\scnfileclass{}
}
\scnitem{~\scnfileclass{}
}
\end{scnhaselementrolelist}
\end{scnsubstruct}
\end{scnsubstruct}
\end{SCn}

\scsubsubsection[
    \protect\scnmonographychapter{Глава 2.2. Семейство внешних языков интеллектуальных компьютерных систем нового поколения, близких языку внутреннего смыслового представления знаний (SCg, SCs, SCn)}
    ]{Предметная область и онтология синтаксиса языка внешнего линейного представления информационных конструкций внутреннего языка ostis-систем}
\label{intro_scs_syntax}

\scsubsubsection[
    \protect\scnmonographychapter{Глава 2.2. Семейство внешних языков интеллектуальных компьютерных систем нового поколения, близких языку внутреннего смыслового представления знаний (SCg, SCs, SCn)}
    ]{Предметная область и онтология денотационной семантики языка внешнего линейного представления информационных конструкций внутреннего языка ostis-систем}
\label{intro_scs_semantic}

\scsubsubsection[
    \protect\scnmonographychapter{Глава 2.2. Семейство внешних языков интеллектуальных компьютерных систем нового поколения, близких языку внутреннего смыслового представления знаний (SCg, SCs, SCn)}
    ]{Предметная область и онтология иерархического семейства подъязыков, семантически эквивалентных языку внешнего линейного представления информационных конструкций внутреннего языка ostis-систем}
\label{intro_scs_sublang}

\scsubsection[
    \protect\scnmonographychapter{Глава 2.2. Семейство внешних языков интеллектуальных компьютерных систем нового поколения, близких языку внутреннего смыслового представления знаний (SCg, SCs, SCn)}
    ]{Предметная область и онтология языка внешнего форматированного представления информационных конструкций внутреннего языка ostis-систем}
\label{intro_scn}
\input{Contents/part_kb/intro_lang/intro_scn.tex}

\scsubsubsection[
    \protect\scnmonographychapter{Глава 2.2. Семейство внешних языков интеллектуальных компьютерных систем нового поколения, близких языку внутреннего смыслового представления знаний (SCg, SCs, SCn)}
    ]{Предметная область и онтология синтаксиса языка внешнего форматированного представления информационных конструкций внутреннего языка ostis-систем}
\label{intro_scn_syntax}

\scsubsubsection[
    \protect\scnmonographychapter{Глава 2.2. Семейство внешних языков интеллектуальных компьютерных систем нового поколения, близких языку внутреннего смыслового представления знаний (SCg, SCs, SCn)}
    ]{Предметная область и онтология денотационной семантики языка внешнего форматированного представления информационных конструкций внутреннего языка ostis-систем}
\label{intro_scn_semantic}

\scsubsubsection[
    \protect\scnmonographychapter{Глава 2.2. Семейство внешних языков интеллектуальных компьютерных систем нового поколения, близких языку внутреннего смыслового представления знаний (SCg, SCs, SCn)}
    ]{Предметная область и онтология иерархического семейства подъязыков, семантически эквивалентных языку внешнего форматированного представления информационных конструкций внутреннего языка ostis-систем}
\label{intro_scn_sublang}

\scsection[
    \protect\scneditor{Банцевич К.А.}
    \protect\scnmonographychapter{Глава 2.3. Структура баз знаний интеллектуальных компьютерных систем нового поколения: иерархическая система предметных областей и онтологий. Онтологии верхнего уровня. Формализация понятий семантической окрестности, предметной области и онтологии в интеллектуальных компьютерных системах нового поколения}
    ]{Предметная область и онтология знаний и баз знаний ostis-систем}
\label{sd_knowledge}
\begin{SCn}
\scnsectionheader{\currentname}
\begin{scnsubstruct}
\begin{scnrelfromlist}{дочерний раздел}
\scnitem{Предметная область и онтология множеств\scnidtf{Предметная область и онтология \textit{знаний о множествах}}
\scntext{note}{\textit{знания о множествах} являются \uline{частным видом} \textit{знаний} и, следовательно, общие свойства сущностей, описываемых знаниями, могут наследоваться \textit{Предметной областью и онтологией множеств}}}
\scnitem{Предметная область и онтология связок и отношений}
\scnitem{Предметная область и онтология параметров, величин и шкал}
\scnitem{Предметная область и онтология чисел и числовых структур}
\scnitem{Предметная область и онтология структур}
\scnitem{Предметная область и онтология темпоральных сущностей}
\scnitem{Предметная область и онтология темпоральных сущностей баз знаний ostis-систем}
\scnitem{Предметная область и онтология семантических окрестностей}
\scnitem{Предметная область и онтология предметных областей}
\scnitem{Предметная область и онтология онтологий}
\scnitem{Предметная область и онтология логических формул, высказываний и формальных теорий}
\scnitem{Предметная область и онтология внешних информационных конструкций и файлов ostis-систем}
\scnitem{Глобальная предметная область действий и задач и соответствующая ей онтология методов и технологий}
\end{scnrelfromlist}
\scnheader{Предметная область знаний и баз знаний ostis-систем}
\scniselement{предметная область}
\begin{scnhaselementrole}{класс объектов исследования}
знание\end{scnhaselementrole}
\begin{scnhaselementrolelist}
\scnitem{исследуемый класс классов}
\end{scnhaselementrolelist}
\scnheader{знание}
\scnidtf{синтаксически корректная (для соответствующего языка) и семантически целостная информационная конструкция}
\scnsubset{информационная конструкция}
\scniselementrole{класс объектов исследования}{\nameref{intro_lang}}
\scnrelfrom{покрытие}{вид знаний\scnidtf{Множество \uline{всевозможных} видов знаний}
\scntext{note}{Тот факт, что семейство \textit{видов знаний} является \textit{покрытием} Множества всевозможных \textit{знаний}, означает то, что каждое \textit{знание} принадлежит по крайней мере одному выделенному нами \textit{виду знаний}}}
\scnheader{вид знаний}
\scnhaselement{спецификация}
\scnidtf{описание заданной сущности}
\scnsuperset{спецификация материальной сущности}
\scnsuperset{спецификация обратной сущности, не являющейся множеством}
\scnsuperset{спецификация геометрической точки}
\scnsuperset{спецификация числа}
\scnsuperset{спецификация множества}
\scnsuperset{спецификация связи}
\scnsuperset{спецификация структуры}
\scnsuperset{спецификация класса}
\scnsuperset{спецификация класса сущностей, не являющихся множествами}
\scnsuperset{спецификация отношения}
\scnidtf{спецификация класса связей (связок)}
\scnsuperset{спецификация класса классов}
\scnsuperset{спецификация параметра}
\scnsuperset{спецификация класса структур}
\scnsuperset{спецификация понятий}
\scnsuperset{пояснение}
\scnsuperset{определение}
\scnsuperset{утверждение}
\scnidtf{утверждение, описывающее свойства экземпляров (элементов) специфицируемого понятия}
\scnidtf{закономерность}
\scnsuperset{семантическая окрестность}
\scnsuperset{однозначная спецификация}
\scnsuperset{сравнительный анализ}
\scnsuperset{достоинства}
\scnsuperset{недостатки}
\scnsuperset{структура специфицируемой сущности}
\scnsuperset{принципы, лежащие в основе}
\scnsuperset{обоснование предлагаемого решения}
\scnidtf{аргументация предлагаемого решения}
\scnhaselement{сравнение}
\scnhaselement{высказывание}
\scnsuperset{фактографическое высказывание}
\scnsuperset{закономерность}
\scnhaselement{формальная теория}
\scnhaselement{предметная область}
\scnhaselement{предметная область и онтология\scnidtf{предметная область и её онтология}
\scnidtf{предметная область и соответствующая ей объединенная онтология}
}
\scnhaselement{метазнание}
\scnidtf{спецификация знания}
\scnsuperset{аннотация}
\scnsuperset{введение}
\scnsuperset{предисловие}
\scnsuperset{заключение}
\scnsuperset{онтология}
\scnsuperset{онтология предметной области}
\scnsuperset{структурная онтология предметной области}
\scnsuperset{теоретико-множественная онтология предметной области}
\scnsuperset{логическая онтология предметной области}
\scnsuperset{терминологическая онтология предметной области}
\scnsuperset{объединенная онтология предметной области}
\scnhaselement{задача}
\scnidtf{спецификация действия}
\scnhaselement{план}
\scnhaselement{протокол}
\scnhaselement{результативная часть протокола}
\scnhaselement{метод}
\scnhaselement{технология}
\scnhaselement{история использования предметной области и её онтологии по решению информационных задач}
\scnhaselement{история использования предметной области и её онтологии по решению задач во внешней среде}
\scnhaselement{история эволюции предметной области и её онтологии}
\scnhaselement{база знаний}
\scnidtf{совокупность знаний, хранимых в памяти интеллектуальной компьютерной системы и \uline{достаточных} для того, чтобы указанная система удовлетворяла соответствующим предъявляемым к ней требованиям (в частности, чтобы она имела соответствующий уровень интеллекта)}
\scnidtf{систематизированная совокупность знаний, хранимая в памяти интеллектуальной компьютерной системы и достаточная для обеспечения целенаправленного (целесообразного, адекватного) функционирования (поведения) этой системы как в своей внешней среде, так и в своей внутренней среде (в собственной базе знаний)}
\begin{scnrelfromset}{обобщенная декомпозиция}
\scnitem{согласованная часть базы знаний\scnidtf{часть базы знаний, признанная коллективом авторов на текущий момент}
}
\scnitem{история эксплуатации базы знаний}
\scnitem{история эволюции базы знаний}
\scnitem{план эволюции базы знаний\scnidtf{система специфицированных и согласованных действий авторов базы знаний, направленных на повышение её качества}
}
\end{scnrelfromset}
\scntext{note}{Основным факторами, определяющими качество интеллектуальной компьютерной системы, являются:\begin{scnitemize}
\item качественная структуризация (систематизация) и \uline{стратификация} базы знаний интеллектуальной компьютерной системы, а также\item систематизация и стратификация \uline{деятельности}, которая осуществляется интеллектуальной компьютерной системой и спецификация которой является важнейшей частью базы знаний этой системы (Смотрите Раздел \textit{Глобальная предметная область действий и задач и соответствующая ей онтология методов и технологий}).\end{scnitemize}
}\scntext{note}{Даже небольшой перечень \textit{видов знаний} свидетельствует об огромном многообразии \textit{видов знаний}}\scnheader{знание}
\begin{scnsubdividing}
\scnitem{декларативное знание\scnidtf{\textit{знание}, имеющее \uline{только} \textit{денотационную семантику}, которая представляется в виде семантической \textit{спецификации} системы \textit{понятий}, используемых в этом \textit{знании}}
}
\scnitem{процедурное знание\scnidtf{\textit{знание}, имеющее не только \textit{денотационную семантику}, но и \textit{операционную семантику}, которая представляется в виде семейства \textit{спецификаций агентов}, осуществляющих интерпретацию \textit{процедурного знания}, направленную на решение некоторой инициированной \textit{задачи}}
\scnidtf{функционально интерпретируемое знание, обеспечивающее решение либо конкретной задачи, либо некоторого множества инициируемых задач}
\scnsuperset{задача}
\scnidtf{формулировка конкретной задачи}
\scnsuperset{декларативная формулировка задачи}
\scnsuperset{процедурная формулировка задачи}
\scnsuperset{план}
\scnidtf{план решения конкретной задачи}
\scnidtf{контекст конкретной задачи, предоставляющий всю информацию для решения всех подзадач для указанной конкретной задачи}
\scnidtf{описание системы подзадач некоторой задачи}
\scnsuperset{метод}
\scnidtf{обобщенное описание плана решения любой задачи из некоторого заданного класса задач}
\scnsuperset{навык}
\scnidtf{метод, детализированный до уровня элементарных подзадач}
}
\end{scnsubdividing}
\scnheader{отношение, заданное на множестве знаний}
\scnhaselement{дочернее знание*}
\scnidtf{знание, которое от материнского знания наследует все описанные там свойства объектов исследования}
\scntext{note}{Факт наследования свойств описываемых объектов от материнского знания подчеркивается использованием прилагательного дочернее в sc-идентификаторе данного отношения, заданного на множестве знаний}\scnsuperset{дочерний раздел*}
\scnidtf{частный раздел*}
\scnsuperset{дочерняя предметная область и онтология*}
\scnhaselement{спецификация*}
\scnidtf{быть знанием, которое является спецификацией (описанием) заданной сущности}
\scntext{note}{специфицируемой сущностью может быть сущность любого вида, в том числе, и другое знание}\scnhaselement{онтология*}
\scnidtf{быть семантической спецификацией заданного знания*}
\scnhaselement{семантическая эквивалентность*}
\scnhaselement{следовательно*}
\scnidtf{логическое следствие*}
\scnhaselement{логическая эквивалентность*}
\end{scnsubstruct}
\end{SCn}


\scsubsection[
    \protect\scnidtf{Предметная область и онтология знаний о множествах}
    \protect\scnmonographychapter{Глава 2.4. Формальные онтологии базовых классов сущностей - множеств, связей, отношений, параметров, величин, чисел, структур, темпоральных сущностей}
    ]{Предметная область и онтология множеств}
\label{sd_sets}
\begin{SCn}
\scnsectionheader{\currentname}
\begin{scnsubstruct}
\scnheader{Предметная область множеств}
\scnidtf{Теоретико-множественная предметная область}
\scnidtf{Предметная область теории множеств}
\scnidtf{Предметная область, объектами исследования которой являются множества}
\scniselement{предметная область}
\begin{scnhaselementrole}{класс объектов исследования}
множество
\end{scnhaselementrole}
\begin{scnhaselementrolelist}{класс объектов исследования}
\begin{itemize}
    \item\textit{конечное множество}
    \item\textit{бесконечное множество}
    \item\textit{счетное множество}
    \item\textit{несчетное множество}
    \item\textit{множество без кратных элементов}
    \item\textit{мультимножество}
    \item\textit{кратность принадлежности}
    \item\textit{класс}
    \item\textit{класс первичных sc-элементов}
    \item\textit{класс множеств}
    \item\textit{класс структур}
    \item\textit{класс классов}
    \item\textit{нечеткое множество}
    \item\textit{четкое множество}
    \item\textit{множество первичных сущностей}
    \item\textit{семейство множеств}
    \item\textit{нерефлексивное множество}
    \item\textit{рефлексивное множество}
    \item\textit{множество первичных сущностей и множеств}
    \item\textit{сформированное множество}
    \item\textit{несформированное множество}
    \item\textit{пустое множество}
    \item\textit{синглетон}
    \item\textit{пара}
    \item\textit{пара разных элементов}
    \item\textit{пара-мультимножество}
    \item\textit{тройка}
    \item\textit{кортеж}
    \item\textit{декартово произведение}
    \item\textit{булеан}
    \item\textit{мощность множества}
\end{itemize}
\end{scnhaselementrolelist}
\begin{scnhaselementrolelist}{исследуемое отношение}
\begin{itemize}
    \item\textit{принадлежность*}
    \item\textit{пример\scnrolesign}
    \item\textit{включение*}
    \item\textit{строгое включение*}
    \item\textit{объединение*}
    \item\textit{разбиение*}
    \item\textit{пересечение*}
    \item\textit{пара пересекающихся множеств*}
    \item\textit{попарно пересекающиеся множества*}
    \item\textit{пересекающиеся множества*}
    \item\textit{пара непересекающихся множеств*}\item\textit{попарно непересекающиеся множества*}\item\textit{непересекающиеся множества*}\item\textit{разность множеств*}\item\textit{симметрическая разность множеств*}\item\textit{декартово произведение*}\item\textit{семейство подмножеств*}\item\textit{булеан*}\item\textit{равенство множеств*}    
\end{itemize}
\end{scnhaselementrolelist}
\scnheader{множество}
\scnidtf{множество sc-элементов}
\scnidtf{sc-множество}
\scnidtf{множество знаков}
\scnidtf{множество знаков описываемых сущностей}
\scnidtf{семантически нормализованное множество}
\scnidtf{sc-знак множества sc-элементов}
\scnidtf{sc-знак множества sc-знаков}
\scnidtf{sc-текст}
\scnidtf{текст SC-кода}
\scnidtf{SC-код}
\begin{scnsubdividing}
\scnitem{конечное множество}
\scnitem{бесконечное множество}
\end{scnsubdividing}
\begin{scnsubdividing}
\scnitem{множество без кратных элементов}
\scnitem{мультимножество}
\end{scnsubdividing}
\begin{scnsubdividing}
\scnitem{связка}
\scnitem{класс\\\scnidtf{sc-элемент, обозначающий класс sc-элементов}
\scnidtf{sc-знак множества sc-элементов, эквивалентных в том или ином смысле}
}
\scnitem{структура\\\scnidtf{sc-знак множества sc-элементов, в состав которого входят sc-связки или структуры, связывающие эти sc-элементы}
}
\end{scnsubdividing}
\begin{scnsubdividing}
\scnitem{четкое множество}
\scnitem{нечеткое множество}
\end{scnsubdividing}
\begin{scnsubdividing}
\scnitem{множество первичных сущностей}
\scnitem{множество множеств}
\scnitem{множество первичных сущностей и множеств}
\end{scnsubdividing}
\begin{scnsubdividing}
\scnitem{рефлексивное множество}
\scnitem{нерефлексивное множество}
\end{scnsubdividing}
\begin{scnsubdividing}
\scnitem{сформированное множество}
\scnitem{несформированное множество}
\end{scnsubdividing}
\begin{scnsubdividing}
\scnitem{кортеж}
\scnitem{неориентированное множество}
\end{scnsubdividing}
\scnsuperset{пустое множество}
\scnsuperset{синглетон}
\scnsuperset{пара}
\scnsuperset{тройка}
\scntext{explanation}{Под \textbf{\textit{множеством}} понимается соединение в некое целое M определённых хорошо различимых предметов m нашего созерцания или нашего мышления (которые будут называться элементами множества M). \textbf{\textit{множество}}  мысленная сущность, которая связывает одну или несколько сущностей в целое.Более формально \textbf{\textit{множество}}  это абстрактный математический объект, состоящий из элементов. Связь множеств с их элементами задается бинарным ориентированным отношением \textbf{\textit{принадлежность*}}.\textbf{\textit{множество}} может быть полностью задано следующими тремя способами:\begin{scnitemize}
\item путем перечисления (явного указания) всех его элементов (очевидно, что таким способом можно задать только конечное множество)\item с помощью определяющего высказывания, содержащего описание общего характеристического свойства, которым обладают все те и только те объекты, которые являются элементами (т.е. принадлежат) задаваемого множества.\item с помощью теоретико-множественных операций, позволяющих однозначно задавать новые множества на основе уже заданных (это операции объединения, пересечения, разности множеств и др.)\end{scnitemize}
Для любого семантически ненормализованного \textbf{\textit{множества}} существует единственное семантически нормализованное \textbf{\textit{множество}}, в котором все элементы, не являющиеся знаками множеств, заменены на знаки множеств.}\scnheader{принадлежность*}
\scnidtf{принадлежность элемента множеству*}
\scnidtf{отношение принадлежности элемента множеству*}
\scniselement{бинарное отношение}
\scniselement{ориентированное отношение}
\scntext{explanation}{\textbf{\textit{принадлежность*}}  это бинарное ориентированное отношение, каждая связка которого связывает множество с одним из его элементов. Элементами отношения \textbf{\textit{принадлежность*}} по умолчанию являются \textit{позитивные sc-дуги принадлежности}.}\scnheader{конечное множество}
\scnidtf{множество с конечным числом элементов}
\scntext{explanation}{\textbf{\textit{конечное множество}}  это \textit{множество}, количество элементов которого конечно, т.е. существует неотрицательное целое число \textit{k}, равное количеству элементов этого множества.}\scnheader{бесконечное множество}
\scnidtf{множество с бесконечным числом элементов}
\begin{scnsubdividing}
\scnitem{счетное множество}
\scnitem{несчетное множество}
\end{scnsubdividing}
\scntext{explanation}{\textbf{\textit{бесконечное множество}}  это \textit{множество}, в котором для любого натурального числа \textit{n} найдётся конечное подмножество из \textit{n} элементов.}\scnheader{счетное множество}
\scntext{explanation}{\textbf{\textit{счетное множество}}  это \textit{бесконечное множество}, для которого существует \textit{взаимно-однозначное соответствие} с натуральным рядом чисел.}\scnheader{несчетное множество}
\scnidtf{континуальное множество}
\scntext{explanation}{\textbf{\textit{несчетное множество}} - это \textit{бесконечное множество}, элементы которого невозможно пронумеровать натуральными числами.}\scnheader{множество без кратных элементов}
\scnidtf{классическое множество}
\scnidtf{канторовское множество}
\scnidtf{множество, состоящее из разных элементов}
\scnidtf{множество без кратного вхождения элементов}
\scnidtf{множество, все элементы которого входят в него однократно}
\scnidtf{множество, не имеющее кратного вхождения элементов}
\scntext{explanation}{\textbf{\textit{множество без кратных элементов}} - это \textit{множество}, для каждого элемента которого существует только одна пара принадлежности, выходящая из знака этого множества в указанный элемент.}\scnheader{мультимножество}
\scnidtf{множество, имеющее кратные вхождения хотя бы одного элемента}
\scnidtf{множество, по крайней мере один элемент которого входит в его состав многократно}
\scntext{explanation}{\textbf{\textit{мультимножество}} - это \textit{множество}, для которого существует хотя бы одна кратная пара принадлежности, выходящая из знака этого множества.}\scnheader{кратность принадлежности}
\scnidtf{кратность принадлежности элемента}
\scnidtf{кратность вхождения элемента во множество}
\scniselement{параметр}
\scntext{explanation}{\textbf{\textit{кратность принадлежности}} - \textit{параметр}, значением которого являются числовые величины, показывающие сколько раз входит тот или иной элемент в рассматриваемое множество. Элементами данного параметра являются классы \textit{позитивных sc-дуг принадлежности}, связывающих данное множество с элементом, кратность вхождения которого в данное множество мы хотим задать.Таким образом, кратное вхождение элемента в мультимножество может быть задано как явным указанием \textit{позитивных sc-дуг принадлежности} этого элемента данному \textit{множеству}, так и склеиванием этих дуг в одну и включением ее в некоторый класс \textbf{\textit{кратности принадлежности}}.}\scnrelfrom{описание примера}{\scnfileimage[20em]{figures/sd_sets/multiplicityOfMembership.png}
}
\scnheader{класс}
\scnidtf{класс sc-элементов}
\begin{scnsubdividing}
\scnitem{класс первичных sc-элементов}
\scnitem{класс множеств}
\end{scnsubdividing}
\scntext{explanation}{\textbf{\textit{класс}}  множество элементов, обладающих какими-либо явно указываемыми общими свойствами.}\scnheader{класс первичных sc-элементов}
\scntext{explanation}{\textbf{\textit{класс первичных sc-элементов}}  класс, элементами которого являются только \textit{sc-элементы}, не являющиеся знаками множеств.}\scnheader{класс множеств}
\begin{scnsubdividing}
\scnitem{отношение}
\scnitem{класс структур}
\scnitem{класс классов}
\end{scnsubdividing}
\scntext{explanation}{\textbf{\textit{класс множеств}}  класс, элементами которого являются только \textit{sc-элементы}, являющиеся знаками множеств.}\scnheader{класс структур}
\scntext{explanation}{\textbf{\textit{класс структур}}  класс, элементами которого являются \textit{структуры}.}\scnheader{класс классов}
\scntext{explanation}{\textbf{\textit{класс классов}}  класс, элементами которого являются \textit{классы}.}\scnheader{нечеткое множество}
\scntext{explanation}{\textbf{\textit{нечеткое множество}}  это \textit{множество}, которое представляет собой совокупность элементов произвольной природы, относительно которых нельзя точно утверждать  обладают ли эти элементы некоторым характеристическим свойством, которое используется для задания этого нечеткого множества. Принадлежность элементов такому множеству указывается при помощи \textit{нечетких позитивных sc-дуг принадлежности}.}\scnheader{четкое множество}
\scntext{explanation}{\textbf{\textit{четкое множество}}  это \textit{множество}, принадлежность элементов которому достоверна и указывается при помощи \textit{четких позитивных sc-дуг принадлежности}.}\scnheader{множество первичных сущностей}
\scnsuperset{класс первичных сущностей}
\scnsubset{множество}
\scntext{explanation}{\textbf{\textit{множество первичных сущностей}}  это \textit{множество}, элементы которого не являются знаками множеств.}\scnheader{семейство множеств}
\scnidtf{множество множеств}
\scnsuperset{класс классов}
\scntext{explanation}{\textbf{\textit{семейство множеств}}  это \textit{множество}, элементами которого являются знаки множеств.}\scnheader{нерефлексивное множество}
\scntext{explanation}{\textbf{\textit{нерефлексивное множеств}}  это \textit{множество}, знак которого не является элементом этого множества}\scnheader{рефлексивное множество}
\scntext{explanation}{\textbf{\textit{рефлексивное множеств}}  это \textit{множество}, знак которого является элементом этого множества}\scnheader{множество первичных сущностей и множеств}
\scntext{explanation}{\textbf{\textit{множество первичных сущностей и множеств}}  это \textit{множество}, элементами которого являются как знаки множеств, так и знаки сущностей, не являющихся множествами.}\scnheader{сформированное множество}
\scniselement{ситуативное множество}
\scntext{explanation}{\textbf{\textit{сформированное множество }} - это \textit{множество}, все элементы которого известны и перечислены в данный момент.}\scnheader{несформированное множество}
\scniselement{ситуативное множество}
\scntext{explanation}{\textbf{\textit{несформированное множество}} - это \textit{множество}, не все элементы которого известны и перечислены в данный момент.}\scnheader{пустое множество}
\scniselement{мощность множества}
\scntext{explanation}{\textbf{\textit{пустое множество}}  это \textit{множество}, которому не принадлежит ни один элемент.}\scnheader{синглетон}
\scniselement{мощность множества}
\scnidtf{множество мощности 1}
\scnidtf{одноэлементное множество}
\scnidtf{одномощное множество}
\scnidtf{множество, мощность которого равна 1}
\scnidtf{множество, имеющее мощность равную единице}
\scnidtf{синглетон из sc-элемента}
\scnidtf{sc-синглеон}
\scnsubset{конечное множество}
\scntext{explanation}{\textbf{\textit{синглетон}}  это \textit{множество}, состоящее из одного элемента.Другими словами - любое множество \textit{Si} есть \textbf{\textit{синглетон}} тогда и только тогда, когда существует принадлежность этому множеству, которая совпадает с любой принадлежностью этому множеству.}\scnheader{пара}
\scniselement{мощность множества}
\scnidtf{множество мощности два}
\scnidtf{двухэлементное множество}
\scnidtf{двумощное множество}
\scnidtf{множество, мощность которого равна 2}
\scnidtf{sc-пара}
\scnidtf{пара sc-элементов}
\scnsubset{конечное множество}
\begin{scnsubdividing}
\scnitem{пара разных элементов}
\scnitem{пара-мультимножество}
\end{scnsubdividing}
\scntext{explanation}{\textbf{\textit{пара}}  это \textit{множество}, состоящее из двух элементов.Другими словами  любое множество есть \textbf{\textit{пара}} тогда и только тогда, когда существуют две различные принадлежности этому множеству такие, что любая принадлежность этому множеству совпадает с одной из них.}\scnheader{пара разных элементов}
\scnidtf{канторовская пара}
\scnidtf{канторовская пара sc-элементов}
\scnidtf{канторовское двумощное множество}
\scnheader{пара-мультимножество}
\scnidtf{пара-петля}
\scnidtf{sc-петля}
\scnidtf{двумощное мультимножество}
\scnheader{тройка}
\scniselement{мощность множества}
\scnidtf{тройка}
\scnidtf{sc-тройка}
\scnidtf{множество мощности три}
\scnidtf{множество, мощность которого равна 3}
\scnsubset{конечное множество}
\scntext{explanation}{\textbf{\textit{тройка}}  это \textit{множество}, состоящее из трех элементов.Другими словами  любое множество есть \textbf{\textit{тройка}} тогда и только тогда, когда существуют три различные принадлежности этому множеству такие, что любая принадлежность этому множеству совпадает с одной из них.}\scnheader{кортеж}
\scnidtf{вектор}
\scntext{explanation}{\textbf{\textit{кортеж}}  это множество, представляющее собой упорядоченный набор элементов, т.е. такое множество, порядок элементов в котором имеет значение. Пары принадлежности элементов \textbf{\textit{кортежу}} могут дополнительно принадлежать каким-либо \textit{ролевым отношениям}, при этом, в рамках каждого \textbf{\textit{кортежа}} должен существовать хотя бы один элемент, роль которого дополнительно уточнена \textit{ролевым отношением}.}\scnheader{пример\scnrolesign}
\scnidtf{типичный пример\scnrolesign}
\scnidtf{типичный экземпляр заданного класса\scnrolesign}
\scniselement{ролевое отношение}
\scntext{explanation}{\textbf{\textit{пример\scnrolesign}}  это \textit{ролевое отношение}, связывающее некоторое \textit{множество} с элементом, являющимся примером этого множества.}\scnheader{включение*}
\scnidtf{включение множеств*}
\scnidtf{быть подмножеством*}
\scniselement{бинарное отношение}
\scniselement{ориентированное отношение}
\scniselement{транзитивное отношение}
\scnrelfrom{область определения}{множество}
\scnsuperset{строгое включение*}
\scntext{определение}{\textbf{\textit{включение*}}  это бинарное ориентированное отношение, каждая связка которого связывает два множества. Будем говорить, что \textit{Множество Si} \textbf{\textit{включает*}} в себя \textit{Множество Sj} в том и только том случае, если каждый элемент \textit{Множества Sj} является также и элементом \textit{Множества Si}.}
\scnrelfrom{описание примера}{\scnfileimage[20em]{figures/sd_sets/inclusion.png}
}
\scntext{explanation}{Множество Sj}
\end{scnsubstruct}
\end{SCn}


\scsubsection[
    \protect\scnmonographychapter{Глава 2.4. Формальные онтологии базовых классов сущностей - множеств, связей, отношений, параметров, величин, чисел, структур, темпоральных сущностей}
    ]{Предметная область и онтология связок и отношений}
\label{sd_rels}
\begin{SCn}
\scnsectionheader{\currentname}
\begin{scnsubstruct}
\scnheader{Предметная область связок и отношений}
\scniselement{предметная область}
\begin{scnhaselementrole}{класс объектов исследования}
связь\end{scnhaselementrole}
\begin{scnhaselementrolelist}{класс объектов исследования}
\begin{itemize}
  \item бинарная связь
  \item sc-коннектор
  \item неатомарная бинарная связь
  \item небинарная связь
  \item неориентированная связь
  \item ориентированная связь
  \item отношение
  \item класс равномощных связок
  \item класс связок разной мощности
  \item унарное отношение
  \item бинарное отношение
  \item квазибинарное отношение
  \item тернарное отношение
  \item небинарное отношение
  \item ориентированное отношение
  \item неориентированное отношение
  \item рефлексивное отношение
  \item антирефлексивное отношение
  \item частично рефлексивное отношение
  \item симметричное отношение
  \item антисимметричное отношение
  \item частично симметричное отношение
  \item транзитивное отношение
  \item антитранзитивное отношение
  \item частично транзитивное отношение
  \item связанное отношение
  \item отношение порядка
  \item отношение строгого порядка
  \item отношение нестрогого порядка
  \item отношение толерантности
  \item отношение эквивалентности
  \item ролевое отношение
  \item числовой атрибут
  \item неролевое отношение
  \item неролевое бинарное отношение
  \item арность
  \item метаотношение
  \item отношение декомпозиции
  \item отношение интеграции
\end{itemize}
\end{scnhaselementrolelist}
\begin{scnhaselementrolelist}{исследуемое отношение}
\begin{itemize}
  \item область определения*\scnrolesign
  \item атрибут отношения*\scnrolesign
  \item домен*\scnrolesign
  \item первый домен*\scnrolesign
  \item второй домен*\scnrolesign
  \item композиция отношений*\scnrolesign
  \item фактор-множество*\scnrolesign
  \item соответствие*\scnrolesign
  \item отношение соответствия*\scnrolesign
  \item область отправления\scnrolesign
  \item область прибытия\scnrolesign
  \item образ\scnrolesign
  \item прообраз\scnrolesign
  \item всюду определенное соответствие*\scnrolesign
  \item частично определенное соответствие*\scnrolesign
  \item сюръективное соответствие*\scnrolesign
  \item несюръективное соответствие*\scnrolesign
  \item однозначное соответствие*\scnrolesign
  \item обратное соответствие*\scnrolesign
  \item обратимое соответствие*\scnrolesign
  \item неоднозначное соответствие*\scnrolesign
  \item инъективное соответствие*\scnrolesign
  \item взаимно однозначное соответствие*\scnrolesign
  \item множество сочетаний*\scnrolesign
  \item множество размещений*\scnrolesign
  \item множество перестановок*\scnrolesign
\end{itemize}
\end{scnhaselementrolelist}
\scnheader{связь}
\scnidtf{связка sc-элементов}
\scnidtf{sc-связка}
\scntext{explanation}{\textbf{\textit{связь}} -- множество, являющееся абстрактной моделью связи между описываемыми сущностями, которые или знаки которых являются элементами этого множества.}\scntext{примечание}{Напомним, что все элементы множества, представленного в SC-коде, являются знаками, но описываемыми сущностями могут быть не только сущности, обозначаемые sc-элементами, но и сами эти sc-элементы.}
\begin{scnsubdividing}
\scnitem{бинарная связь}
\scnitem{небинарная связь}
\end{scnsubdividing}
\begin{scnsubdividing}
\scnitem{неориентированная связь}
\scnitem{ориентированная связь}
\end{scnsubdividing}
\scnheader{бинарная связь}
\begin{scnsubdividing}
\scnitem{sc-коннектор}
\scnitem{неатомарная бинарная связь}
\end{scnsubdividing}
\scntext{примечание}{Данное разбиение осуществляется на основе синтаксического признака, а не семантического, поскольку каждый \textit{sc-коннектор} может быть записан в памяти при помощи семантически эквивалентной конструкции, содержащей знак самой связи и пары принадлежности, ведущие к ее элементам, уточненные, при необходимости ролевыми отношениями.}
\scnheader{sc-коннектор}
\scnidtf{атомарная бинарная связь}
\scntext{explanation}{Каждый \textbf{\textit{sc-коннектор}} представлен в \textit{sc-памяти} одним \textit{sc-элементом} и семантически эквивалентен конструкции, содержащей знак некоторой \textit{бинарной связи} и пары принадлежности, ведущие к элементам этой связи, уточненные, при необходимости ролевыми отношениями.Такая конструкция может быть обозначена \textbf{\textit{sc-коннектором}} только в случае, когда роли компонентов соответствующей бинарной связи указываются только при помощи \textit{числовых атрибутов 1\scnrolesign} и \textit{2\scnrolesign} или не уточняются вообще.}\scnheader{неатомарная бинарная связь}
\scntext{explanation}{\textbf{\textit{неатомарная бинарная связь}} -- \textit{бинарная связь}, роли компонентов которой не могут быть заданы только при помощи \textit{ролевых отношений 1\scnrolesign} и \textit{2\scnrolesign}, или не заданы совсем, а требуют дополнительного уточнения при помощи более частных ролевых отношений.}\scnheader{небинарная связь}
\scntext{explanation}{\textbf{\textit{небинарная связь}} -- связь, имеющая больше двух элементов.}\scnheader{неориентированная связь}
\scnsuperset{неориентированное множество}
\scntext{explanation}{\textbf{\textit{неориентированная связь}} -- связь, все элементы которой имеют одинаковые роли (при этом соответствующее ролевое отношение, как правило, явно не указывается).}\scnheader{ориентированная связь}
\scnsuperset{кортеж}
\scntext{explanation}{\textbf{\textit{ориентированная связь}} -- связь, в которой с помощью ролевых отношений, указываются роли компонентов этой связи.}\scnheader{отношение}
\scnidtf{класс связей}
\scnidtf{класс sc-связок}
\scnidtf{множество отношений}
\scnidtf{Множество всевозможных отношений}
\scntext{определение}{\textbf{\textit{отношение}}, \textit{заданное на множестве M} -- это подмножество \textit{декартового произведения} этого множества самого на себя некоторое количество раз.В более широком смысле \textbf{\textit{отношение}} -- это математическая структура, которая формально определяет свойства различных объектов и их взаимосвязи.}
\begin{scnsubdividing}
\scnitem{класс равномощных связок}
\scnitem{класс связок разной мощности}
\end{scnsubdividing}
\begin{scnsubdividing}
\scnitem{бинарное отношение}
\scnitem{небинарное отношение}
\end{scnsubdividing}
\begin{scnsubdividing}
\scnitem{ориентированное отношение}
\scnitem{неориентированное отношение}
\end{scnsubdividing}
\begin{scnsubdividing}
\scnitem{ролевое отношение}
\scnitem{неролевое отношение}
\end{scnsubdividing}
\scnheader{класс равномощных связок}
\scnidtf{класс связок фиксированной арности}
\scnidtf{отношение, обладающее свойством арности}
\scnsuperset{унарное отношение}
\scnsuperset{бинарное отношение}
\scnsuperset{тернарное отношение}
\scntext{определение}{\textbf{\textit{класс равномощных связок}} -- класс связок, имеющих одинаковую мощность.}
\scnheader{класс связок разной мощности}
\scnidtf{отношение нефиксированной арности}
\scnsubset{небинарное отношение}
\scntext{определение}{\textbf{\textit{класс связок разной мощности}} -- класс связок, имеющих разную мощность.}
\scnheader{унарное отношение}
\scnidtf{отношение арности один}
\scnidtf{одноместное отношение}
\scnidtf{множество синглетонов}
\scntext{определение}{\textbf{\textit{унарное отношение}} -- это множество таких отношений на множестве M, являющихся любым подмножеством множества M.}
\scnheader{бинарное отношение}
\scnidtf{отношение арности два}
\scnidtf{двухместное отношение}
\scnsuperset{квазибинарное отношение}
\scnsuperset{отношение порядка}
\scnsuperset{отношение толерантности}
\begin{scnsubdividing}
\scnitem{рефлексивное отношение}
\scnitem{антирефлексивное отношение}
\scnitem{частично рефлексивное отношение}
\end{scnsubdividing}
\begin{scnsubdividing}
\scnitem{симметричное отношение}
\scnitem{антисимметричное отношение}
\scnitem{частично симметричное отношение}
\end{scnsubdividing}
\begin{scnsubdividing}
\scnitem{транзитивное отношение}
\scnitem{антитранзитивное отношение}
\scnitem{частично транзитивное отношение}
\end{scnsubdividing}
\begin{scnsubdividing}
\scnitem{ролевое отношение}
\scnitem{неролевое бинарное отношение}
\end{scnsubdividing}
\scntext{определение}{\textbf{\textit{бинарное отношение}} -- это множество таких отношений на множестве \textbf{\textit{M}}, являющихся подмножеством \textit{декартова произведения} множества \textbf{\textit{M}}.\\Если \textbf{\textit{бинарное отношение R}} задано на \textit{множестве} \textbf{\textit{М}} и два элемента этого множества \textbf{\textit{a}} и \textbf{\textit{b}} связаны данным отношением, то будем обозначать такую связь как \textbf{\textit{aRb}}.}
\scnheader{квазибинарное отношение}
\scntext{explanation}{\textbf{\textit{квазибинарное отношение}} -- множество ориентированных пар, первые компоненты которых являются связками.\\Таким образом, \textit{sc-дуги}, принадлежащие \textbf{\textit{квазибинарным отношениям}}, всегда выходят из связок.}\scntext{sc-утверждение}{В область определения квазибинарного отношения будем включать:\begin{scnitemize}
\item вторые компоненты ориентированных пар, принадлежащих этому отношению;\item элементы первых компонентов ориентированных пар, принадлежащих этому отношению;\item других элементов область определения квазибинарного отношения не содержит.\end{scnitemize}
}
\scnheader{тернарное отношение}
\scnidtf{отношение арности три}
\scnidtf{трехместное отношение}
\scntext{explanation}{\textbf{\textit{тернарное отношение}} -- это множество отношений, связывающих между собой три элемента.}\scnheader{небинарное отношение}
\scntext{explanation}{\textbf{\textit{небинарное отношение}} -- это множество отношений, хотя бы одна из связок каждого из которых имеет значение мощности больше двух.}\scnheader{ориентированное отношение}
\scntext{определение}{\textbf{\textit{ориентированное отношение}} -- это множество таких отношений, каждая связка которых является кортежем.}
\scnheader{неориентированное отношение}
\scntext{определение}{\textbf{\textit{неориентированное отношение}} -- это множество таких отношений, каждая связка которых является неориентированным множеством.}
\scnheader{рефлексивное отношение}
\scntext{определение}{\textbf{\textit{рефлексивное отношение}} -- это \textit{бинарное отношение}, любая пара которого есть канторовское множество.}
\scnheader{антирефлексивное отношение}
\scntext{определение}{\textbf{\textit{антирефлексивное отношение R}} на \textit{множестве} \textbf{\textit{A}} -- это \textit{бинарное отношение}, в котором все элементы множества \textbf{\textit{A}} не находятся в отношении \textbf{\textit{R}} к самому себе.}
\scnheader{частично рефлексивное отношение}
\scntext{определение}{\textbf{\textit{частично рефлексивное отношение R}} на \textit{множестве} \textbf{\textit{A}} -- это \textit{бинарное отношение},  в котором хотя бы один (но не все) элемент множества \textbf{\textit{A}} находится в отношении \textbf{\textit{R}} к самому себе.}
\scnheader{симметричное отношение}
\scntext{определение}{\textbf{\textit{симметричное отношение R}} на \textit{множестве} \textbf{\textit{A}} -- это \textit{бинарное отношение}, в котором для каждой пары элементов \textbf{\textit{а}} и \textbf{\textit{b}} этого множества выполнение отношения \textbf{\textit{aRb}} влечёт выполнение \textbf{\textit{bRa}}.}
\scnheader{антисимметричное отношение}
\scntext{определение}{\textbf{\textit{антисимметричное отношение R}} на \textit{множестве} \textbf{\textit{A}} -- это \textit{бинарное отношение}, в котором для каждой пары элементов \textbf{\textit{а}} и \textbf{\textit{b}} этого множества выполнение отношений \textbf{\textit{aRb}} и \textbf{\textit{bRa}} влечёт равенство \textbf{\textit{a}} и \textbf{\textit{b}}.}
\scnheader{частично симметричное отношение}
\scntext{определение}{\textbf{\textit{частично симметричное отношение R}} на \textit{множестве} \textbf{\textit{A}} -- это \textit{бинарное отношение}, в котором для каждой пары элементов \textbf{\textit{а}} и \textbf{\textit{b}} (но не для всех таких пар) этого множества выполнение отношения \textbf{\textit{aRb}} влечёт выполнение \textbf{\textit{bRa}}.}
\scnheader{транзитивное отношение}
\scntext{определение}{\textbf{\textit{транзитивное отношение R}} на \textit{множестве} \textbf{\textit{A}} -- это \textit{бинарное отношение}, в котором для любых трёх элементов этого множества \textbf{\textit{a, b, c}} выполнение отношений \textbf{\textit{aRb}} и \textbf{\textit{bRc}} влечёт выполнение отношения \textbf{\textit{aRc}}.}
\scnheader{антитранзитивное отношение}
\scntext{определение}{\textbf{\textit{антитранзитивное отношение R}} на \textit{множестве} \textbf{\textit{A}} -- это \textit{бинарное отношение}, в котором для любых трёх элементов этого множества \textbf{\textit{a, b, c}} выполнение отношений \textbf{\textit{aRb}} и \textbf{\textit{bRc}} не влечёт выполнение отношения \textbf{\textit{aRc}}.}
\scnheader{частично транзитивное отношение}
\scntext{определение}{\textbf{\textit{частично транзитивное отношение R}} на \textit{множестве} \textbf{\textit{A}} -- это \textit{бинарное отношение}, в котором для каждых трёх элементов этого множества \textbf{\textit{a, b, c}} (но не для всех таких троек) выполнение отношений \textbf{\textit{aRb}} и \textbf{\textit{bRc}} влечёт выполнение отношения \textbf{\textit{aRc}}.}
\scnheader{связанное отношение*}
\scniselement{бинарное отношение}
\scntext{определение}{\textbf{\textit{связанное отношение* R}} на \textit{множестве} \textbf{\textit{A}} -- это \textit{бинарное отношение}, в котором для каждой пары элементов \textbf{\textit{а}} и \textbf{\textit{b}} этого множества выполняется одно из двух отношений: \textbf{\textit{aRb}} или \textbf{\textit{bRa}}.}
\scnheader{отношение порядка}
\begin{scnsubdividing}
\scnitem{отношение строгого порядка}
\scnitem{отношение нестрогого порядка}
\end{scnsubdividing}
\scntext{определение}{\textbf{\textit{отношение порядка}} -- это \textit{бинарное отношение}, обладающее свойством транзитивности и антисимметричности.}
\scnheader{отношение строгого порядка}
\scntext{определение}{\textbf{\textit{отношение строгого порядка}} -- это \textit{отношение порядка}, обладающее свойством антирефлексивности.}
\scnheader{отношение нестрогого порядка}
\scntext{определение}{\textbf{\textit{отношение нестрогого порядка}} -- это \textit{отношение порядка}, обладающее свойством рефлексивности.}
\scnheader{отношение толерантности}
\scntext{определение}{\textbf{\textit{отношение толерантности}} -- это \textit{бинарное отношение}, принадлежащее классам \textit{симметричное отношение} и \textit{рефлексивное отношение}.}
\scnheader{отношение эквивалентности}
\scnidtf{максимальное семейство отношений эквивалентности}
\scnsubset{отношение толерантности}
\scntext{определение}{\textbf{\textit{отношение эквивалентности}} -- это \textit{отношение толерантности}, принадлежащее классу \textit{транзитивных отношений}}
\scntext{примечание}{Каждое отношение эквивалентности уточняет то, что мы считаем эквивалентными сущностями, т.е. то, на какие сходства этих сущностей мы обращаем внимание и какие их отличия мы игнорируем (не учитываем).}
\scnheader{ролевое отношение}
\scnidtf{атрибут}
\scnidtf{атрибутивное отношение}
\scnidtf{отношение, которое задает роль элементов в рамках некоторого множества}
\scnidtf{отношение, являющееся подмножеством отношения принадлежности}
\scnrelto{семейство подмножеств}{принадлежность*}
\scnsubset{бинарное отношение}
\scnsuperset{числовой атрибут}
\scntext{explanation}{\textbf{\textit{ролевое отношение}} -- это отношение, являющееся подмножеством отношения принадлежности.}\scntext{правило идентификации экземпляров}{В конце каждого \textit{идентификатора}, соответствующего экземплярам класса \textbf{\textit{ролевое отношение}}, не являющегося системным, ставится знак ``\scnrolesign.Например:\\\textit{ключевой экземпляр\scnrolesign}Из-за ограничений в разрешенном алфавите символов, в системном идентификаторе не может быть использовать знак ``\scnrolesign, поэтому в начале каждого \textit{системного идентификатора}, соответствующего экземплярам класса \textbf{\textit{ролевое отношение}} ставится префикс ``rrel\_.Например:\\\textit{rrel\_key\_sc\_element}}
\scnheader{числовой атрибут}
\scnidtf{порядковый номер}
\scnidtf{номер компонента ориентированной связки}
\scnhaselement{\textbf{1\scnrolesign}; \textbf{2\scnrolesign}; \textbf{3\scnrolesign}; \textbf{4\scnrolesign}; \textbf{5\scnrolesign}; \textbf{6\scnrolesign}; \textbf{7\scnrolesign}; \textbf{8\scnrolesign}; \textbf{9\scnrolesign}; \textbf{10\scnrolesign}}
\scntext{explanation}{\textbf{\textit{числовой атрибут}} -- \textit{ролевое отношение}, задающее порядковый номер элемента некоторой ориентированной связки, не уточняя при этом семантику такой принадлежности. Во многих случаях бывает достаточно использовать числовые атрибуты, чтобы различать компоненты связки, семантика каждого из которых дополнительно оговаривается, например, при определении отношения, которому данная связка принадлежит.}\scnheader{неролевое отношение}
\begin{scnsubdividing}
\scnitem{небинарное отношение}
\scnitem{неролевое бинарное отношение}
\end{scnsubdividing}
\scntext{explanation}{\textbf{\textit{неролевое отношение}} -- отношение, не являющееся подмножеством отношения принадлежности.}\scntext{правило идентификации экземпляров}{В конце каждого \textit{идентификатора}, соответствующего экземплярам класса \textbf{\textit{неролевое отношение}}, не являющегося системным, ставится знак ``*.Например:\\\textit{включение*}Из-за ограничений в разрешенном алфавите символов, в системном идентификаторе не может быть использовать знак ``*, поэтому в начале каждого \textit{системного идентификатора}, соответствующего экземплярам класса \textbf{\textit{неролевое отношение}} ставится префикс ``nrel\_.Например:\\\textit{nrel\_inclusion}}
\scnheader{неролевое бинарное отношение}
\scntext{explanation}{\textbf{\textit{неролевое бинарное отношение}} -- \textit{бинарное отношение}, не являющееся \textit{ролевым отношением}.}\scnheader{арность}
\scnidtf{арность отношения}
\scniselement{параметр}
\scntext{explanation}{\textbf{\textit{арность}} -- это параметр, каждый элемент которого представляет собой класс \textit{отношений}, каждая связка которых имеет одинаковую \textit{мощность}. Значение данного \textit{параметра} совпадает со значением \textit{мощности} каждой из таких связок.}\scnrelfrom{описание примера}{\scnfileimage[20em]{figures/sd_relations/arity.png}
}
\scnheader{область определения*}
\scnidtf{область определения отношения*}
\scniselement{бинарное отношение}
\scntext{explanation}{\textbf{\textit{область определения*}} -- это \textit{бинарное отношение}, связывающее отношение со множеством, являющимся его областью определения.Областью определения отношения будем называть результат теоретико-множественного объединения всех связок этого отношения, или, другими словами, результат теоретико-множественного объединения всех множеств, являющихся доменами данного отношения.}\scnrelfrom{описание примера}{\scnfileimage[20em]{figures/sd_relations/domain.png}
}
\scnheader{атрибут отношения*}
\scnidtf{ролевой атрибут, используемый в связках заданного отношения*}
\scniselement{бинарное отношение}
\scntext{explanation}{\textbf{\textit{атрибут отношения*}} -- это \textit{бинарное отношение}, связывающее заданное отношение с \textit{ролевым отношением}, используемым в данном отношении для уточнения роли того или иного элемента связок данного отношения.}\scnrelfrom{описание примера}{\scnfileimage[20em]{figures/sd_relations/relationshipAttribute.png}
}
\newpage\scnheader{домен*}
\scnidtf{домен отношения по заданному атрибуту*}
\scniselement{бинарное отношение}
\scntext{explanation}{\textbf{\textit{домен*}} -- это \textit{бинарное отношение}, связывающее связку отношения \textit{атрибут отношения*} со множеством, являющимся доменом заданного отношения по заданному атрибуту. Множество \textbf{\textit{di}} является доменом отношения \textbf{\textit{ri}} по атрибуту \textbf{\textit{ai}} в том и только том случае, если элементами этого множества являются все те и только те элементы связок отношения \textbf{\textit{ri}}, которые имеют в рамках этих связок атрибут \textbf{\textit{ai}}.}\scnrelfrom{описание примера}{\scnfileimage[20em]{figures/sd_relations/domen.png}
}
\scnheader{первый домен*}
\scniselement{бинарное отношение}
\scntext{определение}{\textbf{\textit{первый домен*}} -- это \textit{бинарное отношение}, связывающее отношение с множеством, являющимся доменом по атрибуту 1 данного отношения.}
\scnheader{второй домен*}
\scniselement{бинарное отношение}
\scntext{определение}{\textbf{\textit{второй домен*}} -- это \textit{бинарное отношение}, связывающее отношение с множеством, являющимся доменом по атрибуту 2 данного отношения.}
\scnheader{композиция отношений*}
\scniselement{квазибинарное отношение}
\scntext{определение}{\textbf{\textit{композиция отношений*}} -- это \textit{квазибинарное отношение}, связывающее два бинарных отношения с отношением, являющимся их композицией. Под композицией бинарных отношений \textbf{\textit{R}} и \textbf{\textit{S}} будем понимать множество $\{(x, y) | \exists z(xSz \wedge zRy)\}$}
\scnrelfrom{описание примера}{\scnfileimage[20em]{figures/sd_relations/relationshipComposition.png}
}
\scnheader{фактор-множество*}
\scnidtf{быть фактор-множеством*}
\scnidtf{множество всевозможных максимальных множеств из попарно эквивалентных элементов*}
\scnidtf{множество всевозможных классов эквивалентности для заданного отношения эквивалентности*}
\scniselement{бинарное отношение}
\scntext{определение}{\textbf{\textit{фактор множество*}} -- это бинарное ориентированное отношение, каждая связка которого связывает некоторое отношение эквивалентности со множеством всех соответствующих этому отношению классов эквивалентности. Каждый такой класс представляет собой максимальное множество сущностей, каждая пара которых принадлежит указанному выше отношению эквивалентности.}
\scnrelfrom{описание примера}{\scnfileimage[20em]{figures/sd_relations/factor_set.png}
}
\scnheader{метаотношение}
\scntext{определение}{метаотношение -- это \textit{отношение}, в каждой связке которого есть по крайней мере один компонент, являющийся знаком некоторого \textit{отношения}.}
\scnheader{отношение декомпозиции}
\scnhaselement{разбиение*}
\scnhaselement{декомпозиция раздела*}
\scnhaselement{декомпозиция абстрактного объекта*}
\scnheader{отношение интеграции}
\scnhaselement{объединение*}
\scnheader{соответствие*}
\scnidtf{наличие соответствия*}
\scniselement{бинарное отношение}
\begin{scnsubdividing}
\scnitem{соответствие между непересекающимися множествами*}
\scnitem{соответствие между строго пересекающимися множествами*}
\scnitem{соответствие, область отправления и область прибытия которого совпадают*}
\end{scnsubdividing}
\begin{scnsubdividing}
\scnitem{всюду определенное соответствие*}
\scnitem{частично определенное соответствие*}
\end{scnsubdividing}
\begin{scnsubdividing}
\scnitem{сюръекция*}
\scnitem{несюръективное соответствие*}
\end{scnsubdividing}
\begin{scnsubdividing}
\scnitem{однозначное соответствие*}
\scnitem{неоднозначное соответствие*}
\end{scnsubdividing}
\scntext{определение}{\textbf{\textit{соответствие*}} -- \textit{бинарное ориентированное отношение}, каждая пара которого связывает два множества и указывает на наличие некоторого отношения, связывающего элементы этих двух множеств.}
\scnrelfrom{описание примера}{\scnfileimage[20em]{figures/sd_relations/conformity.png}
}
\scnheader{отношение соответствия*}
\scniselement{бинарное отношение}
\scntext{определение}{\textbf{\textit{отношение соответствия*}} -- \textit{бинарное отношение}, связывающее ориентированную пару множеств, на которых задано \textit{соответствие*} и некоторое подмножество \textit{декартова произведения*} этих \textit{множеств}.}
\scnrelfrom{описание примера}{\scnfileimage[20em]{figures/sd_relations/relationshipConformity.png}
}
\scnheader{область отправления\scnrolesign}
\scnidtf{область отправления соответствия\scnrolesign}
\scnidtf{область определения соответствия\scnrolesign}
\scnidtf{первый компонент пары в отношении соответствия\scnrolesign}
\scniselement{ролевое отношение}
\scntext{определение}{\textbf{\textit{область отправления\scnrolesign}} -- \textit{ролевое отношение}, указывающее на первый компонент пары в рамках отношения \textit{соответствие*}.}
\scnrelfrom{описание примера}{\scnfileimage[20em]{figures/sd_relations/departureArea.png}
}
\scnheader{область прибытия\scnrolesign}
\scnidtf{область прибытия соответствия\scnrolesign}
\scnidtf{область значений соответствия\scnrolesign}
\scniselement{ролевое отношение}
\scntext{определение}{\textbf{\textit{область прибытия\scnrolesign}} -- \textit{ролевое отношение}, указывающее на второй компонент пары в рамках отношения \textit{соответствие*}.}
\scnrelfrom{описание примера}{\scnfileimage[20em]{figures/sd_relations/arrivalArea.png}
}
\scnheader{образ\scnrolesign}
\scnidtf{образ соответствия\scnrolesign}
\scniselement{ролевое отношение}
\scntext{определение}{\textbf{\textit{образ\scnrolesign}} -- \textit{ролевое отношение}, указывающее на второй компонент каждой пары в рамках множества пар, которое является вторым компонентом \textit{отношения соответствия*}.}
\scnrelfrom{описание примера}{\scnfileimage[20em]{figures/sd_relations/form.png}
}
\newpage\scnheader{прообраз\scnrolesign}
\scnidtf{прообраз соответствия\scnrolesign}
\scniselement{ролевое отношение}
\scntext{определение}{\textbf{\textit{прообраз\scnrolesign}} -- \textit{ролевое отношение}, указывающее на первый компонент каждой пары в рамках множества пар, которое является первым компонентом \textit{отношения соответствия*}.}
\scnrelfrom{описание примера}{\scnfileimage[20em]{figures/sd_relations/prototype.png}
}
\scnheader{всюду определенное соответствие*}
\scnidtf{полное соответствие*}
\scnidtf{наличие всюду определенного соответствия*}
\scntext{определение}{\textbf{\textit{всюду определенное соответствие*}} -- это \textit{соответствие*}, при котором существует \textit{образ\scnrolesign} для каждого элемента \textit{области отправления\scnrolesign} данного \textit{соответствия*}.}
\scnrelfrom{описание примера}{\scnfileimage[20em]{figures/sd_relations/surjection.png}
}
\scnrelfrom{изображение}{\scnfileimage{[20em]{figures/sd_relations/surjection2.png}}
}
\scnheader{частично определенное соответствие*}
\scnidtf{наличие частично определенного соответствия*}
\scntext{определение}{\textbf{\textit{частично определенное соответствие*}} -- это \textit{соответствие*}, при котором существует \textit{образ\scnrolesign} для некоторых, но не всех элементов \textit{области отправления\scnrolesign} данного \textit{соответствия*}.}
\scnrelfrom{описание примера}{\scnfileimage[20em]{figures/sd_relations/partiallyDefinedConformity.png}
}
\scnrelfrom{изображение}{\scnfileimage{[20em]{figures/sd_relations/partiallySurjection.png}}
}
\newpage\scnheader{сюръективное соответствие*}
\scnidtf{наличие сюръективного соответствия*}
\scnidtf{сюръекция*}
\scntext{определение}{\textbf{\textit{сюръективное соответствие*}} -- это \textit{соответствие*}, при котором существует \textit{прообраз\scnrolesign} для каждого элемента \textit{области прибытия\scnrolesign} данного \textit{соответствия*}.}
\scnrelfrom{описание примера}{\scnfileimage[20em]{figures/sd_relations/surjectiveConformity.png}
}
\scnrelfrom{изображение}{\scnfileimage{[20em]{figures/sd_relations/surjectiveConformity2.png}}
}
\scnheader{несюръективное соответствие*}
\scnidtf{наличие несюръективного соответствия*}
\scntext{определение}{\textbf{\textit{несюръективное соответствие*}} -- это \textit{соответствие*}, при котором не для каждого элемента \textit{области прибытия\scnrolesign} данного \textit{соответствия*} существует \textit{прообраз\scnrolesign}.}
\scnrelfrom{описание примера}{\scnfileimage[20em]{figures/sd_relations/nonSurjectiveConformity.png}
}
\scnrelfrom{изображение}{\scnfileimage{[20em]{figures/sd_relations/nonSurjectiveConformity2.png}}
}
\scnheader{однозначное соответствие*}
\scnidtf{наличие однозначного соответствия*}
\scnidtf{функциональное соответветствие*}
\scnidtf{функция*}
\scntext{определение}{\textbf{\textit{однозначное соответствие*}} -- это \textit{соответствие*}, при котором каждому элементу из \textit{области отправления\scnrolesign} соответствия ставится не более, чем один элемент из \textit{области прибытия\scnrolesign} соответствия.}
\scnrelfrom{описание примера}{\scnfileimage[20em]{figures/sd_relations/singleConformity.png}
}
\scnrelfrom{изображение}{\scnfileimage{[20em]{figures/sd_relations/singleConformity2.png}}
}
\scnheader{обратное соответствие*}
\scniselement{бинарное отношение}
\scnrelfrom{область определения}{соответствие*}
\scntext{определение}{\textbf{\textit{обратное соответствие*}} -- \textit{бинарное отношение}, связывающее два \textit{соответствия*}, при этом выполняются следующие условия:\begin{scnitemize}
\item \textit{область отправления\scnrolesign} первого соответствия является \textit{областью прибытия\scnrolesign} второго;\item \textit{область прибытия\scnrolesign} первого соответствия является \textit{областью отправления\scnrolesign} второго;\item для каждой пары, входящей в состав отношения первого соответствия, существует пара, входящая в состав отношения второго соответствия, при этом \textit{образ\scnrolesign} и \textit{прообраз\scnrolesign} в рамках первой указанной пары являются соответственно \textit{прообразом\scnrolesign} и \textit{образом\scnrolesign} в рамках второй.\end{scnitemize}
}
\scnheader{обратимое соответствие*}
\scnsubset{однозначное соответствие*}
\scntext{определение}{\textbf{\textit{обратимое соответствие*}} -- такое \textit{однозначное соответствие*}, для которого \textit{обратное соответствие*} также является \textit{однозначным соответствием*}.}
\newpage\scnheader{неоднозначное соответствие*}
\scntext{определение}{\textbf{\textit{неоднозначное соответствие*}} -- это \textit{соответствие*}, при котором хотя бы одному элементу из \textit{области отправления\scnrolesign} соответствия ставится более, чем один элемент из \textit{области прибытия\scnrolesign} соответствия.}
\scnrelfrom{описание примера}{\scnfileimage[20em]{figures/sd_relations/nonSingleConformity.png}
}
\scnrelfrom{изображение}{\scnfileimage{[20em]{figures/sd_relations/nonSingleConformity2.png}}
}
\scnheader{инъективное соответствие*}
\scnidtf{инъекция*}
\scnsubset{однозначное соответствие*}
\scntext{определение}{\textbf{\textit{инъективное соответствие*}} -- это \textit{соответствие*}, при котором разным элементам из \textit{области отправления\scnrolesign} соответствия всегда соответствуют разные элементы из \textit{области прибытия\scnrolesign} соответствия и наоборот.}
\scnrelfrom{описание примера}{\scnfileimage[20em]{figures/sd_relations/injectiveConformity.png}
}
\scnrelfrom{изображение}{\scnfileimage{[20em]{figures/sd_relations/injectiveConformity2.png}}
}
\scnheader{взаимно однозначное соответствие*}
\scnidtf{биекция*}
\scnsubset{всюду определенное соответствие*}
\scnsubset{сюръективное соответствие*}
\scnsubset{инъективное соответствие*}
\scntext{определение}{\textbf{\textit{взаимно однозначное соответствие*}} -- это \textit{инъективное соответствие*}, являющееся всюду определенным и сюръективным.}
\scnrelfrom{описание примера}{\scnfileimage[20em]{figures/sd_relations/bijectiveConformity.png}
}
\scnrelfrom{изображение}{\scnfileimage{[20em]{figures/sd_relations/bijectiveConformity2.png}}
}
\scnheader{множество сочетаний*}
\scnidtf{множество всевозможных сочетаний*}
\scnidtf{множество всевозможных сочетаний заданной арности из элементов заданного множества*}
\scnidtf{множество всех неориентированных связок заданной арности*}
\scnidtf{множество всех подмножеств заданной мощности*}
\scnidtf{семейство всевозможных сочетаний*}
\scntext{определение}{\textbf{\textit{множество сочетаний*}} -- \textit{отношение}, связывающее некоторое множество и семейство всевозможных множеств, имеющих значение мощности, меньше либо равное мощности исходного множества и состоящих из тех же элементов, что и это множество.}
\scntext{утверждение}{Мощность \textbf{\textit{множества сочетаний*}} может быть вычислена как \textbf{n!/(k!(n-k)!)}, где \textbf{\textit{n}} -- мощность исходного множества, \textbf{\textit{k}} -- мощность элементов множества сочетаний.}
\scnrelfrom{описание примера}{\scnfileimage[20em]{figures/sd_relations/setsOfCombinations.png}
\scntext{explanation}{Для Множества \textbf{\textit{Si}} представлено множество сочетаний по 2 элемента.}}
\scnheader{множество размещений*}
\scntext{определение}{\textbf{\textit{множество размещений*}} -- \textit{отношение}, связывающее некоторое множество и семейство всевозможных кортежей, имеющих значение мощности, меньше либо равное мощности исходного множества и состоящих из тех же элементов, что и это множество.}
\scntext{утверждение}{Мощность \textbf{\textit{множества размещений*}} может быть вычислена как \textbf{n!/(n-k)!}, где \textbf{\textit{n}} -- мощность исходного множества, \textbf{\textit{k}} -- мощность элементов множества сочетаний.}
\scnrelfrom{описание примера}{\scnfileimage[20em]{figures/sd_relations/setsOfPlacements.png}
\scntext{explanation}{Для Множества \textbf{\textit{Si}} представлено множество размещений по 2 элемента.}}
\scnheader{множество перестановок*}
\scnsubset{множество размещений*}
\scntext{определение}{\textbf{\textit{множество перестановок*}} -- \textit{отношение}, связывающее некоторое множество и семейство всевозможных кортежей, равномощных исходному множеству и состоящих из тех же элементов, что и это множество.}
\scntext{утверждение}{Мощность \textbf{\textit{множества перестановок*}} может быть вычислена как \textbf{n!}, где \textbf{\textit{n}} -- мощность исходного множества.}
\scnrelfrom{описание примера}{\scnfileimage[20em]{figures/sd_relations/setsOfPermutations.png}
\scntext{explanation}{Для Множества \textbf{\textit{Si}} представлено его множество перестановок.}}
\bigskip
\end{scnsubstruct}
\end{SCn}


\scsubsection[
    \protect\scnmonographychapter{Глава 2.4. Формальные онтологии базовых классов сущностей - множеств, связей, отношений, параметров, величин, чисел, структур, темпоральных сущностей}
    ]{Предметная область и онтология параметров, величин и шкал}
\label{sd_params}
\begin{SCn}
\scnsectionheader{\currentname}
\begin{scnsubstruct}
\scnheader{Предметная область параметров, величин и шкал}
\scnidtf{Предметная область параметров и классов эквивалентности, являющихся их элементами (значениями, величинами)}
\scniselement{предметная область}
\begin{scnhaselementrole}{класс объектов исследования}
параметр\end{scnhaselementrole}
\begin{scnhaselementrolelist}{класс объектов исследования}

измеряемый параметр;неизмеряемый параметр;уровень класса эквивалентности;величина;точная величина;неточная величина;интервальная величина;параметрическая модель;измерение с фиксированной единицей измерения ;измерение по шкале;арифметическое выражение на величинах;арифметическая операция на величинах;действие. измерение;задача. измерение

\end{scnhaselementrolelist}
\begin{scnhaselementrolelist}{исследуемое отношение}

область определения параметра*;эталон\scnrolesign;измерение*;точность*;единица измерения*;нулевая отметка*;единичная отметка*;сумма величин*;произведение величин*;возведение величин в степень*;большая величина*;равенство величин*;большая или равная величина*

\end{scnhaselementrolelist}
\scnheader{параметр}
\scnidtf{характеристика}
\scnidtf{свойство}
\scnidtf{признак}
\scnidtf{класс классов}
\scnidtf{измеряемое свойство}
\scnidtf{признак классификации или покрытия некоторого класса сущностей}
\scnidtf{признак разбиения или покрытия некоторого класса сущностей}
\scnidtf{семейство множеств, разбивающих или покрывающих некоторый класс сущностей}
\scnidtf{семейство классов сущностей, обладающих одинаковым соответствующим свойством}
\scnidtf{фактор-множество, соответствующее некоторому отношению эквивалентности, или аналог фактор-множества, соответствующий некоторому отношению толерантности}
\begin{scnsubdividing}

\scnitem{измеряемый параметр}
\scnitem{неизмеряемый параметр}

\end{scnsubdividing}
\scnsuperset{ориентированный параметр}
\scntext{explanation}{Каждый \textbf{\textit{параметр}} представляет собой класс, являющийся семейством всевозможных классов эквивалентности или толерантности, задаваемых либо \textit{отношением эквивалентности}, либо \textit{отношением толерантности} (симметричным, рефлексивным, но частично транзитивным). Так, например, элементами (значениями, величинами) \textbf{\textit{параметра}} \textit{длина} являются либо классы эквивалентности, задаваемые отношением эквивалентности ``иметь точно одинаковую длину*, либо классы толерантности, задаваемые отношением вида ``иметь приблизительно одинаковую длину с указываемой точностью*, либо интервальные классы, задаваемые бинарными отношениями вида ``иметь длину, находящуюся в одном и том же указываемом интервале* (например, от 1 метра до 2 метров).\\Примерами параметров как отношений эквивалентности являются:\begin{scnitemize}
\item равновеликость геометрических фигур (по длине, площади, объему -- в зависимости от размерности этих фигур);\item иметь одинаковый цвет (быть эквивалентными по цвету);\item эквивалентность, по вкусу, запаху, твердости и т.д.\end{scnitemize}
Заметим, что среди элементов (значений, величин) параметра могут встречаться пересекающиеся множества (классы), но объединение всех элементов каждого параметра есть не что иное, как класс всевозможных сущностей, обладающих этим параметром (свойством, характеристикой). Например, класс всех сущностей, имеющих длину, класс всех сущностей, обладающих цветом.Каждый конкретный параметр (характеристика), т.е. каждый элемент класса всевозможных параметров (характеристик) есть, по сути, признак классификации сущностей, обладающих это характеристикой, по принципу эквивалентности (одинаковости значения) этой характеристики. Например, параметр \textit{цвет} разбивает множество сущностей имеющих цвет на классы, каждый из которых включает в себя сущности, имеющие одинаковый цвет. Параметр может разбиваться на классы для уточнения некоторого свойства, например элементами параметра цвет будут классы, соответствующие конкретным цветам (синий, красный и т.д.), в свою очередь каждый конкретный цвет может включать более частные классы, уточняющие данное свойство, например, темно-синий, светло-красный и т.д.Другими словами, каждому множеству сущностей может ставиться в соответствие набор (семейство) параметров (параметрическое пространство), которыми обладают сущности этого множества -- аналог семейства отношений, определенных (заданных) на этом множестве. Часто бывает важно построить такое параметрическое пространство, точки}
\end{scnsubstruct}
\end{SCn}


\scsubsection[
    \protect\scnmonographychapter{Глава 2.4. Формальные онтологии базовых классов сущностей - множеств, связей, отношений, параметров, величин, чисел, структур, темпоральных сущностей}
    ]{Предметная область и онтология чисел и числовых структур}
\begin{SCn}
\scnsectionheader{\currentname}
\begin{scnsubstruct}
\scnheader{Предметная область чисел и числовых структур}
\scniselement{предметная область}
\begin{scnhaselementrole}{класс объектов исследования}
число\end{scnhaselementrole}
\begin{scnhaselementrolelist}{класс объектов исследования}
\begin{itemize}
  \item натуральное число
  \item целое число
  \item рациональное число
  \item иррациональное число
  \item действительное число
  \item комплексное число
  \item отрицательное число
  \item положительное число
  \item арифметическое выражение
  \item арифметическая операция
  \item Число Пи
  \item Нуль
  \item Единица
  \item Мнимая единица
  \item числовая структура
  \item система счисления
  \item десятичная система счисления
  \item двоичная система счисления
  \item шестнадцатеричная система счисления
  \item дробь
  \item обыкновенная дробь
  \item десятичная дробь
  \item цифра
  \item арабская цифра
  \item римская цифра
\end{itemize}
\end{scnhaselementrolelist}
\begin{scnhaselementrolelist}{исследуемое отношение}
противоположные числа*;модуль*;сумма*;произведение*;возведение в степень*;больше*;равенство*;больше или равно*
\end{scnhaselementrolelist}
\scnheader{число}
\scnidtf{множество чисел}
\scnsubset{абстрактная терминальная сущность}
\scntext{explanation}{\textbf{\textit{число}} -- это основное понятие математики, используемое для количественной характеристики, сравнения, нумерации объектов и их частей. Письменными знаками для обозначения чисел служат \textit{цифры}.}\scnheader{цифра}
\scnidtf{множество цифр}
\scnsubset{внутренний файл ostis-системы}
\begin{scnrelfromlist}{включение}
\scnitem{арабская цифра}
\scnitem{римская цифра}
\end{scnrelfromlist}
\scntext{explanation}{\textbf{\textit{цифра}} - это множество файлов, обозначающих вхождение этой цифры во всевозможные записи чисел с помощью этой цифры.}\scnheader{натуральное число}
\scnidtf{множество натуральных чисел}
\scntext{explanation}{\textbf{\textit{натуральное число}} -- это подмножество множества \textit{целых чисел}, которые используются при счете предметов.}\scnsubset{целое число}
\scnheader{целое число}
\scnidtf{множество целых чисел}
\scntext{explanation}{\textbf{\textit{целое число}} -- это подмножество множества \textit{рациональных чисел}, получаемых объединением \textit{натуральных чисел} с множеством чисел, \textit{противоположных* натуральным} и \textit{нулём}.}\scnsubset{рациональное число}
\scnheader{рациональное число}
\scnidtf{множество рациональных чисел}
\scntext{explanation}{\textbf{\textit{рациональное число}} -- это число, представляемое \textit{обыкновенной дробью}, где числитель  \textit{целое число}, а знаменатель  \textit{натуральное число}.}\scnsubset{действительное число}
\scnheader{дробь}
\scnidtf{множество дробей}
\begin{scnrelfromlist}{включение}
\scnitem{обыкновенная дробь}
\scnitem{ десятичная дробь}
\end{scnrelfromlist}
\scntext{explanation}{\textbf{\textit{дробь}}  это число, состоящее из одной или нескольких равных частей (долей) единицы}\scnidtf{множество обыкновенных дробей}
\scnidtf{множество простых дробей}
\scntext{explanation}{}\scnheader{десятичная дробь}
\scnidtf{множество десятичных дробей}
\scntext{explanation}{\textbf{\textit{десятичная дробь}} - Десятичная дробь  разновидность дроби, которая представляет собой способ представления действительных чисел в виде , где ,  десятичная запятая, служащая разделителем между целой и дробной частью числа,  десятичные цифры.}\scnheader{иррациональное число}
\scnidtf{множество иррациональных чисел}
\scntext{explanation}{\textbf{\textit{иррациональное число}} -- это \textit{вещественное число}, которое не является рациональным, то есть не может быть представлено в виде дроби, где числитель  \textit{целое число}, знаменатель  \textit{натуральное число}. Любое \textbf{\textit{иррациональное число}} может быть представлено в виде бесконечной непериодической десятичной дроби.}\scnsubset{действительное число}
\scnheader{действительное число}
\scnidtf{вещественное число}
\scnidtf{множество вещественных чисел}
\begin{scnreltoset}{объединение}
\scnitem{рациональное число}
\scnitem{иррациональное число}
\end{scnreltoset}
\begin{scnsubdividing}
\scnitem{положительное число}
\scnitem{отрицательное число}
\scnitem{$\{$Нуль$\}$}
\end{scnsubdividing}
\scntext{explanation}{\textbf{\textit{действительное число}} -- это множество чисел, получаемое в результате объединения иррациональных и \textit{рациональных чисел}.}\scnsubset{комплексное число}
\scnheader{комплексное число}
\scnidtf{множество комплексных чисел}
\scntext{explanation}{\textbf{\textit{комплексное число}} -- число вида \textit{z=a+b*i}, где \textit{a} и \textit{b} -- \textit{вещественные числа}, \textit{i} -- \textit{Мнимая единица}.}\scnheader{отрицательное число}
\scnidtf{множество отрицательных чисел}
\scntext{explanation}{\textbf{\textit{отрицательное число}} -- число \textit{меньше*} нуля.}\scnheader{положительное число}
\scnidtf{множество положительных чисел}
\scntext{explanation}{\textbf{\textit{положительное число}} -- число \textit{больше*} нуля.}\scnheader{противоположные числа*}
\scniselement{бинарное неориентированное отношение}
\scntext{explanation}{\textbf{\textit{противоположные числа*}} -- \textit{отношение}, связывающее два числа, одно из которых является \textit{отрицательным числом}, второе -- \textit{положительным}, при этом \textit{модули*} этих чисел \textit{равны*}.}\scnheader{модуль*}
\scnidtf{модуль числа*}
\scniselement{бинарное отношение}
\scntext{explanation}{Связки отношения \textbf{\textit{модуль*}} связывают некоторое \textit{число} (которое может быть как \textit{отрицательным}, так и \textit{положительным}) и другое \textit{число} (всегда \textit{положительное}), которое выражает расстояние от указанного числа до \textit{Нуля} в единицах.}\scnheader{арифметическое выражение}
\scnidtf{множество арифметических выражений}
\scntext{explanation}{Каждое \textbf{\textit{арифметическое выражение}} представляет собой \textit{связку}, компонентами которой являются \textit{числа} или множества \textit{чисел}.}\scnheader{арифметическая операция}
\scnidtf{множество арифметических операций}
\scnrelto{семейство подмножеств}{арифметическое выражение}
\scntext{explanation}{Каждая \textbf{\textit{арифметическая операция}} представляет собой \textit{отношение}, элементами которого являются \textit{арифметические выражения}, то есть множество \textit{арифметических выражений} какого-либо одного вида.}\scnheader{сумма*}
\scnidtf{сложение*}
\scniselement{арифметическая операция}
\scniselement{квазибинарное отношение}
\scntext{explanation}{\textbf{\textit{сумма*}} -- это арифметическая операция, в результате которой по данным числам (слагаемым) находится новое число (сумма), обозначающее столько единиц, сколько их содержится во всех слагаемых.Первым компонентом связки отношения \textbf{\textit{сумма*}} является \textit{множество чисел} (слагаемых), содержащее два или более элемента, вторым компонентом -- \textit{число}, являющееся результатом сложения.Отдельно отметим, что каждая связка отношения \textbf{\textit{сумма*}} вида a = b+c может также трактоваться и как запись о вычитании чисел, например b = a-c, в связи с чем \textit{арифметическая операция} разности чисел отдельно не вводится.}\scnrelfrom{описание примера}{\scnfileimage[20em]{figures/sd_numbers/sum.png}
}
\scnheader{произведение*}
\scnidtf{умножение*}
\scniselement{арифметическая операция}
\scniselement{квазибинарное отношение}
\scntext{explanation}{\textbf{\textit{произведение*}} -- это \textit{арифметическая операция}, в результате которой один аргумент складывается столько раз, сколько показывает другой, затем результат складывается столько раз, сколько показывает третий и т.д.Первым компонентом связки отношения \textbf{\textit{произведение*}} является \textit{множество чисел} (множителей), содержащее два или более элемента, вторым компонентом -- \textit{число}, являющееся результатом произведения.Отдельно отметим, что каждая связка отношения \textbf{\textit{произведение*}} вида a = b*c может также трактоваться и как запись о делении чисел, например b = a/c, в связи с чем \textit{арифметическая операция} деления чисел отдельно не вводится.}\scnrelfrom{описание примера}{\scnfileimage[20em]{figures/sd_numbers/multiplication.png}
}
\scnheader{возведение в степень*}
\scniselement{арифметическая операция}
\scniselement{бинарное отношение}
\scntext{explanation}{\textbf{\textit{возведение в степень*}} -- это \textit{арифметическая операция}, в результате которой число, называемое основанием степени, умножается само на себя столько раз, каков показатель степени.Первым компонентом связки отношения \textbf{\textit{возведение в степень*}} является ориентированная пара, первым компонентом которой является \textit{число}, которое является основанием степени, вторым -- \textit{число}, которое является показателем степени; Вторым компонентом связки отношения \textbf{\textit{возведение в степень*}} является \textit{число}, которое является результатом возведения в степень.Отдельно отметим, что каждая связка отношения \textbf{\textit{возведение в степень*}} вида a = $b^c$ может также трактоваться и как запись об извлечении корня или взятии логарифма, в связи с чем \textit{арифметические операции} извлечения корня и взятия логарифма отдельно не вводится.}\scnrelfrom{описание примера}{\scnfileimage[20em]{figures/sd_numbers/pow.png}
}
\scnheader{больше*}
\scniselement{бинарное отношение}
\scniselement{отношение строгого порядка}
\scntext{explanation}{\textbf{\textit{больше*}} -- это \textit{бинарное отношение} сравнения чисел. Из двух чисел на координатной прямой больше то, которое расположено правее. Соответственно, первым компонентом связки \textit{отношения} \textbf{\textit{больше*}} является большее из двух \textit{чисел}.}\scnrelfrom{описание примера}{\scnfileimage[20em]{figures/sd_numbers/more.png}
}
\scnheader{больше или равно*}
\scniselement{бинарное отношение}
\scniselement{отношение нестрогого порядка}
\scntext{explanation}{\textbf{\textit{больше или равно*}} -- это \textit{бинарное отношение} сравнения чисел, при которой первое \textit{число} (первый компонент связки) может быть \textit{больше*} второго или \textit{равняться*} ему.}\scntext{note}{Отношение \textit{равенство*} явно не вводится, поскольку в рамках SC-кода равные числа трактуются как совпадающие числа, то есть обозначаемые одним и тем же \textit{sc-элементом}.}\scnrelfrom{описание примера}{\scnfileimage[20em]{figures/sd_numbers/more_or_equal.png}
}
\scnheader{Число Пи}
\scniselement{иррациональное число}
\scntext{explanation}{\textbf{\textit{Число Пи}} -- это  математическая константа, равная отношению длины окружности к длине её диаметра.}\scnheader{Нуль}
\scnidtf{0}
\scniselement{целое число}
\scntext{explanation}{\textbf{\textit{Нуль}} -- это \textit{целое число}, разделяющее на числовой прямой \textit{положительные числа} и \textit{отрицательные числа}.}\scnheader{Единица}
\scnidtf{1}
\scniselement{целое число}
\scniselement{натуральное число}
\scntext{explanation}{\textbf{\textit{Единица}} -- это наименьшее \textit{натуральное число}.}\scnheader{Мнимая единица}
\scnidtf{i}
\scniselement{комплексное число}
\scntext{explanation}{\textbf{\textit{Мнимая единица}} -- это \textit{число}, при возведении которого в степень 2 результатом будет число, противоположное \textit{Единице}.}\scnheader{числовая структура}
\scnsubset{структура}
\scntext{explanation}{\textbf{\textit{числовая структура}} -- \textit{структура}, в состав которой входят знаки \textit{арифметических выражений}, а также знаки их элементов и связи между выражениями и их элементами.}\scnheader{система счисления}
\scniselement{параметр}
\scntext{explanation}{Каждая \textbf{\textit{система счисления}} представляет собой класс синтаксически эквивалентных файлов, хранимых в sc-памяти, каждый из которых может являться идентификатором какого-либо \textit{числа}.Каждая \textbf{\textit{система счисления}} характеризуется алфавитом, т.е. конечным множеством символов (цифр), которые допускается использовать при построении файлов принадлежащих данной \textbf{\textit{системе счисления}}.}\scnheader{десятичная система счисления}
\scniselement{система счисления}
\scnheader{двоичная система счисления}
\scniselement{система счисления}
\scnheader{шестнадцатеричная система счисления}
\scniselement{система счисления}
\bigskip
\end{scnsubstruct}
\end{SCn}

\scsubsection[
    \protect\scnmonographychapter{Глава 2.3. Структура баз знаний интеллектуальных компьютерных систем нового поколения: иерархическая система предметных областей и онтологий. Онтологии верхнего уровня. Формализация понятий семантической окрестности, предметной области и онтологии в интеллектуальных компьютерных системах нового поколения}
    ]{Предметная область и онтология структур}
\label{sd_structures}
\begin{SCn}
\scnsectionheader{\currentname}
\begin{scnsubstruct}
\scnheader{Предметная область структур}
\begin{scnhaselementrole}{класс объектов исследования}
структура\end{scnhaselementrole}
\begin{scnhaselementrolelist}{класс объектов исследования}

связная структура;несвязная структура;тривиальная структура;нетривиальная структура;структура второго уровня;семантический уровень структурного элемента;количество семантических уровней элементов структуры

\end{scnhaselementrolelist}
\begin{scnhaselementrolelist}{исследуемое отношение}

элемент структуры\scnrolesign;непредставленное множество\scnrolesign;полностью представленное множество\scnrolesign;частично представленное множество\scnrolesign;элемент структуры, не являющийся множеством\scnrolesign;максимальное множество\scnrolesign;немаксимальное множество\scnrolesign;первичный элемент\scnrolesign;вторичный элемент\scnrolesign;элемент второго уровня\scnrolesign;метасвязь\scnrolesign;полиморфность*;полиморфизм*;гомоморфность*;гомоморфизм*;изоморфность*;изоморфизм*;автоморфность*;автоморфизм*;аналогичность структур*;сходство*;различие*;первичная синтаксическая структура sc-текста*

\end{scnhaselementrolelist}
\scnheader{структура}
\scnidtf{sc-структура}
\scnidtf{структура, представленная в виде текста SC-кода}
\begin{scnsubdividing}

\scnitem{связная структура}
\scnitem{несвязная структура}

\end{scnsubdividing}
\begin{scnsubdividing}

\scnitem{тривиальная структура}
\scnitem{нетривиальная структура}

\end{scnsubdividing}
\scntext{explanation}{\textbf{\textit{структура}} -- множество \textit{sc-элементов}, удаление одного из которых может привести к нарушению целостности этого множества.}\begin{scnsubdividing}

\scnitem{процесс\\\scnidtf{динамическая структура}
\scnidtf{нестационарная структура}
}
\scnitem{статическая структура\\\scnidtf{стационарная структура}
\scnidtf{структура, не изменяющаяся во времени}
}

\end{scnsubdividing}
\begin{scnsubdividing}

\scnitem{временная структура\\\scnidtf{(структура $\cap$ временная сущность)}
}
\scnitem{постоянно существующая структура\\\scnidtf{(структура $\cap$ постоянная сущность)}
}

\end{scnsubdividing}
\scnheader{связная структура}
\scntext{explanation}{\textit{Структуре}, представленной в \textit{SC-коде}, поставим в соответствие орграф, вершинами которого являются \textit{\mbox{sc-элементы}}, а дугами -- пары инцидентности, связывающие \textit{sc-коннекторы} с инцидентными им \textit{\mbox{sc-элементами}}, которые являются компонентами указанных \textit{sc-коннекторов}.Если полученный таким способом орграф является связным орграфом, то исходную структуру будем считать \textbf{\textit{связной структурой}}.}\scnheader{несвязная структура}
\scntext{explanation}{\textit{Структуре}, представленной в \textit{SC-коде}, поставим в соответствие орграф, вершинами которого являются \textit{\mbox{sc-элементы}}, а дугами -- пары инцидентности, связывающие \textit{sc-коннекторы} с инцидентными им \textit{\mbox{sc-элементами}}, которые являются компонентами указанных \textit{sc-коннекторов}.Если полученный таким способом орграф не является связным орграфом, то исходную структуру будем считать \textbf{\textit{несвязной структурой}}.}\scnheader{тривиальная структура}
\scnidtf{структура первого уровня}
\scntext{explanation}{\textbf{\textit{тривиальная структура}} -- \textit{структура}, не содержащая в качестве элементов связок.}\scnheader{нетривиальная структура}
\scnsuperset{структура второго уровня}
\scntext{explanation}{\textbf{\textit{нетривиальная структура}} -- \textit{структура}, среди элементов которой есть хотя бы одна связка.}\scnheader{элемент структуры\scnrolesign}
\scniselement{неосновное понятие}
\begin{scnsubdividing}

\scnitem{непредставленное множество\scnrolesign}
\scnitem{полностью представленное множество\scnrolesign}
\scnitem{частично представленное множество\scnrolesign}
\scnitem{элемент структуры, не являющийся множеством\scnrolesign}

\end{scnsubdividing}
\begin{scnsubdividing}

\scnitem{максимальное множество\scnrolesign}
\scnitem{немаксимальное множество\scnrolesign}

\end{scnsubdividing}
\scntext{explanation}{\textbf{\textit{элемент структуры\scnrolesign}} -- \textit{неосновное понятие}, \textit{ролевое отношение}, указывающее на все элементы каждой структуры.В рамках заданной структуры ее элементы можно классифицировать по заданным признакам:\begin{scnitemize}
\item насколько полно в рамках \uline{заданной \textit{структуры}} представлено множество, обозначаемое \textit{заданным sc-элементом} вместе с соответствующими дугами принадлежности;\item существуют ли в рамках \uline{заданной \textit{структуры}} \textit{sc-элементы}, обозначающие множества, являющиеся надмножествами того множества, которое обозначается \uline{заданным \textit{sc-элементом}};\item уровень (этаж) иерархии перехода от знаков к метазнакам для \uline{заданного \textit{sc-элемента}} в рамках заданной \textit{структуры}.\end{scnitemize}
}\scnheader{непредставленное множество\scnrolesign}
\scnidtf{множество, не представленное в рамках данной структуры\scnrolesign}
\scnidtf{быть знаком множества, элементы которого не являются элементами данной структуры\scnrolesign}
\scniselement{ролевое отношение}
\scntext{explanation}{\textbf{\textit{непредставленное множество\scnrolesign}} -- \textit{ролевое отношение}, связывающее структуру со знаком множества, все элементы которого не являются элементами данной структуры.}\scnheader{полностью представленное множество\scnrolesign}
\scnidtf{множество, полностью представленное в рамках данной структуры\scnrolesign}
\scnidtf{множество, все элементы которого являются элементами данной структуры\scnrolesign}
\scnidtf{полностью представленный класс\scnrolesign}
\scniselement{ролевое отношение}
\scntext{explanation}{\textbf{\textit{полностью представленное множество\scnrolesign}} -- \textit{ролевое отношение}, связывающее \textit{структуру} со знаком множества (любого семантического типа -- класса, связки или структуры), все элементы которого являются элементами данной \textit{структуры}.}\scnheader{частично представленное множество\scnrolesign}
\scnidtf{множество, частично представленное в рамках данной структуры\scnrolesign}
\scnidtf{множество, некоторые элементы которого являются элементами данной структуры\scnrolesign}
\scnidtf{быть знаком множества, некоторые элементы которого являются элементами данной структуры\scnrolesign}
\scniselement{ролевое отношение}
\scntext{explanation}{\textbf{\textit{частично представленное множество\scnrolesign}} -- ролевое отношение, связывающее структуру со знаком множества, не все элементы которого являются элементами данной структуры.}\scnheader{элемент структуры, не являющийся множеством\scnrolesign}
\scniselement{ролевое отношение}
\scnheader{максимальное множество\scnrolesign}
\scntext{explanation}{\textbf{\textit{максимальное множество\scnrolesign}} -- \textit{ролевое отношение}, связывающее \textit{структуру} со знаком множества, для которого не существует множества, которое было бы надмножеством указанного множества и знак которого был бы элементом этой же структуры.}\scnheader{немаксимальное множество\scnrolesign}
\scntext{explanation}{\textbf{\textit{немаксимальное множество\scnrolesign}} -- \textit{ролевое отношение}, связывающее \textit{структуру} со знаком множества, для которого в рамках данной \textit{структуры} существует множество, являющееся надмножеством указанного множества.}\scnheader{первичный элемент\scnrolesign}
\scnidtf{первичный элемент данной структуры\scnrolesign}
\scnidtf{sc-элемент первого уровня в рамках данной структуры\scnrolesign}
\scniselement{ролевое отношение}
\scniselement{семантический уровень структурного элемента}
\scnsubset{элемент структуры\scnrolesign}
\scntext{explanation}{\textbf{\textit{первичный элемент\scnrolesign}} -- ролевое отношение, указывающее на элемент \textit{структуры}, являющийся либо терминальным элементом, либо знаком множества, такого что не существует другого элемента этой же структуры, который был бы элементом множества, обозначаемого первым из указанных элементов структуры. При этом соответствующая пара принадлежности может существовать, но в состав данной структуры не входить.}\scnheader{вторичный элемент\scnrolesign}
\scnidtf{вторичный элемент данной структуры\scnrolesign}
\scnidtf{элемент данной структуры имеющий семантический уровень более 2\scnrolesign}
\scnidtf{непервичный элемент\scnrolesign}
\scniselement{ролевое отношение}
\scnsubset{элемент структуры\scnrolesign}
\scntext{explanation}{\textbf{\textit{вторичный элемент\scnrolesign}} -- ролевое отношение, указывающее на элемент структуры, обозначающий множество, все или некоторые элементы которого являются элементами указанной структуры.}\scnsuperset{элемент второго уровня\scnrolesign}
\scnheader{элемент второго уровня\scnrolesign}
\scniselement{ролевое отношение}
\scniselement{семантический уровень структурного элемента}
\scntext{explanation}{\textbf{\textit{элементом второго уровня\scnrolesign}} в рамках заданной \textit{структуры} может быть связка первичных элементов, тривиальная структура из первичных элементов или класс первичных элементов.}\scnheader{структура второго уровня\scnrolesign}
\scntext{explanation}{\textbf{\textit{структура второго уровня}} - \textit{структура}, среди элементов которой есть хотя бы один \textit{элемент второго уровня\scnrolesign}.}\scnheader{семантический уровень структурного элемента}
\scniselement{параметр}
\scntext{explanation}{\textbf{\textit{семантический уровень структурного элемента}} представляет собой \textit{параметр}, каждый элемент которого является классом \textit{sc-дуг принадлежности}, связывающих некоторую \textit{структуру} с теми ее элементами, который имеют одинаковый семантический уровень в рамках данной структуры. Значением данного параметра является число, обозначающее указанный семантический уровень.\textbf{\textit{семантический уровень структурного элемента}} вычисляется следующим образом:\begin{scnitemize}
\item элементы структуры, входящие в нее с атрибутом \textit{первичный элемент\scnrolesign} имеют семантический уровень 1;\item уровень элемента, не являющегося \textit{первичным элементом\scnrolesign} структуры вычисляется путем прибавления 1 к максимальному из уровней элементов этого элемента (множества), входящих в эту же структуру. Например, \textit{sc-дуга}, соединяющая два \textit{первичных элемента\scnrolesign структуры} будет иметь семантический уровень 2, а \textit{sc-элемент}, обозначающий отношение, которому принадлежит указанная \textit{sc-дуга} -- семантический уровень 3.\end{scnitemize}
}\scnrelfrom{описание примера}{\scnfileimage[20em]{figures/sd_structures/sem_level_struct_elem.png}
}
\scnheader{количество семантических уровней элементов структуры}
\scniselement{параметр}
\scntext{explanation}{\textbf{\textit{количество семантических уровней элементов структуры}} -- параметр, каждый элемент которого представляет собой класс структур, у которых совпадает максимальный среди семантических уровней элементов этих структур.Значением данного параметра является число, совпадающее с указанным максимальным семантическим уровнем элементов.}\scnheader{метасвязь\scnrolesign}
\scniselement{ролевое отношение}
\scnsubset{вторичный элемент\scnrolesign}
\scntext{explanation}{\begin{scnenumerate}
\item Каждая входящая в структуру связь, хотя бы одним компонентом которой является связь, входящая в эту же структуру, элементами которой являются \textit{первичные элементы\scnrolesign} этой структуры, является \textbf{\textit{метасвязью\scnrolesign}} указанной структуры;\item Каждая входящая в структуру связь, хотя бы одним компонентом которой является \textbf{\textit{метасвязь\scnrolesign}} этой структуры также является \textbf{\textit{метасвязью\scnrolesign}} указанной структуры;\end{scnenumerate}
}\scnheader{полиморфность*}
\scnsubset{соответствие*}
\scniselement{бинарное отношение}
\scntext{explanation}{\textbf{\textit{полиморфность*}} - это \textit{соответствие}, заданное на \textit{структурах}, при котором каждому элементу из области определения соответствия (первой \textit{структуры}) ставится в соответствие один или более элемент из области значения соответствия (второй \textit{структуры}), при этом существует хотя бы один элемент области определения соответствия, которому соответствуют два или более элемента из области значения соответствия.}\scnheader{полиморфизм*}
\scniselement{бинарное отношение}
\scnheader{гомоморфность*}
\scnidtf{гомоморфность структур*}
\scnsubset{соответствие*}
\scniselement{бинарное отношение}
\scntext{explanation}{\textbf{\textit{гомоморфность*}} - это \textit{соответствие}, заданное на \textit{структурах}, при котором каждому элементу из области определения соответствия (первой \textit{структуры}) ставится в соответствие только один элемент из области значения соответствия (второй \textit{структуры}).}\scnrelfrom{описание примера}{\scnfileimage[20em]{figures/sd_structures/homomorphism.png}
}
\scnheader{гомоморфизм*}
\scniselement{бинарное отношение}
\scnheader{изоморфность*}
\scnidtf{изоморфное соответствие*}
\scnidtf{изоморфность структур*}
\scnsubset{гомоморфность*}
\scniselement{бинарное отношение}
\scntext{explanation}{\textbf{\textit{изоморфность*}} - это \textit{гомоморфность*}, при которой для каждого элемента из области значения существует ровно один соответствующий элемент из области определения.}\scnrelfrom{описание примера}{\scnfileimage[20em]{figures/sd_structures/isomorphism.png}
}
\scnheader{изоморфизм*}
\scniselement{бинарное отношение}
\scnheader{автомоморфность*}
\scnsubset{гомоморфность*}
\scniselement{бинарное отношение}
\scntext{explanation}{\textbf{\textit{автоморфность*}} - это \textit{изоморфность*}, у которой область определения соответствия и область значения соответствия совпадают.}\scnrelfrom{описание примера}{\scnfileimage[20em]{figures/sd_structures/automorphism.png}
}
\scnheader{автоморфизм*}
\scniselement{бинарное отношение}
\scnheader{аналогичность структур*}
\scnsubset{соответствие*}
\scniselement{бинарное отношение}
\scntext{explanation}{\textbf{\textit{аналогичность структур*}} - \textit{соответствие*}, задаваемое на структурах, и фиксирующее факт наличия некоторой аналогии на подструктурах (подмножествах) указанных структур. Каждой ориентированной паре, принадлежащей \textbf{\textit{аналогичности структур*}} может быть поставлено в соответствие множество пар, задающих \textit{сходства*} некоторых подструктур и \textit{различия*} некоторых подструктур исходных структур.}\scnrelfrom{описание примера}{\scnfileimage[20em]{figures/sd_structures/analogy.png}
}
\scnheader{сходство*}
\scniselement{бинарное отношение}
\scnheader{различие*}
\scniselement{бинарное отношение}
\scnheader{первичная синтаксическая структура sc-текста*}
\scniselement{бинарное отношение}
\scntext{explanation}{\textbf{\textit{первичная синтаксическая структура sc-текста*}} - это бинарное отношение, связывающее некоторый \textit{sc-текст} с другим \textit{sc-текстом}, формируемым по следующим правилам:\begin{scnitemize}
\item каждому \textit{sc-узлу} первого \textit{sc-текста} соответствует \textit{синглетон} (\textit{знак sc-узла}) в рамках второго \textit{sc-текста};\item каждому \textit{sc-коннектору} из первого \textit{sc-текста} в рамках второго \textit{sc-текста} соответствует \textit{синглетон}, обозначающий данный \textit{sc-коннектор} и соединенный с другими \textit{синглетонами} второго \textit{sc-текста} парами инцидентности двух типов, в зависимости от того, началом или концом данного \textit{sc-коннектора} являются обозначаемые этими \textit{синглетонами sc-элементы}. В случае, когда \textit{sc-коннектор} является \textit{sc-ребром}, то достаточно пар инцидентности первого типа.\item для каждого \textit{синглетона} в рамках второго \textit{sc-текста} явно указывается синтаксический тип, определяемый типом соответствующего ему элемента из первого \textit{sc-текста} (\textit{знак sc-константы}, \textit{знак sc-узла} и т.п.).\end{scnitemize}
Стоит отметить, что подобным образом может быть задана синтаксическая структура любого текста, а не только sc-текста. В этом случае понадобятся другие отношения инцидентности другие классы синтаксических типов.}\scnrelfrom{описание примера}{\scnfileimage[20em]{figures/sd_structures/primary_sc_syntax.png}
}
\bigskip
\end{scnsubstruct}
\end{SCn}


\scsubsection[
    \protect\scnmonographychapter{Глава 2.4. Формальные онтологии базовых классов сущностей - множеств, связей, отношений, параметров, величин, чисел, структур, темпоральных сущностей}
    ]{Предметная область и онтология темпоральных сущностей}
\label{sd_temp_entities}
\begin{SCn}
\scnsectionheader{\currentname}
\begin{scnsubstruct}
\scnheader{Предметная область темпоральных сущностей}
\scnidtf{Предметная область темпоральных связей и отношений}
\scnidtf{Предметная область временных сущностей}
\scniselement{предметная область}
\begin{scnhaselementrole}{класс объектов исследования}временная сущность
\end{scnhaselementrole}
\begin{scnhaselementrolelist}{класс объектов исследования}

прошлая сущность;настоящая сущность;будущая сущность;временная связь;ситуация;процесс;процесс в sc-памяти;процесс во внешней среде ostis-системы;материальная сущность;воздействие;отношение;класс временных связей;класс временных и постоянных связей;множество;ситуативное множество;неситуативное множество;частично ситуативное множество;темпоральная связь;темпоральное отношение;начало\scnsupergroupsign;завершение\scnsupergroupsign;длительность\scnsupergroupsign;тысячелетие;век;год;месяц;сутки;час;минута;секунда

\end{scnhaselementrolelist}
\begin{scnhaselementrolelist}{исследуемое отношение}

воздействующая сущность*;объект воздействия*;начальная ситуация*;причинная ситуация*;конечная ситуация*;событие*;последний добавленный sc-элемент\scnrolesign;темпоральное включение*;темпоральная часть*;начальный этап*;конечный этап*;промежуточный этап*;темпоральное включение без совпадения начальных и конечных моментов*;темпоральное включение с совпадением начальных моментов*;темпоральное включение с совпадением конечных моментов*;темпоральное совпадение*;темпоральное объединение*;темпоральная декомпозиция*;темпоральная смежность*;темпоральная последовательность с промежутком*;темпоральная последовательность с пересечением*;номер тысячелетия\scnrolesign;номер века\scnrolesign;номер года\scnrolesign;номер месяца в году\scnrolesign;номер суток в месяце\scnrolesign;номер часа в дне\scnrolesign;номер минуты в часе\scnrolesign;номер секунды в минуте\scnrolesign

\end{scnhaselementrolelist}
\scnheader{временная сущность}
\scnidtf{временно существующая сущность}
\scnidtf{нестационарная сущность}
\scnidtf{сущность, имеющая и/или начало, и/или конец своего существования}
\scnidtf{sc-элемент, являющийся знаком некоторой временно существующей сущности}
\scnidtf{сущность, обладающая темпоральными характеристиками (длительностью, начальным моментом, конечным моментом и т.д.)}
\begin{scnsubdividing}

\scnitem{прошлая сущность}
\scnitem{настоящая сущность}
\scnitem{будущая сущность}

\end{scnsubdividing}
\begin{scnsubdividing}

\scnitem{временная связь}
\scnitem{темпоральная структура\\\scnidtf{структура, содержащая хотя бы одну временную сущность}
\scnrelfrom{включение}{структура}
\scntext{note}{Следует отличать:
\begin{scnitemize}
\item временный характер самой структуры как sc-элемента
\scnitem{\item временный характер sc-элементов, принадлежащих данной структуре, и сущностей, обозначаемых этими sc-элементами}
\scnitem{\item временный характер пар принадлежности, связывающих структуру с ее элементами.}
\end{scnitemize}
}\scnidtf{структура, описывающая темпоральные свойства (свойства, связанные со временем) окружающей среды, частью которой являются также и различные базы знаний кибернетических систем (в том числе и собственная база знаний).}
\begin{scnsubdividing}

\scnitem{ситуация\\\scnidtf{статическая темпоральная структура}
}
\scnitem{процесс\\\scnidtf{динамическая структура}
\scnidtf{динамическая темпоральная структура}
}

\end{scnsubdividing}
}
\scnitem{материальная сущность}

\end{scnsubdividing}
\begin{scnsubdividing}

\scnitem{непрерывная временная сущность\\\begin{scnsubdividing}

\scnitem{точечная временная сущность\\\scnidtf{атомарная временная сущность}
\scnidtf{условно мгновенная временная сущность}
\scnidtf{временная сущность, длительность существования которой в данном контексте считается несущественной (пренебрежительно малой)}
}
\scnitem{длительная непрерывная временная сущность}

\end{scnsubdividing}
}
\scnitem{дискретная временная сущность\\\scnidtf{временная сущность, которая может быть декомпозирована на последовательность точечных временных сущностей}
\scnidtf{временная сущность, которой соответствует некоторый временной ряд параметров (состояний) точечных временных сущностей, на которые декомпозируется исходная временная сущность}
}
\scnitem{прерывистая временная сущность\\\scnidtf{временная сущность, являющаяся результатом соединения нескольких не только точечных временных сущностей}
\scnidtf{временная сущность с прерываниями}
}

\end{scnsubdividing}
\scntext{note}{Следует отметить, что приведенная классификация \textit{временных сущностей} характеризует не столько сами \textit{временные сущности}, сколько наши знания о них и степень детализации знаний об этих сущностях, с которой они описаны в базе знаний. Так, если для решения конкретных задач не важно, как изменялась некоторая \textit{временная сущность} в рамках какого-либо периода времени, а важно только ее начальное и конечное состояние, то она может рассматриваться как \textit{точечная временная сущность}. Впоследствии же та же \textit{временная сущность} может быть рассмотрена и описана с большей степенью детализации, и таким образом, уже не будет точечной.}\scntext{explanation}{Следует отличать:\begin{scnitemize}
\item временный характер сущности, обозначаемой \textit{sc-элементом};\item временный характер существования самого \textit{sc-элемента} в рамках \textit{sc-памяти}, поскольку в ходе обработки информации каждый \textit{sc-элемент} может быть удален из \textit{sc-памяти}; \item временный характер описываемых ситуаций, событий и самих процессов;\item временный характер хранения в sc-памяти тех sc-конструкций, которые являются самими описаниями соответствующих ситуаций, событий и процессов.\end{scnitemize}
}\scnheader{следует отличать*}
\begin{scnhaselementset}

\scnitem{временная сущность}
\scnitem{процесс}

\end{scnhaselementset}
\scntext{note}{Следует отличать, например, \textit{материальную сущность} (некоторый физический или, в частности, биологический объект) от различных динамических структур (\textit{процессов}), которые с той или иной степенью детализации и в том или ином ракурсе отражают (описывают) динамику изменений этой \textit{материальной сущности}. При этом сам \textit{процесс} как уточнение динамики некоторой последовательности ситуаций и событий, также является сущностью, принадлежащей к классу \textit{временных сущностей}.}\scnheader{прошлая сущность}
\scnidtf{сущность, существовавшая в прошлом времени}
\scnidtf{сущность прошлого времени}
\scnidtf{сущность, завершившая свое существование}
\scnheader{настоящая сущность}
\scnidtf{сущность, существующая в текущий момент времени}
\scnidtf{сущность, существующая сейчас}
\scnidtf{сущность настоящего времени}
\scnheader{будущая сущность}
\scnidtf{возможно будущая сущность}
\scnidtf{прогнозируемая временная сущность}
\scnidtf{временная сущность, которая может существовать в будущем}
\scnidtf{сущность, которая может или должна начать свое существование в будущем времени}
\scnrelfrom{включение}{инициированное действие}
\scntext{explanation}{Каждой \textbf{\textit{будущей сущности}} можно поставить в соответствие вероятность ее возникновения.}\scnheader{временная связь}
\scnidtf{нестационарная связь}
\scnidtf{временно существующая связь}
\scntext{explanation}{Каждая \textbf{\textit{временная связь}} представляет собой \textit{связку}, принадлежащую множеству \textit{временных сущностей}.Понятие \textbf{\textit{временной связи}} не следует путать с понятием \textit{темпоральной связи}, которая сама является \textit{постоянной сущностью}, описывающей то, как связаны во времени некоторые \textit{временные сущности}.}\scnheader{ситуация}
\scnidtf{состояние}
\scnidtf{временная структура}
\scnidtf{временно существующая структура}
\scnidtf{квазистационарная структура}
\scnidtf{состояние некоторой динамической системы, описываемое с некоторой степенью детализации (подробности)}
\scnidtf{квазистационарная структура, существующая временно (в течение некоторого отрезка времени)}
\scntext{explanation}{Под ситуацией понимается \textit{структура}, содержащая, по крайней мере, один элемент, который является \textit{временной сущностью}. Наличие в рамках ситуации нескольких \textit{временных сущностей} говорит о том, что существует момент времени (в прошлом, настоящем или будущем), в который все они существуют одновременно.}\scnheader{процесс}
\scnidtf{процесс преобразования некоторой временной сущности из одного состояния в другое}
\scnidtf{процесс перехода от одной ситуации к другой}
\scnidtf{абстрактный процесс}
\scnidtf{информационная модель некоторых преобразований}
\scnidtf{динамическая sc-модель}
\scnidtf{динамическая структура}
\scnrelfrom{включение}{воздействие}
\scntext{explanation}{Каждый \textbf{\textit{процесс}} определяется (задается) либо возникновением или исчезновением связей, связывающих заданную \textit{временную сущность} с другими сущностями, либо возникновением или исчезновением связей, связывающих части указанной \textit{временной сущности} с другими сущностями. Многим \textbf{\textit{процессам}} можно поставить в соответствие \textit{ситуацию}, которая является его \textit{начальной ситуацией*} и \textit{ситуацию}, которая является его \textit{конечной ситуацией*}, то есть указать \textit{ситуации}, переход между которыми осуществляется в ходе \textbf{\textit{процесса}}.При этом знаки тех \textit{временных сущностей}, с которыми связаны изменения, описываемые некоторым \textbf{\textit{процессом}}, входят в данный \textbf{\textit{процесс}} как элементы и при необходимости уточняются дополнительными \textit{ролевыми отношениями}.}\begin{scnsubdividing}

\scnitem{процесс в sc-памяти}
\scnitem{процесс во внешней среде ostis-системы}

\end{scnsubdividing}
\scntext{note}{Каждой \textbf{\textit{материальной сущности}} можно поставить в соответствие различные \textit{процессы}, описывающие ее преобразование из одного состояния в другое.}\scntext{note}{Поскольку \textit{процесс} представляет собой изменяющуюся во времени динамическую структуру, то полностью представить процесс в базе знаний в общем случае не представляется возможным. Однако, можно ввести sc-элемент, обозначающий конкретный процесс, с необходимой степенью детализации описать его декомпозицию на более частные подпроцессы и/или описать ситуации, соответствующие состояниям процесса в разные моменты времени. В данном случае можно провести некоторую аналогию с \textit{бесконечными множествами}, все элементы которых физически не могут быть представлены в базе знаний одновременно, тем не менее, само множество и некоторые из его элементов могут быть описаны с необходимой степенью детализации.}\scnheader{воздействие}
\scnidtf{процесс, осуществляющийся на основе (в результате) воздействия одной сущности на другую}
\scnrelfrom{включение}{действие}
\scntext{explanation}{Каждому \textbf{\textit{воздействию}} может быть поставлена в соответствие (1) некоторая \textit{воздействующая сущность*}, т.е. сущность, осуществляющая \textbf{\textit{воздействие}} (в частности, это может быть некоторое физическое поле), и (2) некоторый \textit{объект воздействия*}, т.е. сущность, на которую воздействие направлено. Если \textbf{\textit{воздействие}} связано с \textit{материальной сущностью}, то его объектом воздействия является либо сама эта \textit{материальная сущность}, либо некоторая ее пространственная часть.}\scnheader{исходная ситуация*}
\scnidtf{начальная ситуация процесса*}
\scnidtf{начальная ситуация*}
\scniselement{бинарное отношение}
\scnrelfrom{первый домен}{процесс}
\scnrelfrom{второй домен}{ситуация}
\scntext{explanation}{Связки отношения \textbf{\textit{исходная ситуация*}} связывают некоторый \textit{процесс} и некоторую ситуацию, являющуюся начальной для этого \textit{процесса}, и, как правило, изменяемой в течение выполнения этого \textit{процесса}.Первым компонентом каждой связки отношения \textbf{\textit{исходная ситуация*}} является знак \textit{процесса}, вторым -- знак начальной \textit{ситуации}.}\scnheader{причинная ситуация*}
\scnsubset{начальная ситуация*}
\scntext{explanation}{Под причинной ситуацией понимается такая \textit{начальная ситуация*}, которая обладает достаточной полнотой для однозначного задания инициируемого \textit{процесса}.}\scnheader{конечная ситуация*}
\scnidtf{конечная ситуация процесса*}
\scnidtf{результирующая ситуация*}
\scniselement{бинарное отношение}
\scnrelfrom{первый домен}{процесс}
\scnrelfrom{второй домен}{ситуация}
\scntext{explanation}{Связки отношения \textbf{\textit{конечная ситуация*}} связывают некоторый \textit{процесс} и некоторую \textit{ситуацию}, ставшую результатом выполнения этого \textit{процесса}, то есть его следствием.Первым компонентом каждой связки отношения \textbf{\textit{конечная ситуация*}} является знак \textit{процесса}, вторым -- знак конечной \textit{ситуации}.}\scnheader{точечный процесс}
\scnidtf{атомарный процесс}
\scnidtf{условно мгновенный процесс}
\scnidtf{процесс, длительность которого в данном контексте считается несущественной (пренебрежимо малой)}
\scnsubset{точечная временная сущность}
\scnheader{элементарный процесс}
\scnidtf{процесс, детализация которого на входящие в него подпроцессы в текущем контексте не производится}
\scnsuperset{точечный процесс}
\scntext{note}{Элементарные процессы могут иметь длительность и, следовательно, не обязательно являются атомарными процессами.}\scntext{note}{Понятия \textit{точечного процесса} и \textit{элементарного процесса}, как и понятие \textit{точечной временной сущности} в целом, характеризуют не столько характеристики самого \textit{процесса}, сколько степень наших знаний о нем и степень детализации описания процесса в базе знаний. Так, очевидно, что любой процесс, протекающий в компьютерной системе, может быть при необходимости детализирован до уровня команд процессора, затем до уровня микропрограмм и даже до уровня физических процессов (изменения физических характеристик сигналов), однако чаще всего такая детализация не требуется.}\scnrelto{примечание}{точечный процесс}
\scnheader{событие}
\scnsubset{точечная временная сущность}
\scnidtf{точечная временная сущность, являющаяся началом и/или завершением какой-либо временной сущности (например, процесса)}
\scnidtf{граничная точка временной сущности}
\scnrelfrom{описание примера}{\scnfileimage[20em]{figures/sd_temp_entities/event.png}
}
\scntext{note}{Одно и то же событие может быть одновременно завершением одной временной сущности и началом другой. В приведенном примере событие $\bm{ei}$ является завершением временной сущности $\bm{si}$ и началом временной сущности $\bm{sj}$.}\scnheader{начало*}
\scnidtf{быть начальным событием заданной временной сущности*}
\scnrelfrom{первый домен}{временная сущность}
\scnrelfrom{второй домен}{событие}
\scnidtf{быть начальной точечной временной частью заданной временной сущности*}
\scnheader{завершение*}
\scnidtf{конец*}
\scnidtf{быть конечным событием заданной временной сущности*}
\scnidtf{быть конечной точечной временной частью заданной временной сущности*}
\scnrelfrom{первый домен}{временная сущность}
\scnrelfrom{второй домен}{событие}
\scnheader{событие*}
\scniselement{бинарное отношение}
\scntext{explanation}{Связки отношения \textbf{\textit{событие*}} связывают знак процесса и ориентированную пару, первым компонентом которой является знак \textit{начальной ситуации*} данного процесса, вторым компонентом -- знак \textit{конечной ситуации*} данного процесса.}\scnrelfrom{описание примера}{\scnfileimage[20em]{figures/sd_temp_entities/nrel_event.png}
}
\scnheader{детализация процесса*}
\scnidtf{Бинарное ориентированное отношение, каждая связка которого связывает некоторый процесс с более детальным его описанием, что предполагает представление декомпозиции этого процесса на систему взаимосвязанных его подпроцессов (в том числе элементарных).}
\scnrelfrom{пример}{Переход от процесса, соответствующего какой-либо программе, к рассмотрению декомпозиции этого процесса (протокола) в терминах языка программирования высокого уровня, затем переход для каждого из полученных подпроцессов (операторов языка высокого уровня) к детализации выполнения этих подпроцессов на уровне машинных операций, выполняемых процессором компьютера (на уровне ассемблера), и далее к детализации выполнения подпроцессов уровня машинных операций к подпроцессам на уровне языка микропрограммирования. Таким образом, детализация процесса может быть иерархической, вплоть до уровня \textit{элементарных процессов}.}
\scnheader{отношение}
\begin{scnsubdividing}

\scnitem{класс временных связей}
\scnitem{класс постоянных связей}
\scnitem{класс временных и постоянных связей}

\end{scnsubdividing}
\scnheader{класс временных связей}
\scnidtf{отношение, все связки которого являются нестационарными}
\scntext{explanation}{В общем случае \textbf{\textit{класс временных связей}} не является \textit{ситуативным множеством}, поскольку факт принадлежности некоторой \textit{временной связи} такому классу следует считать постоянным, а не временным, поскольку временность/постоянство связи и ее семантический тип, задаваемый классом (отношением), это принципиально разные параметры (характеристики, признаки) любой связи.}\scnheader{класс постоянных связей}
\scnidtf{отношение, все связки которого являются стационарными}
\scnheader{класс временных и постоянных связей}
\scnidtf{отношение, некоторые (но не все) связки которого являются нестационарными}
\scnheader{множество}
\begin{scnsubdividing}

\scnitem{ситуативное множество}
\scnitem{неситуативное множество}
\scnitem{частично ситуативное множество}

\end{scnsubdividing}
\scnheader{ситуативное множество}
\scnidtf{полностью ситуативное множество}
\scntext{explanation}{Под \textbf{\textit{ситуативным множеством}} понимается постоянное множество, у которого все выходящие из него связи принадлежности являются \textit{временными сущностями}.В частности, ситуативное множество может использоваться как вспомогательная динамическая структура, которая содержит элементы некоторых структур, обрабатываемые в данный момент, например, это может быть копия некоторого множества, из которой постепенно удаляются элементы по мере их просмотра и обработки. В случае, когда такая структура содержит всего один элемент, ее можно считать \underline{указателем} на данный элемент, при этом в разные моменты времени это могут быть разные элементы.}\scnheader{последний добавленный sc-элемент\scnrolesign}
\scniselement{ролевое отношение}
\scnheader{неситуативное множество}
\scntext{explanation}{Под \textbf{\textit{неситуативным множеством}} понимается постоянное множество, у которого все выходящие из него связи принадлежности являются \textit{постоянными сущностями}.}\scnheader{частично ситуативное множество}
\scntext{explanation}{Под \textbf{\textit{частично ситуативным множеством}} понимается постоянное множество, у которого некоторые (но не все) выходящие из него связи принадлежности являются \textit{временными сущностями}.}\scnheader{темпоральная связь}
\scnidtf{связь во времени}
\scnidtf{\uline{постоянная} связь, описывающая связь во времени между временными сущностями}
\scnheader{темпоральное отношение}
\scnrelto{семейство подмножеств}{темпоральная связь}
\scnidtf{класс темпоральных связей}
\scnidtf{отношение, задающее темпоральные связи между временными сущностями}
\scnhaselement{темпоральное включение*}
\scnhaselement{темпоральное объединение*}
\scnhaselement{темпоральная декомпозиция*}
\scnhaselement{темпоральная последовательность*}
\begin{scnsubdividing}

\scnitem{темпоральная смежность*}
\scnitem{темпоральная последовательность с промежутком*}
\scnitem{темпоральная последовательность с пересечением*}

\end{scnsubdividing}
\scnheader{темпоральное включение*}
\scntext{explanation}{Связки отношения \textbf{\textit{темпоральное включение*}} связывают две \textit{временные сущности}, период существования одной из которых полностью включается в период существования второй.\\Первым компонентом каждой связки отношения \textbf{\textit{темпоральное включение*}} является знак \textit{временной сущности}, \textit{длительность} существования которой больше.}\scnsuperset{темпоральная часть*}
\scnsuperset{темпоральное включение без совпадения начальных и конечных моментов*}
\scnsuperset{темпоральное совпадение*}
\scnsuperset{темпоральное включение с совпадением начальных моментов*}
\scnsuperset{темпоральное включение с совпадением конечных моментов*}
\scnheader{темпоральная часть*}
\scnidtf{этап (период) заданной временной сущности*}
\scnidtf{этап процесса существования временной сущности*}
\scnsuperset{начальный этап*}
\scnsuperset{конечный этап*}
\scnsuperset{промежуточный этап*}
\scnsuperset{подпроцесс*}
\scnrelfrom{первый домен}{процесс}
\scnrelfrom{второй домен}{процесс}
\scnrelfrom{описание примера}{\scnfileimage[20em]{figures/sd_temp_entities/temporal_part.png}
}
\scnrelfrom{иллюстрация}{\scnfileimage{[20em]{figures/sd_temp_entities/img_temporal_part.png}}
}
\scntext{примечание}{Связки отношения \textbf{\textit{темпоральная часть*}} связывают две \textit{временные сущности}, одна из которых является частью другой, например, действие и одно из его поддействий. Соответственно, период существования одной из этих сущностей всегда будет включаться в период существования другой (большей).В отличие от более общего отношения \textit{темпоральное включение*}, связки которого могут связывать любые \textit{временные сущности}, связки отношения \textbf{\textit{темпоральная часть*}} связывают только \textit{временные сущности}, одна из которых является частью другой.}
\scnheader{следует отличать*}
\begin{scnhaselementset}

\scnitem{темпоральная часть*\\\scnsuperset{подпроцесс*}
}
\scnitem{темпоральное включение*\\\scntext{note}{Связь \textit{темпорального включения*} может связывать абсолютно разные \textit{временные сущности}, существующие в общем случае в разных местах, а не только \textit{временные сущности}, одна из которых является частью другой. Хотя формально и можно объединить любые разные \textit{временные сущности} в одну общую \textit{временную сущность}, далеко не всегда имеет смысл это делать.}}

\end{scnhaselementset}
\scnheader{темпоральное включение без совпадения начальных и конечных моментов*}
\scnidtf{строгое темпоральное включение*}
\scnrelfrom{описание примера}{\scnfileimage[20em]{figures/sd_temp_entities/strict_temporal_inclusion.png}
}
\scnrelfrom{иллюстрация}{\scnfileimage{[20em]{figures/sd_temp_entities/img_strict_temporal_inclusion.png}}
}
\scnheader{темпоральное включение с совпадением начальных моментов*}
\scnrelfrom{описание примера}{\scnfileimage[20em]{figures/sd_temp_entities/temporal_include_with_match_start_points.png}
}
\scnrelfrom{иллюстрация}{\scnfileimage{[20em]{figures/sd_temp_entities/img_temporal_include_with_match_start_points.png}}
}
\scnheader{темпоральное включение с совпадением конечных моментов*}
\scnrelfrom{описание примера}{\scnfileimage[20em]{figures/sd_temp_entities/temporal_include_with_terminal_point_match.png}
}
\scnrelfrom{иллюстрация}{\scnfileimage{[20em]{figures/sd_temp_entities/img_temporal_include_with_terminal_point_match.png}}
}
\scnheader{темпоральное совпадение*}
\scnidtf{совпадение начала и завершения*}
\scniselement{отношение эквивалентности}
\scnheader{темпоральное объединение*}
\scnidtf{преобразование нескольких временных сущностей в одну общую временную сущность, которая может оказаться прерывистой или даже дискретной*}
\scnrelboth{аналог}{объединение множеств*}
\scntext{note}{С формальной точки зрения объединять можно любые временные сущности. Но делать это надо только тогда, когда это имеет смысл, точно так же, как и в случае объединения множеств.}\scnrelfrom{описание примера}{\scnfileimage[20em]{figures/sd_temp_entities/temporal_union.png}
}
\scnrelfrom{иллюстрация}{\scnfileimage{[20em]{figures/sd_temp_entities/img_temporal_union.png}}
}
\scnheader{темпоральная декомпозиция*}
\scnidtf{Темпоральное отношение, связывающее временную сущность и множество смежных во времени временных сущностей, которые являются темпоральными частями исходной сущности и результатом темпорального объединения которых является исходная сущность*}
\scnrelboth{аналог}{разбиение*}
\scnrelfrom{описание примера}{\scnfileimage[20em]{figures/sd_temp_entities/temporal_decomposition.png}
}
\scnrelfrom{иллюстрация}{\scnfileimage{[20em]{figures/sd_temp_entities/img_temporal_decomposition.png}}
}
\scnheader{темпоральная смежность*}
\scnidtf{сразу позже*}
\scnidtf{смежность во времени*}
\scnidtf{строгая темпоральная последовательность (без темпорального промежутка)*}
\scnidtf{темпоральная последовательность без промежутка*}
\scnrelfrom{описание примера}{\scnfileimage[20em]{figures/sd_temp_entities/temporal_adjacency.png}
}
\scnrelfrom{иллюстрация}{\scnfileimage{[20em]{figures/sd_temp_entities/img_temporal_adjacency.png}}
}
\scnheader{темпоральная последовательность с промежутком*}
\scnidtf{позже*}
\scnrelfrom{описание примера}{\scnfileimage[20em]{figures/sd_temp_entities/temporal_sequence_with_intermediate.png}
}
\scnrelfrom{иллюстрация}{\scnfileimage{[20em]{figures/sd_temp_entities/img_temporal_sequence_with_intermediate.png}}
}
\scnheader{темпоральная последовательность с пересечением*}
\scnrelfrom{описание примера}{\scnfileimage[20em]{figures/sd_temp_entities/temporal_sequence_with_intersection.png}
}
\scnrelfrom{иллюстрация}{\scnfileimage{[20em]{figures/sd_temp_entities/img_temporal_cross_sequence.png}}
}
\scnheader{начало\scnsupergroupsign}
\scnidtf{одновременность начинаний\scnsupergroupsign}
\scnidtf{класс одновременно начавшихся сущностей\scnsupergroupsign}
\scniselement{параметр}
\scntext{explanation}{Каждый элемент множества \textbf{начало} представляет собой класс \textit{временных сущностей}, у которых совпадает момент начала их существования. Конкретное значение данного \textit{параметра} может быть как \textit{точной величиной}, так и \textit{неточной величиной} или \textit{интервальной величиной}.}\scnrelfrom{описание примера}{\scnfileimage[20em]{figures/sd_temp_entities/start.png}
}
\scntext{explanation}{В данном примере \textbf{\textbf{\textit{ki}}} обозначает класс сущностей, начавших свое существование 19 февраля 2015 года по григорианскому календарю. Конкретные примеры таких сущностей -- \textbf{\textit{bi}} и \textbf{\textit{bj}}. \textbf{\textit{ti}} обозначает временную точку григорианского календаря, соответствующую 19 февраля 2015 года.}\scnheader{завершение\scnsupergroupsign}
\scnidtf{конец\scnsupergroupsign}
\scnidtf{одновременность завершений\scnsupergroupsign}
\scnidtf{класс одновременно завершившихся сущностей\scnsupergroupsign}
\scniselement{параметр}
\scntext{explanation}{Каждый элемент множества \textbf{\textit{завершение}} представляет собой класс \textit{временных сущностей}, у которых совпадает конечный момент их существования (момент завершения существования). Конкретное значение данного \textit{параметра} может быть как \textit{точной величиной}, так и \textit{неточной величиной} или \textit{интервальной величиной}.}\scnrelfrom{описание примера}{\scnfileimage[20em]{figures/sd_temp_entities/completion.png}
}
\scntext{explanation}{В данном примере \textbf{\textit{ki}} обозначает класс сущностей, завершивших свое существование 21 февраля 2015 года по григорианскому календарю. Конкретные примеры таких сущностей -- \textbf{\textit{bi}} и \textbf{\textit{bj}}. \textbf{\textit{ti}} обозначает временную точку григорианского календаря, соответствующую 21 февраля 2015 года.}\scnheader{одновременность\scnsupergroupsign}
\scnidtf{параметр, значениями (элементами) которого являются классы либо одновременно существующих (происходящих) \textit{точечных временных сущностей}, одновременность которых рассматривается с заданной степенью точности, либо одновременно начинающихся и заканчивающихся длительных процессов}
\scntext{explanation}{Важно отметить, что элементами некоторого значения параметра \textit{одновременности} с заданной точностью могут быть только те временные сущности, которые и начались, и завершились в течение периода времени, заданного указанным значением этого параметра, но при этом начало и завершение этих временных сущностей не обязательно должно совпадать с началом и завершением указанного периода времени. Так, например, можно ввести значение параметра \textit{одновременности} ``\textit{2022 год по Григорианскому календарю}, элементами которого будут все временные сущности, начавшие и закончившие свое существовавшие в рамках 2022 года. При этом не обязательно, чтобы эти временные сущности начались именно в полночь 1 января 2022 года и закончились в полночь 1 января 2023 года, это могут быть временные сущности, существовавшие, например, в течение июля 2022 года.}\scnrelfrom{описание примера}{\textit{}\scnfileimage[20em]{figures/sd_temp_entities/simultaneity.png}
}
\scntext{note}{Некоторые значения параметра одновременности могут быть подмножествами других значений того же параметра. Семантика такой связи будет выражаться в том, что первое из указанных значений описывает \textit{одновременность} \textit{временных сущностей} с большей точностью. Так, в приведенном примере величина ``\textit{2002 год} описывает одновременность временных сущностей с точностью до года, а величина ``\textit{июль 2022 года} описывает одновременность временных сущностей с точностью до месяца. При этом очевидно, что сущности, входящие во величину ``\textit{июль 2022 года} будут также входить и в величину ``\textit{2022 год} (как например временная сущность $\bm{sk})$. В приведенном примере для простоты предполагается, что все измерения производятся по Григорианскому календарю.}\scnheader{соединение значений ориентированного параметра*}
\scnrelfrom{описание примера}{\scnfileimage[20em]{figures/sd_temp_entities/temporal_values_join.png}
}
\scntext{note}{В приведенном примере множество сущностей, существовавших 10.01.2022, и множество сущностей, существовавших 12.01.2022, при помощи отношения \textit{соединение значений ориентированного параметра*} образуют множество сущностей, существовавших в период 10-12.02.2022.}\scnheader{следует отличать*}
\begin{scnhaselementset}

\scnitem{темпоральное совпадение*\\\scniselement{отношение эквивалентности}
}
\scnitem{одновременность\scnsupergroupsign\\\scnidtf{фактор-множество для отношения темпоральное совпадение*}
}

\end{scnhaselementset}
\scnheader{длительность\scnsupergroupsign}
\scnidtf{класс временных сущностей, имеющих одинаковую длительность\scnsupergroupsign}
\scniselement{параметр}
\scnhaselement{тысячелетие}
\scnhaselement{век}
\scnhaselement{год}
\scnhaselement{месяц}
\scnhaselement{день}
\scnhaselement{час}
\scnhaselement{минута}
\scnhaselement{секунда}
\scntext{explanation}{Каждый элемент множества \textbf{\textit{длительность}} представляет собой класс \textit{временных сущностей}, у которых совпадает длительность их существования. Конкретное значение данного \textit{параметра} может быть как \textit{точной величиной}, так и \textit{неточной величиной} или \textit{интервальной величиной}.}\scnrelfrom{описание примера}{\scnfileimage[20em]{figures/sd_temp_entities/duration.png}
}
\scntext{explanation}{В данном примере \textbf{\textit{ki}} обозначает класс сущностей, существовавших в течение 2 месяцев. Конкретный пример такой сущности -- \textbf{\textit{bi}}.}\bigskip
\end{scnsubstruct}
\end{SCn}


\scsubsubsection[
    \protect\scnmonographychapter{Глава 2.4. Формальные онтологии базовых классов сущностей - множеств, связей, отношений, параметров, величин, чисел, структур, темпоральных сущностей}
    ]{Предметная область и онтология ситуаций и событий, описывающих динамику баз знаний ostis-систем}
\label{sd_temp_know_base}
\begin{SCn}
\scnsectionheader{\currentname}
\begin{scnsubstruct}
\scntext{введение}{Обработка информации в \textit{sc-памяти} (т.е. динамика базы знаний, хранимой в \textit{sc-памяти}) в конечном счете сводится:\begin{scnitemize}
\item к появлению в \textit{sc-памяти} новых актуальных \textit{sc-узлов} и \textit{sc-коннекторов};\item к логическому удалению актуальных \textit{sc-элементов}, т.е. к переводу их в неактуальное состояние (это необходимо для хранения протокола изменения состояния базы знаний, в рамках которого могут описываться действия по удалению \textit{sc-элементов});\item к возврату логически удаленных \textit{sс-элементов} в статус актуальных (необходимость в этом может возникнуть при откате базы знаний к какой-нибудь ее прошлой версии);\item к физическому удалению \textit{sc-элементов};\item к изменению состояния актуальных (логически не удаленных \textit{sc-элементов}) -- \textit{sc-узел} может превратиться в \textit{sc-ребро}, \textit{sc-ребро} может превратиться в \textit{sc-дугу}, \textit{sc-дуга} может поменять направленность, \textit{sc-дуга} общего вида может превратиться в \textit{константную стационарную sc-дугу принадлежности}, и т.д.;\end{scnitemize}
Подчеркнем, что временный характер самого \textit{sc-элемента} (т.к. он может появиться или исчезнуть) никак не связан с возможно временным характером сущности, обозначаемой этим \textit{sc-элементом}. Т.е. временный характер самого sc-элемента и временный характер сущности, которую он обозначает -- абсолютно разные вещи.Таким образом, следует четко отличать динамику внешнего мира, описываемого базой знаний, а динамику самой базы знаний (динамику внутреннего мира). При этом очень важно, чтобы описание динамики базы знаний также входило в состав каждой базы знаний.К числу понятий, используемых для описания динамики базы знаний относятся:\begin{scnitemize}
\item логически удаленный sc-элемент;\item сформированное множество;\item вычисленное число;\item сформированное высказывание;\end{scnitemize}
}
\scnheader{Предметная область ситуаций и событий, описывающих динамику баз знаний ostis-систем}
\scnidtf{Предметная область, описывающая динамику базы знаний, хранимой в sc-памяти}
\scniselement{предметная область}
\begin{scnhaselementrole}{класс объектов исследования}
ситуация
\end{scnhaselementrole}
\begin{scnhaselementrolelist}{класс объектов исследования}

sc-элемент;наcтоящий sc-элемент;логически удаленный sc-элемент;число;невычисленное число;вычисленное число;понятие;основное понятие;неосновное понятие;понятие, переходящее из основного в неосновное;понятие, переходящее из неосновного в основное;специфицированная сущность;недостаточно специфицированная сущность;достаточно специфицированная сущность;средне специфицированная сущность;структура;файл;событие в sc-памяти*;элементарное событие в sc-памяти*;событие добавления sc-дуги, выходящей из заданного sc-элемента*;событие добавления sc-дуги, входящей в заданный sc-элемент*;событие добавления sc-ребра, инцидентного заданному sc-элементу*;событие удаления sc-дуги, выходящей из заданного sc-элемента*;событие удаления sc-дуги, входящей в заданный sc-элемент*;событие удаления sc-ребра, инцидентного заданному sc-элементу*;событие удаления sc-элемента*;событие изменения содержимого файла*

\end{scnhaselementrolelist}
\scnheader{sc-элемент}
\begin{scnreltoset}{разбиение}

\scnitem{наcтоящий sc-элемент}
\scnitem{логически удаленный sc-элемент}

\end{scnreltoset}
\scnheader{наcтоящий sc-элемент}
\scniselement{ситуативное множество}
\scnheader{логически удаленный sc-элемент}
\scniselement{ситуативное множество}
\scnheader{число}
\begin{scnsubdividing}

\scnitem{невычисленное число}
\scnitem{вычисленное число}

\end{scnsubdividing}
\scnheader{невычисленное число}
\scniselement{ситуативное множество}
\scnheader{вычисленное число}
\scnheader{понятие}
\begin{scnsubdividing}

\scnitem{основное понятие}
\scnitem{неосновное понятие}
\scnitem{понятие, переходящее из основного в неосновное}
\scnitem{понятие, переходящее из неосновного в основное}

\end{scnsubdividing}
\scnheader{основное понятие}
\scnidtf{основное понятие для данной ostis-системы}
\scniselement{ситуативное множество}
\scntext{explanation}{К \textbf{\textit{основным понятиям}} относятся те понятия, которые активно используются в системе и могут быть ключевыми элементами sc-агентов. К \textbf{\textit{основным понятиям}} относятся также все неопределяемые понятия.}\scnheader{неосновное понятие}
\scnidtf{дополнительное понятие}
\scnidtf{вспомогательное понятие}
\scnidtf{неосновное понятие для данной ostis-системы}
\scniselement{ситуативное множество}
\scntext{explanation}{Каждое \textbf{\textit{неосновное понятие}} должно быть строго определено на основе \textit{основных понятий}. Такие \textbf{\textit{неосновные понятия}} используются только для понимания и правильного восприятия вводимой информации, в том числе, для выравнивания онтологий. Ключевым элементом \textit{sc-агентов} \textbf{\textit{неосновные понятия}} быть не могут.}\scntext{правило идентификации экземпляров}{В случае, когда некоторое понятие полностью перешло из \textit{основных понятий} в неосновные, то есть стало \textbf{\textit{неосновным понятием}}, и соответствующее ему \textit{основное понятие} (через которое оно определяется) в рамках некоторого внешнего языка имеет одинаковый с ним основной идентификатор, то к идентификатору \textbf{\textit{неосновного понятия}} спереди добавляется знак \#. Если при этом соответствуюшее \textit{основное понятие} имеет в идентификаторе знак \$, добавленный в процессе перехода, то этот знак удаляется. Если указанные понятия имеют разные основные идентификаторы в рамках этого внешнего языка, то никаких дополнительных средств идентификации не используется.Например:\\\textit{\#трансляция sc-текста}\\\textit{\#scp-программа}}
\scnheader{понятие, переходящее из основного в неосновное}
\scniselement{ситуативное множество}
\scnheader{понятие, переходящее из неосновного в основное}
\scniselement{ситуативное множество}
\scntext{правило идентификации экземпляров}{В случае, когда текущее \textit{основное понятие} и соответствующее ему \textbf{\textit{понятие, переходящее из неосновного в основное}} в рамках некоторого внешнего языка имеют одинаковый основной идентификатор, то к идентификатору понятия, переходящего из неосновного в основное спереди добавляется знак \$. Если указанные понятия имеют разные основные идентификаторы в рамках этого внешнего языка, то никаких дополнительных средств идентификации не используется.Например:\\\textit{\$трансляция sc-текста}\\\textit{\$scp-программа}}
\scnheader{специфицированная сущность}
\begin{scnsubdividing}

\scnitem{недостаточно специфицированная сущность}
\scnitem{достаточно специфицированная сущность}
\scnitem{средне специфицированная сущность}

\end{scnsubdividing}
\scnheader{достаточно специфицированная сущность}
\scntext{explanation}{К \textbf{\textit{достаточно специфицированным сущностям}} предъявляются следующие требования:\begin{scnitemize}
\item если сущность не является понятием, то для нее должны быть указаны\begin{scnitemizeii}
\item различные варианты обозначающих ее внешних знаков;\item классы, которым она принадлежит;\item связки, которыми она связана с другими сущностями (с указанием соответствующего отношения);\item значения параметров, которыми она обладает;\item те разделы базы знаний, в которых указанная сущность является ключевой;\item предметные области, в которые данная сущность входит.\end{scnitemizeii}
\item если специфицированная сущность является понятием, то для нее должны быть указаны:\begin{scnitemizeii}
\item различные варианты внешних обозначений этого понятия;\item предметные области, в которых это понятие исследуется;\item определение понятия;\item пояснения\item разделы базы знаний, в которых это понятие является ключевым;\item описание примера -- пример экземпляра понятия.\end{scnitemizeii}
\end{scnitemize}
}\scnheader{структура}
\begin{scnsubdividing}

\scnitem{сформированная структура}
\scnitem{несформированная структура}

\end{scnsubdividing}
\begin{scnsubdividing}

\scnitem{недостаточно сформированная структура}
\scnitem{достаточно сформированная структура}
\scnitem{структура, имеющая средний уровень сформированности}

\end{scnsubdividing}
\scnheader{файл}
\begin{scnsubdividing}

\scnitem{недостаточно сформированный внутренний файл}
\scnitem{достаточно сформированный внутренний файл}
\scnitem{внутренний файл, имеющий средний уровень сформированности}

\end{scnsubdividing}
\scnheader{событие в sc-памяти}
\scnsuperset{событие}
\scnheader{элементарное событие в sc-памяти}
\scnsubset{событие в sc-памяти}
\scntext{explanation}{Под \textbf{\textit{элементарным событием в sc-памяти}} понимается такое \textit{событие}, в результате выполнения которого изменяется состояние только одного \textit{sc-элемента}.}\begin{scnsubdividing}

\scnitem{событие добавления sc-дуги, выходящей из заданного sc-элемента}
\scnitem{событие добавления sc-дуги, входящей в заданный sc-элемент}
\scnitem{событие добавления sc-ребра, инцидентного заданному sc-элементу}
\scnitem{событие удаления sc-дуги, выходящей из заданного sc-элемента}
\scnitem{событие удаления sc-дуги, входящей в заданный sc-элемент}
\scnitem{событие удаления sc-ребра, инцидентного заданному sc-элементу}
\scnitem{событие удаления sc-элемента}
\scnitem{событие изменения содержимого файла}

\end{scnsubdividing}
\scnheader{точечный процесс}
\scnidtf{атомарный процесс}
\scnidtf{условно мгновенный процесс}
\scnidtf{процесс, длительность которого в данном контексте считается несущественной (пренебрежимо малой)}
\scnheader{элементарный процесс}
\scnidtf{процесс, детализация которого на входящие в него подпроцессы в текущем контексте не производится}
\bigskip
\end{scnsubstruct}
\end{SCn}


\scsubsection[
    \protect\scnmonographychapter{Глава 2.4. Формальные онтологии базовых классов сущностей - множеств, связей, отношений, параметров, величин, чисел, структур, темпоральных сущностей}
    ]{Предметная область и онтология пространственных сущностей различных форм}
\label{sd_spatial_entities}

\scsubsection[
    \protect\scnmonographychapter{Глава 2.4. Формальные онтологии базовых классов сущностей - множеств, связей, отношений, параметров, величин, чисел, структур, темпоральных сущностей}
    ]{Предметная область и онтология материальных сущностей}
\label{sd_material_entities}

\scsubsection[
    \protect\scnmonographychapter{Глава 2.3. Структура баз знаний интеллектуальных компьютерных систем нового поколения: иерархическая система предметных областей и онтологий. Онтологии верхнего уровня. Формализация понятий семантической окрестности, предметной области и онтологии в интеллектуальных компьютерных системах нового поколения}
    ]{Предметная область и онтология семантических окрестностей}
\label{sd_sem_neigh}
\begin{SCn}
\scnsectionheader{\currentname}
\begin{scnsubstruct}
\scnheader{Предметная область семантических окрестностей}
\scniselement{предметная область}
\begin{scnhaselementrole}{класс объектов исследования}
семантическая окрестность
\end{scnhaselementrole}
\begin{scnhaselementrolelist}{класс объектов исследования}

семантическая окрестность по инцидентным коннекторам;семантическая окрестность по выходящим дугам;семантическая окрестность по выходящим дугам принадлежности;семантическая окрестность по входящим дугам;семантическая окрестность по входящим дугам принадлежности;полная семантическая окрестность;базовая семантическая окрестность;специализированная семантическая окрестность;пояснение;примечание;правило идентификации экземпляров;терминологическая семантическая окрестность;теоретико-множественная семантическая окрестность;описание декомпозиции;логическая семантическая окрестность;спецификация типичного экземпляра;сравнительный анализ

\end{scnhaselementrolelist}
\scnheader{семантическая окрестность}
\scnidtf{sc-окрестность}
\scnidtf{семантическая окрестность, представленная в виде sc-текста}
\scnidtf{sc-текст, являющийся семантической окрестностью некоторого sc-элемента}
\scnidtf{спецификация заданной сущности, знак которой указывается как ключевой элемент этой спецификации}
\scnidtf{описание заданной сущности, знак которой указывается как ключевой элемент этой спецификации}
\scnsubset{знание}
\scnsuperset{семантическая окрестность по инцидентным коннекторам}
\scnsuperset{полная семантическая окрестность}
\scnsuperset{базовая семантическая окрестность}
\scnsuperset{специализированная семантическая окрестность}
\scnidtftext{пояснение}{\textit{знание}, являющееся спецификацией (описанием) некоторой \textit{сущности}, знак которой является \textit{ключевым знаком\scnrolesign} указанного \textit{знания}. Заметим, что каждая \textit{семантическая окрестность} в отличие от \textit{знаний} других видов имеет только один \textit{ключевой знак\scnrolesign} (ключевой элемент\scnrolesign, знак описываемой сущности\scnrolesign). Заметим также, что многообразие видов семантических окрестностей свидетельствует о многообразии семантических видов описаний различных сущностей.}
\scntext{note}{Понятие \textit{семантической окрестности}, как и любой другой \uline{семантически} выделяемый класс \textit{знаний}, абсолютно не зависит от \textit{языка представления знаний}. Этим \textit{языком} может быть не только \textit{SC-код} или другой \textit{формальный язык представления знаний} или даже \textit{естественный язык}, тексты которых в \textit{памяти ostis-системы} представляются в виде \textit{файлов}.}\scnheader{семантическая окрестность по инцидентным коннекторам}
\scnsuperset{семантическая окрестность по выходящим дугам}
\scnsuperset{семантическая окрестность по входящим дугам}
\scnidtftext{пояснение}{вид \textit{семантической окрестности}, в которую входят все коннекторы, инцидентные заданному элементу, а также все элементы, инцидентные указанным коннекторам.}
\scnheader{семантическая окрестность по выходящим дугам}
\scnsuperset{семантическая окрестность по выходящим дугам принадлежности}
\scnidtftext{пояснение}{вид \textit{семантической окрестности}, в которую входят все дуги, выходящие из заданного sc-элемента и вторые компоненты этих дуг. Также указывается факт принадлежности этих дуг каким-либо отношениям.}
\scnheader{семантическая окрестность по выходящим дугам принадлежности}
\scnidtftext{пояснение}{вид \textit{семантической окрестности}, в которую входят все дуги принадлежности, выходящие из заданного \textit{sc-элемента}, а также их вторые компоненты. При необходимости может указывается факт \textit{принадлежности} этих дуг каким-либо \textit{ролевым отношениям}.}
\scnheader{семантическая окрестность по входящим дугам}
\scnsuperset{семантическая окрестность по входящим дугам принадлежности}
\scnidtftext{пояснение}{вид \textit{семантической окрестности}, в которую входят все дуги, входящие в заданный sc-элемент, а также их первые компоненты. Также указывается факт принадлежности этих дуг каким-либо отношениям.}
\scnheader{семантическая окрестность по входящим дугам принадлежности}
\scnidtftext{пояснение}{вид \textit{семантической окрестности}, в которую входят все дуги принадлежности, входящие в заданный sc-элемент, а также их первые компоненты. При необходимости может указывается факт принадлежности этих дуг каким-либо ролевым отношениям.}
\scnheader{полная семантическая окрестность}
\scnidtf{полная спецификация некоторой описываемой сущности}
\scnidtftext{пояснение}{вид \textit{семантической окрестности}, включающий описание всех связей описываемой сущности. Структура полной семантической окрестности определяется прежде всего семантической типологией описываемой сущности. Так, например, для \textit{понятия} в \textit{полную семантическую окрестность} необходимо включить следующую информацию (при наличии):\begin{scnitemize}
\item варианты идентификации на различных внешних языках (sc-идентификаторы);\item принадлежность некоторой \textit{предметной области} с указанием роли, выполняемой в рамках этой предметной области;\item теоретико-множественные связи заданного \textit{понятия} с другими \textit{sc-элементами};\item определение или пояснение;\item высказывания, описывающие свойства указанного \textit{понятия};\item задачи и их классы, в которых данное \textit{понятие} является ключевым;\item описание типичного примера использования указанного \textit{понятия};\item экземпляры описываемого \textit{понятия}.\end{scnitemize}
Для понятия, являющегося отношением дополнительно указываются:\begin{scnitemize}
\item домены;\item область определения;\item схема отношения;\item классы отношений, которым принадлежит описываемое отношение.\end{scnitemize}
}
\scnheader{базовая семантическая окрестность}
\scnidtf{минимально достаточная семантическая окрестность}
\scnidtf{минимальная спецификация описываемой сущности}
\scnidtf{сокращенная спецификация описываемой сущности}
\scnidtf{основная семантическая окрестность}
\scnidtftext{пояснение}{вид \textit{семантической окрестности}, содержащий минимальную (краткую) информацию об описываемой сущностиСтруктура базовой семантической окрестности определяется прежде всего семантической типологией описываемой сущности. Так, например, для \textit{понятия} в базовую семантическую окрестность необходимо включить следующую информацию (при наличии): \begin{scnitemize}
\item варианты идентификации на различных внешних языках (sc-идентификаторы);\item принадлежность некоторой \textit{предметной области} с указанием роли, выполняемой в рамках этой предметной области;\item \textit{определение} или пояснение.\end{scnitemize}
Для \textit{понятия}, являющегося \textit{отношением} дополнительно указываются:\begin{scnitemize}
\item \textit{домены};\item \textit{область определения};\item описание типичного примера связки указанного отношения (спецификация типичного экземпляра).\end{scnitemize}
}
\scnheader{специализированная семантическая окрестность}
\scnsuperset{пояснение}
\scnsuperset{примечание}
\scnsuperset{правило идентификации экземпляров}
\scnsuperset{терминологическая семантическая окрестность}
\scnsuperset{теоретико-множественная семантическая окрестность}
\scnsuperset{логическая семантическая окрестность}
\scnsuperset{описание типичного экземпляра}
\scnsuperset{описание декомпозиции}
\scnidtftext{пояснение}{вид \textit{семантической окрестности}, набор связей для которой уточняется отдельно для каждого типа такой окрестности.}
\scnheader{пояснение}
\scnidtf{sc-пояснение}
\scnidtftext{пояснение}{знак sc-текста, поясняющего описываемую сущность.}
\scnheader{примечание}
\scnidtf{sc-примечание}
\scnidtftext{пояснение}{знак sc-текста, являющегося примечанием к описываемой сущности. В примечании обычно описываются особые свойства и исключения из правил для описываемой сущности.}
\scnheader{правило идентификации экземпляров}
\scnidtf{правило идентификации экземпляров заданного класса}
\scnidtftext{пояснение}{sc-текст являющийся описанием правил построения идентификаторов элементов заданного класса.}
\scnheader{терминологическая семантическая окрестность}
\scnidtftext{пояснение}{\textit{семантическая окрестность}, описывающая внешнюю идентификацию указанной сущности, т.е. её sc-идентификаторы}
\scnheader{теоретико-множественная семантическая окрестность}
\scnidtftext{пояснение}{описание связи описываемого множества с другими множествами с помощью теоретико-множественных отношений}
\scnheader{описание декомпозиции}
\scnidtf{\textit{семантическая окрестность}, описывающая декомпозицию некоторой сущности}
\scnidtftext{пояснение}{\textit{семантическая окрестность}, описывающая декомпозицию некоторой сущности на её части}
\scnheader{логическая семантическая окрестность }
\scnidtftext{пояснение}{\textit{семантическая окрестность}, описывающая семейство высказываний, описывающих свойства данного \textit{понятия} или какого-либо конкретного экземпляра некоторого понятия}
\scnheader{спецификация типичного экземпляра}
\scnidtf{описание типичного экземпляра заданного класса}
\scnidtftext{пояснение}{sc-текст являющийся описанием типичного примера рассматриваемого класса.}
\scnheader{сравнительный анализ}
\scnidtftext{пояснение}{описание сравнения некоторой сущности с другими аналогичными сущностями}
\scnheader{сравнение}
\scnidtftext{пояснение}{описание сравнения (сходств и отличий) двух сущностей, которые заданы \textit{парой} (двухмощными множествами), которому принадлежат знаки обеих сравниваемых сущностей}
\scnheader{семантическая окрестность}
\scntext{note}{Всему классу \textit{семантических окрестностей} и всем подклассам этого \textit{класса}, а так же всем другим классам \textit{знаний}, ставятся в соответствие \textit{бинарные ориентированые отношения}, вторыми \textit{доменами} которых являются указанные \textit{классы} и объединение которых является \textit{обратным отношением} для \textit{отношения} быть ключевым знаком\scnrolesign{}. Эти отношения не следует причислять к основным отношениям, т.к. они вместе с выделенными классами семантических окрестностей привносят дополнительную логическую эквивалентность в базу знаний.}\scnheader{семантическая окрестность}
\begin{scnrelfromlist}{отношение, заданное на}

\scnitem{семантическая окрестность*}
\scnitem{семантическая окрестность по инцидентным коннекторам*}
\scnitem{полная семантическая окрестность*}
\scnitem{базовая семантическая окрестность*}
\scnitem{специализированная синтетическая окрестность*}
\scnitem{и т.д.}

\end{scnrelfromlist}
\scntext{note}{Понятие семантической окрестности, дополненное уточнением таких понятий, как семантическое расстояние между знаками (семантическая близость знаков), радиус семантической окрестности, является перспективной основой для исследования свойств смыслового пространства.}\bigskip
\end{scnsubstruct}
\end{SCn}


\scsubsection[
    \protect\scnmonographychapter{Глава 2.3. Структура баз знаний интеллектуальных компьютерных систем нового поколения: иерархическая система предметных областей и онтологий. Онтологии верхнего уровня. Формализация понятий семантической окрестности, предметной области и онтологии в интеллектуальных компьютерных системах нового поколения}
    ]{Предметная область и онтология предметных областей}
\label{sd_sd}
\begin{SCn}
\scnsectionheader{\currentname}
\begin{scnsubstruct}
\begin{scnreltovector}{конкатенация сегментов}
\scnitem{Что такое предметная область}
\scnitem{Роли знаков, входящих в состав предметных областей}
\scnitem{Типология предметных областей и отношения, заданных на множестве предметных областей}
\scnitem{Что такое sc-язык}
\end{scnreltovector}
\scnheader{Предметная область предметных областей}
\scnidtf{Предметная область, объектами исследования которой являются предметные области}
\scntext{explanation}{В состав \textbf{\textit{Предметной области предметных областей}} входят структурные спецификации всех \textit{предметных областей}, входящих в состав базы знаний \textit{ostis-системы}, в том числе, самой \textbf{\textit{Предметной области предметных областей}}. Таким образом, \textbf{\textit{Предметная область предметных областей}} является, во-первых, \textit{рефлексивным множеством}, во-вторых, рефлексивной предметной областью, то есть \textit{предметной областью}, одним из объектов исследования которой является она сама.}\scniselement{рефлексивное множество}
\begin{scnhaselementrole}{класс объектов исследования}
предметная область\end{scnhaselementrole}
\begin{scnhaselementrolelist}{класс объектов исследования}
статическая предметная область;динамическая предметная область;понятие;sc-язык
\end{scnhaselementrolelist}
\begin{scnhaselementrolelist}{исследуемое отношение}
понятие предметной области\scnrolesign ;исследуемое понятие\scnrolesign ;максимальный класс объектов исследования\scnrolesign ;немаксимальный класс объектов исследования\scnrolesign ;исследуемый класс первичных элементов\scnrolesign ;исследуемое отношение\scnrolesign ;класс исследуемых структур\scnrolesign ;понятие, исследуемое в дочерней предметной области\scnrolesign ;понятие, исследуемое в материнской предметной области\scnrolesign ;вспомогательное понятие\scnrolesign ;дочерняя предметная область*;дочерняя предметная область по классу первичных элементов*;дочерняя предметная область по исследуемым отношениям*;предметная область sc-языка*
\end{scnhaselementrolelist}
\end{scnsubstruct}
\scnsegmentheader{Что такое предметная область}
\begin{scnsubstruct}
\scnheader{предметная область}
\scnidtf{sc-модель предметной области}
\scnidtf{sc-текст предметной области}
\scnidtf{sc-граф предметной области}
\scnidtf{представление предметной области в \textit{SC-коде}}
\scnsubset{знание}
\scnsubset{бесконечное множество}
\scntext{explanation}{\textbf{\textit{предметная область}} -- это результат интеграции (объединения) частичных семантических окрестностей, описывающих все исследуемые сущности заданного класса и имеющих одинаковый (общий) предмет исследования (то есть один и тот же набор отношений, которым должны принадлежать связки, входящие в состав интегрируемых семантических окрестностей).\textbf{\textit{предметная область}} -- \textit{структура}, в состав которой входят:\begin{scnitemize}
\item \textnormal{основные исследуемые (описываемые) объекты -- первичные и вторичные;}\item \textnormal{различные классы исследуемых объектов;}\item \textnormal{различные связки, компонентами которых являются исследуемые объекты (как первичные, так и вторичные), а также, возможно, другие такие связки -- то есть связки (как и объекты исследования) могут иметь различный структурный уровень;}\item \textnormal{различные классы указанных выше связок (то есть отношения);}\item \textnormal{различные классы объектов, не являющихся ни объектами исследования, ни указанными выше связками, но являющихся компонентами этих связок.}\end{scnitemize}
При этом все классы, объявленные исследуемыми понятиями, должны быть полностью представлены в рамках данной предметной области вместе со своими элементами, элементами элементов и т.д. вплоть до терминальных элементов.Можно говорить о типологии \textbf{\textit{предметных областей}} по разным структурным признакам:\begin{scnitemize}
\item наличие метасвязей;\item наличие исследуемых структур, входящих в состав предметной области;\item наличие исследуемых (смежных, дополнительных) объектов, которых исследуются в других предметных областях;\end{scnitemize}
Понятие \textbf{\textit{предметной области}} является важнейшим методологическим приемом, позволяющим выделить из всего многообразия исследуемого Мира только определенный класс исследуемых сущностей и только определенное семейство отношений, заданных на указанном классе. То есть осуществляется локализация, фокусирование внимания только на этом, абстрагируясь от всего остального исследуемого Мира.Во всем многообразии \textbf{\textit{предметных областей}} особое место занимают\begin{scnitemize}
\item \textit{Предметная область предметных областей}, объектами исследования которой являются всевозможные \textbf{\textit{предметные области}}, а предметом исследования -- всевозможные \textit{ролевые отношения}, связывающие предметные области с их элементами, отношения, связывающие предметные области между собой, отношение, связывающее предметные области с их онтологиями\item \textit{Предметная область сущностей}, являющаяся предметной областью самого высокого уровня и задающая базовую семантическую типологию \textit{sc-элементов}(знаков, входящих в тексты \textit{SC-кода})\item Семейство \textbf{\textit{предметных областей}}, каждая из которых задает семантику и синтаксис некоторого \textit{sc-языка}, обеспечивающего представление онтологий соответствующего вида (например, \textit{теоретико-множественных онтологий}, \textit{логических онтологий}, \textit{терминологических онтологий}, \textit{онтологий задач и способов их решения} и т.д.)\item Семейство \textbf{\textit{предметных областей}} верхнего уровня, в которых классами объектов исследования являются весьма крупные классы сущностей. К таким классам, в частности\begin{scnitemizeii}
\item класс всевозможных \textit{материальных сущностей},\item класс всевозможных \textit{множеств},\item класс всевозможных \textit{связей},\item класс всевозможных \textit{отношений},\item класс всевозможных \textit{структур},\item класс всевозможных \textit{временных (временно существующих, непостоянных сущностей) сущностей},\item класс всевозможных \textit{действий} (акций),\item класс всевозможных \textit{параметров} (характеристик),\item класс \textit{знаний} всевозможного вида \item и т.п.\end{scnitemizeii}
\end{scnitemize}
Каждой \textbf{\textit{предметной области}} можно поставить в соответствие:\begin{scnitemize}
\item семейство соответствующих ей \textit{онтологий} разного вида;\item некий язык (в нашем случае -- язык, построенный на основе \textit{SC-кода}), тексты которого представляют различные фрагменты соответствующей предметной области\end{scnitemize}
Указанные языки будем называть \textit{sc-языками}. Их синтаксис и семантика полностью задается \textit{SС-кодом} и \textit{онтологией} соответствующей \textbf{\textit{предметной области}}. Очевидно, что в первую очередь нас должны интересовать те \textit{sc-языки}, которые соответствуют \textbf{\textit{предметным областям}}, имеющим общий (условно говоря, предметно независимый) характер. К таким предметным областям, в частности, относятся:\begin{scnitemize}
\item \textit{Предметная область множеств}, описывающая множества и различные связи между ними\item \textit{Предметная область отношений и соответствий}\item \textit{Предметная область структур} (в частности, графовых)\item \textit{Предметная область чисел и числовых структур}\item и т.д\end{scnitemize}
Каждому типу знаний можно поставить в соответствие предметную область, которая является результатом интеграции всех знаний данного типа. Эти знания и становятся объектами исследования в рамках указанной предметной области.Понятие \textbf{\textit{предметной области}} может рассматриваться как обобщение понятия алгебраической системы. При этом семантическая структура базы знаний может рассматриваться как иерархическая система различных \textbf{\textit{предметных областей}}.}\scnidtf{система связей некоторого множества объектов исследования, \uline{ключевыми} элементами которой являются:\begin{scnitemize}
\item классы (точнее, знаки классов) объектов исследования (объектов, описываемых этой предметной областью);\item конкретные объекты исследования, обладающие особыми свойствами;\item классы связей, входящих в состав рассматриваемой системы -- отношения, заданные на множестве элементов рассматриваемой системы;\item параметры, заданные на множестве элементов рассматриваемой системы;\item классы структур, являющихся фрагментами рассматриваемой системы.\end{scnitemize}
}
\scnidtf{структура, представляющая собой множество связей (точнее, знаков связей) и соответствующее множество компонентов этих связей, к числу которых относится:\begin{scnitemize}
\item элементы (экземпляры) некоторых заданных классов \uline{объектов исследования} (первичных исследуемых сущностей);\item сами связи, входящие в состав указанной структуры;\item введенные классы объектов исследования;\item введенные отношения (классы связей);\item введенные параметры (классы классов эквивалентных сущностей);\item значения параметров (и, в частности, величины для измеряемых параметров);\item введенные структуры, являющиеся фрагментами (подструктурами) рассматриваемой структуры;\item введенные классы подструктур рассматтриваемой структуры.\end{scnitemize}
}
\scntext{note}{Выделяемые в рамках \textit{базы знаний} интеллектуальной системы \textit{предметные области} и соответствующие им \textit{онтологии} -- это, своего рода, семантические страты, кластеры, позволяющие разложить все хранимые в памяти \textit{знания} по семантическим полочкам при наличии четких критериев, позволяющих \uline{однозначно} определить то, на какой полочке должны находиться те или иные \textit{знания}}\scntext{note}{Существуют предметные области, в которых основным исследуемым понятием является множество всевозможных связей между экземплярами понятий, исследуемых в других предметных областях. Так, например, можно ввести Предметную область треугольников, Предметную область окружностей, а также Предметную область связей между треугольниками и окружностями.}
\end{scnsubstruct}
\scnsegmentheader{Роли знаков, входящих в состав предметной области}
\begin{scnsubstruct}
\scnheader{роль элемента предметной области}
\scnidtf{ролевое отношения, связывающее предметные области с их ключевыми знаками}
\scnidtf{роль ключевого элемента (знака ключевой сущностей) предметной области}
\scnidtf{роль ключевого знака предметной области}
\scnhaselement{класс объектов исследования\scnrolesign}
\scnhaselement{максимальный класс объектов исследования\scnrolesign}
\scnhaselement{ключевой объект исследования\scnrolesign}
\scnhaselement{понятие, используемое в предметной области\scnrolesign}
\scnhaselement{первичный исследуемый элемент предметной области\scnrolesign}
\scnhaselement{вторичный исследуемый элемент предметной области\scnrolesign}
\scnhaselement{неисследуемый элемент предметной области\scnrolesign}
\scnheader{класс объектов исследования\scnrolesign}
\scnidtf{быть классом \uline{первичных} (для данной предметной области) объектов исследования\scnrolesign}
\scntext{note}{Понятие \uline{первичного} объекта исследования для предметной области является понятием \uline{относительным} и абсолютно не зависит от типа и уровня сложности этого объекта. Само исследование (спецификация) таких первичных исследуемых объектов осуществляется:\begin{scnitemize}
\item путем введения различных классов объектов исследования, которым эти объекты принадлежат;\item путем введения различных связок из первичных объектов исследования и различных классов таких связок (отношений), которым принадлежат введенные связки;\item путем введения таких классов первичных объектов исследования, которые являются значениями вводимых параметров;\item путем введения различных структур, состоящих из первичных объектов исследования, из связок таких объектов, из введенных отношений и классов первичных объектов, из введенных параметров и значений этих параметров, и путем введения различных классов таких структур;\item путем введения различных связок из вторичных объектов исследования (т.е. из связок и структур) и путем введения различных классов таких связок;\item и далее можно переходить к объектам исследования более высокого уровня сложности, к параметрам, элементами значений которых являются такие объекты, а также к структурам, элементами которых являются объекты такого уровня и, соответственно, к классам таких структур.\end{scnitemize}
}\scnrelfrom{второй домен}{класс}
\scnsuperset{\begin{scnset}
множество;отношение\\\scnsubset{множество}
;параметр\\\scnsubset{класс классов}
;значение параметра\\\scnsubset{класс}
;структура\\\scnsubset{множество}
;темпоральная сущность;темпоральная сущность базы знаний ostis-системы;семантическая окрестность;предметная область;онтология;логическая формула;действие;задача;информационная конструкция;язык;sc-конструкция;кибернетическая система;интеллектуальная компьютерная система;знание;база знаний;решатель задач интеллектуальной компьютерной системы;интерфейс интеллектуальной компьютерной системы;компьютерная система, основанная на смысловом представлении информации;смысловое представление информации;многоагентная модель решения задач, основанная на смысловом представлении информации;логико-семантическая модель интерфейсов компьютерных систем, основанных на смысловом представлении информации;решатель задач ostis-системы;действие, выполняемое ostis-системой;задача, решаемая ostis-системой:план решения задачи, реализуемый ostis-системой;протокол решения задачи, реализованный ostis-системой;метод решения класса задач, реализуемый ostis-системой;sc-агент\\\scnidtf{внутренний агент ostis-системы, осуществляющий выполнение некоторого вида действий в памяти ostis-системы}
\scnsuperset{sc-агент обработки информации в памяти ostis-системы}
\scnsuperset{sc-агент управления внешними действиями ostis-системы}
;Базовый язык программирования ostis-систем\\\scnidtf{Язык SCP}
;искусственная нейронная сеть;интерфейс ostis-системы;интерфейсное действие пользователя ostis-системы;sc-агент интерфейса ostis-системы;естественный язык;базовый интерпретатор логико-семантических моделей ostis-систем;базовый интерпретатор логико-семантических моделей ostis-систем, реализованный программно на современных компьютерах;семантический ассоциативный компьютер;обучение пользователей ostis-систем;ostis-система персональной адаптивной поддержки всех видов деятельности пользователя;ostis-система управления рецептурным производством;ostis-система, реализующая интеллектуальный портал научно-технических знаний
\end{scnset}
}
\scntext{note}{Здесь приведено семейство тех \textit{классов объектов исследования}, для которых в текущей версии \textit{Стандарта OSTIS} представлены соответствующие \textit{предметные области}. Очевидно, что это семейство должно быть существенно расширено и включить в себя, например, такие \textit{классы} сущностей, как:\begin{scnitemize}
\item материальная сущность\item вещество\item физическое поле\item персона\item пространственная сущность\item юридическое лицо\item предприятие\item географический объект\item и многие другие\end{scnitemize}
}\scntext{note}{Особого внимания требуют те \textit{классы объектов исследования}, которые носят наиболее общий характер  которым соответствуют \textit{предметные области и онтологии} \uline{высокого уровня}. Здесь важна продуманная система декомпозиции всего множества окружающих нас сущностей на иерархическую систему \textit{классов объектов исследования}, которой соответствует иерархическая система \textit{предметных областей и онтологий}, определяющая направления \uline{наследования свойств} исследуемых объектов.}\scnheader{максимальный класс объектов исследования\scnrolesign}
\scnidtf{класс объектов исследования, для которого \uline{в заданной} (!) предметной области отсутствует другой класс объектов исследования, который был бы его надмножеством\scnrolesign}
\scntext{note}{В некоторых предметных областях может быть \uline{несколько} максимальных классов объектов исследования}\scnheader{ключевой объект исследования\scnrolesign}
\scnidtf{особый объект исследования\scnrolesign}
\scnidtf{быть знаком особого исследуемого объекта в рамках заданной предметной области\scnrolesign}
\scnidtf{объект исследования, обладающий особыми свойствами\scnrolesign}
\scnhaselementrole{пример}{$\langle$Предметная область чисел; Нуль$\rangle$}
\scntext{note}{Особыми свойствами Числа \textit{Нуль} являются:\begin{scnitemize}
\item Результатом сложения Числа \textbf{\textit{Нуль}} с любым числом \textbf{\textit{x}} является число \textbf{\textit{x}};\item Результатом умножения Числа \textbf{\textit{Нуль}} на любое число является Число \textbf{\textit{Нуль}}\end{scnitemize}
}\scnhaselement{$\langle$Предметная область чисел; Единица$\rangle$}
\scnhaselement{$\langle$Предметная область чисел; Число Пи$\rangle$}
\scnhaselement{$\langle$Предметная область чисел; Число Е$\rangle$}
\scnheader{ключевой элемент предметной области\scnrolesign}
\scnidtf{входящий в состав предметной области знак ключевой сущности\scnrolesign}
\begin{scnsubdividing}
\scnitem{понятие, используемое в предметной области\scnrolesign}
\scnitem{ключевой объект исследования\scnrolesign \\\scnidtf{знак ключевого объекта исследования\scnrolesign}
}
\end{scnsubdividing}
\scnheader{понятие, используемое в предметной области\scnrolesign}
\scnidtf{понятие, используемое в заданной предметной области не в качестве одного из объектов исследования, а в качестве \uline{ключевого} понятия\scnrolesign}
\scnsubset{используемое понятие\scnrolesign}
\scnidtf{понятие, используемое в sc-знании\scnrolesign}
\scnsubset{используемое понятие*}
\scnidtf{понятие, используемое в знании, которое может быть представлено не только в SC-коде*}
\scntext{note}{Уточнение характера использования понятия в предментной области осуществляется по трем признакам:\begin{scnitemize}
\item семантический тип используемого понятия;\item полнота вхождения элементов понятия в данную предметную область;\item наличие первого упоминания понятия;\item наличие определения понятия или объявления его неопределяемостис подробным пояснением и примерами;\item наличие исследования понятия.\end{scnitemize}
}\scnrelfrom{разбиение}{семантический тип используемого понятия}
\begin{scneqtoset}
\scnitem{класс объектов исследования\scnrolesign}
\scnitem{отношение, используемое в предметной области\scnrolesign}
\scnitem{параметр, используемый в предметной области\scnrolesign}
\scnitem{класс структур, используемый в предметной области\scnrolesign}
\end{scneqtoset}
\scnrelfrom{разбиение}{полнота вхождения элементов понятия в данную предметную область}
\begin{scneqtoset}
\scnitem{используемое понятие, все элементы которого входят в данную предметную область\scnrolesign \\\scntext{note}{Для каждого используемого отношения в предметную область здесь должны входить не только знаки связок, но и все связки целиком с их компонентами}}
\scnitem{используемое понятие, не все элементы которого входят в данную предметную область\scnrolesign}
\end{scneqtoset}
\scnrelfrom{разбиение}{наличие первого упоминания понятия}
\begin{scneqtoset}
\scnitem{понятие, вводимое в данной предметной области\scnrolesign}
\scnitem{понятие, которое в данной предметной области используется, но не вводится\scnrolesign}
\end{scneqtoset}
\scntext{note}{Будем считать, что понятие вводится в данной предметной области в том и только в том случае, если ни в одной предметной области более высокого уровня это понятие не используется. Т.е. речь идет о первом упоминании этого понятия в рамках последовательности предметных областей от родительских к дочерним}\scnrelfrom{разбиение}{наличие определения понятия или объявления его неопределяемости с подробным пояснением и примерами}
\begin{scneqtoset}
\scnitem{понятие, которое в данной предметной области определено или объявлено как неопределяемое}
\scnitem{понятие, которое в данной предметной области не имеет ни определения, ни указания факта его неопределяемости}
\end{scneqtoset}
\scnrelfrom{разбиение}{наличие исследования понятия}
\begin{scneqtoset}
\scnitem{понятие, исследуемое в данной предметной области\scnrolesign}
\scnitem{понятие, которое в данной предметной области испольуется, но не исследуется\scnrolesign}
\end{scneqtoset}
\scntext{note}{Понятие, используемое в базе знаний, может быть введено (впервые упомянуто) в одной предметной области, определено в другой, а исследоваться -- в третьей}\scnheader{первичный исследуемый элемент предметной области\scnrolesign}
\scnidtf{знак первичного объекта исследования в рамках заданной предметной области\scnrolesign}
\scnheader{вторичный исследуемый элемент предметной области\scnrolesign}
\scnidtf{знак вторичного объекта исследования в рамках предметной области\scnrolesign}
\scnsuperset{связка элементов предметной области\scnrolesign}
\scnsuperset{связка первичных элементов предметной области\scnrolesign}
\scnsuperset{метасвязка элементов предметной области\scnrolesign}
\scnsuperset{метасвязка, в число компонентов которой входят связки элементов предметной области\scnrolesign}
\scnsuperset{метасвязка, в число компонентов которой входят классы элементов предметной области\scnrolesign}
\scnsuperset{метасвязка, в число компонентов которой входят структуры элементов предметной области\scnrolesign}
\scnsuperset{класс элементов предметной области\scnrolesign}
\scnsuperset{класс первичных элементов предметной области\scnrolesign}
\scnsuperset{класс связок элементов предметной области\scnrolesign}
\scnsuperset{класс классов элементов предметной области\scnrolesign}
\scnsuperset{класс структур элементов предметной области\scnrolesign}
\scnsuperset{структура элементов предметной области\scnrolesign}
\scnsuperset{тривиальная структура первичных элементов предметной области\scnrolesign}
\scnsuperset{структура, в число подмножеств которой входят связки элементов предметной области вместе со своими компонентами\scnrolesign}
\scnsuperset{структура, в число подмножеств которой входят классы элементов предметной области вместе со своими знаками\scnrolesign}
\scnsuperset{структура, в число подмножеств которой входят другие структуры вместе со своими знаками\scnrolesign}
\scnheader{неисследуемый элемент предметной области\scnrolesign}
\scnidtf{вспомогательный элемент предметной области, исследуемый в другой (смежной) предметной области\scnrolesign}
\scntext{note}{С помощью неисследуемых элементов предметной области описываются и исследуются различные вида связи между элементами, исследуемыми в данной \textit{предметной области} с элементами, исследуемыми в других \textit{предметных областях}. При этом \textit{связки}, компонентами которых являются как исследуемые, так и неисследуемые элементы данной \textit{предметной области} считаются \uline{исследуемыми} связками этой \textit{предметной области}. Примерами неисследуемых элементов, напримр, геометрической \textit{предметной области} являются \textit{числа}, являющиеся \textit{значениями величин} таких \textit{параметров}, как \textit{расстояние}\scnsupergroupsign, \textit{длина}\scnsupergroupsign, \textit{площадь}\scnsupergroupsign, \textit{объем}\scnsupergroupsign, а также различные числовые \textit{отношения} (\textit{сложение}*, \textit{умножение}*, \textit{возведение в степень}*), теоретико-множественные \textit{отношения} (\textit{включение}*, \textit{объединение}*, \textit{пересечение}*, \textit{принадлежность}*)}\newpage\scnheader{понятие}
\scnidtf{концепт}
\scnidtf{класс сущностей, который входит в состав по крайней мере одной предметной области в качестве (в роли) ключевого исследуемого понятия}
\scntext{note}{Семейство всех введенных понятий -- это, своего рода, семантическая система координат, позволяющая специфицировать всевозможные сущности в смысловом пространстве.}\scnidtf{класс сущностей, который по крайней мере в одной \textit{предметной области} объявлен как \textit{понятие} (вводимое, исследуемое или вспомогательное)}
\scntext{note}{Каждому \textit{понятию} соответствует по крайней мере одна \textit{предметная область}, в которой это понятие является \textit{исследуемым понятием} и в которой рассматриваются основные характеристики этого \textit{понятия}. Если же в какой-либо \textit{предметной области} необходимо рассмотреть дополнительные связи этого \textit{понятия} с другими \textit{понятиями}, то оно объявляется как \textit{вспомогательное понятия}\scnrolesign .}\scnidtf{Второй домен Отношения \textit{используемое понятие}*}
\scnrelto{второй домен}{используемое понятие*}
\scnidtf{класс сущностей (класс связок (в т.ч. отношение), класс классов (в т.ч. параметр), класс структур), который по крайней мере в одной \textit{предметной области} является \textit{используемым понятием}\scnrolesign}
\end{scnsubstruct}
\scnsegmentheader{Типология предметных областей и отношения, заданные на множестве предметных областей}
\begin{scnsubstruct}
\scnheader{предметная область}
\begin{scnsubdividing}
\scnitem{статическая предметная область\\\scnidtf{стационарная предметная область}
\scnidtf{\textit{предметная область}, в которой связи между сущностями, входящими в ее состав, не зависят от времени (не меняются во времени), элементами \textbf{\textit{статической предметной области}} не могут быть \textit{временные сущности}}
}
\scnitem{квазистатическая предметная область\\\scnidtf{\textit{предметная область}, решение задач в которой не требует учета темпоральных свойств объектов исследования}
}
\scnitem{динамическая предметная область\\\scnidtf{нестационарная предметная область}
\scnidtf{\textit{предметная область}, которая описывает изменение состояния (в том числе внутренней структуры) объектов исследования и/или изменение конфигурации связей между объектами исследования}
\scnidtf{\textit{предметная область}, в которой некоторые связи между сущностями, входящими в ее состав, меняются со временем (то есть носят ситуационный, нестационарный характер, другими словами, являются \textit{временными сущностями})}
}
\end{scnsubdividing}
\begin{scnsubdividing}
\scnitem{первичная предметная область\\\scnidtf{\textit{предметная область}, объектами исследования которой являются \uline{внешние} сущности (обозначаемые первичными \textit{sc-элементами})}
}
\scnitem{вторичная предметная область\\\scnidtf{метапредметная область}
\scnidtf{\textit{предметная область}, объектами исследования которой являются \textit{sc-множества} (отношения, параметры, структуры, классы структур, знания, языки и др.)}
}
\end{scnsubdividing}
\scntext{note}{Во всем многообразии предметных областей \uline{особое} местро занимают:\begin{scnitemize}
\item \textbf{\textit{Предметная область предметных областей}}, объектами исследования которой являются всевозможные предметные области, а предметом исследования являются -- всевозможные ролевые отношения, связывающие предметные области с их элементами, отношения, связывающие предметные области между собой, отношение, связывающее предметные области с их онтологиями;\item \textbf{\textit{Предметная область сущностей}}, являющаяся предметной областью самого высокого уровня и задающая базовую семантическую типологию sc-элементов (знаков, входящих в тексты SC-кода);\item Семейство \textit{предметных областей}, каждая из которых задает семантику и синтаксис некоторого \textit{sc-языка}, обеспечивающего представление \textit{\uline{онтологий}} соответствующего вида (например, теоретико множественных онтологий терминологических онтологий);\item Семейство \textit{предметных областей} \uline{верхнего уровня}, в которых классами объектов исследования являются весьма крупные классы сущностей. К таким классам, в частности, относятся: \begin{scnitemizeii}
\item класс всевозможных материальных сущностей,\item класс всевозможных множеств,\item класс всевозможных связей,\item класс всевозможных отношений,\item класс всевозможных структур,\item класс всевозможных темпоральных (нестационарных) сущностей,\item класс всевозможных действий (воздествий, акций),\item класс всевозможных параметров (характеристик),\item класс знаний всевозможного вида и т.п.;\end{scnitemizeii}
\item Предметные области абстрактных пространств (в том числе предметные области метрических пространств). Примерами абстрактного пространства являются Евклидово пространство геометрических точек и фигур, пространство всевозможных множеств, числовое пространство, SC-пространство (унифицированное смысловое пространство знаков всевозможных сущностей).\end{scnitemize}
}\scnheader{отношение, заданное на множестве предметных областей}
\scnhaselement{\scnkeyword{дочерняя предметная область*}
}
\scnidtf{частная предметная область*}
\scnidtf{быть частной предметной областью*}
\scnidtf{близлежащий потомок предметной области*}
\scnidtf{сужение предметной области по классу объектов исследования*}
\scnidtf{предметная область, детализирующая описание одного из классов объектов исследования другой (более общей) предметной области*}
\scnidtf{предметная область, объединение классов объектов исследования которой является подмножеством объединения классов объектов исследования заданной предметной области*}
\scniselement{бинарное отношение}
\scniselement{ориентированное отношение}
\scniselement{неролевое отношение}
\scnsuperset{частная предметная область по классу первичных элементов*}
\scnsuperset{частная предметная область по исследуемым отношениям*}
\scntext{explanation}{\textit{дочерняя предметная область*} -- бинарное ориентированное отношение, с помощью которого задается иерархия предметных областей путем перехода от менее детального к более детальному рассмотрению соответствующих исследуемых понятий.}\scntext{note}{Для любой \textit{предметной области} все свойства ее \textit{объектов исследования} \uline{наследуются} всеми ее \textit{дочерними предметными областями*}.}\scnhaselement{\scnkeyword{интеграция предметных областей*}
}
\scnidtf{Отношение, связывающее заданное семейство предметных областей с предметной областью, которая является результатом их интеграции (это не только теоретико-множественное объединение заданных предметных областей, но и уточнение ролей ключевых понятий в интегрированной предметной области, поскольку одно и то же понятие в интегрируемых предметных областях может иметь разные роли).}
\scnhaselement{\scnkeyword{изоморфность предметных областей*}
}
\scnhaselement{\scnkeyword{гомоморфность предметных областей*}
}
\scnheader{расширение семейства исследуемых отношений*}
\scntext{explanation}{Переход от одной предметной области к предметной области с тем же максимальным классомобъектов исследования, но с расширенным семейством отношений и, возможно, с расширенным семейством явно выделенных классов объектов исследования (подклассов максимального класса).}\scnheader{переход к рассмотрению внутренней структуры объектов исследования*}
\scntext{explanation}{Переход от рассмотрения внешних связей объектов исследования к рассмотрению их внутренней структуры путем декомпозиции исследуемых объектов на части и путем включения в число исследуемых объектов тех, которые являются указанными частями.}\scnheader{переход к рассмотрению структур из объектов исследования*}
\scntext{explanation}{Переход от описаниязаданного класса исследуемых объектов к описанию класса всевозможных множеств, элементами которых являются указанные объекты (например, переход от предметной области геометрических точек к предметной области геометрических фигур).}
\end{scnsubstruct}
\end{SCn}


\scsubsection[
    \protect\scnmonographychapter{Глава 2.3. Структура баз знаний интеллектуальных компьютерных систем нового поколения: иерархическая система предметных областей и онтологий. Онтологии верхнего уровня. Формализация понятий семантической окрестности, предметной области и онтологии в интеллектуальных компьютерных системах нового поколения}
    ]{Предметная область и онтология онтологий}
\label{sd_ontologies}
\begin{SCn}
\scnsectionheader{\currentname}
\begin{scnsubstruct}
\begin{scnreltovector}{конкатенация сегментов}
\scnitem{Что такое онтология}
\scnitem{Типология онтологий предметной области}
\scnitem{Понятие объединенной онтологии предметной области, понятие предметной области и онтологии}
\scnitem{Отношения, заданные на множестве онтологий}
\end{scnreltovector}
\scnheader{Предметная область \textit{онтологий}}
\scnidtf{Предметная область теории \textit{онтологий}}
\scnidtf{Предметная область, объектами исследования которой являются \textit{онтологии}}
\scniselement{предметная область}
\begin{scnhaselementrole}{класс объектов исследования}
онтология\end{scnhaselementrole}
\begin{scnhaselementrolelist}{класс объектов исследования}
\begin{itemize}
  \item объединенная онтология
  \item структурная спецификация предметной области
  \item теоретико-множественная онтология предметной области
  \item логическая онтология предметной области
  \item логическая иерархия понятий предметной области
  \item логическая иерархия высказываний предметной области
  \item терминологическая онтология предметной области
\end{itemize}
\end{scnhaselementrolelist}
\begin{scnhaselementrolelist}{исследуемое отношение}
\begin{itemize}
  \item онтология*
  \item используемые константы*
  \item используемые утверждения*
\end{itemize}
\end{scnhaselementrolelist}
\end{scnsubstruct}
\scnsegmentheader{Что такое онтология}
\begin{scnsubstruct}
\scnheader{онтология}
\scnidtf{sc-онтология}
\scnidtf{онтология, представленная в SC-коде}
\scntext{note}{Поскольку термин ``\textit{онтология} в SC-коде соответствует множеству всевозможных онтологий, представленных в SC-коде, то для формальных онтологий, представленных на других языках, необходимо использовать sc-идентификатор, содержащий явное указание этих языков, например, ``owl-онтология.}\scnidtf{sc-текст онтологии}
\scnidtf{sc-модель онтологии}
\scnidtf{семантическая спецификация \textit{знаний}}
\scnidtf{семантическая спецификация любого знания, имеющего достаточно сложную структуру, любого целостного фрагмента базы знаний -- предметной области, метода решения сложных задач некоторого класса, описания истории некоторого вида деятельности, описания области выполнения некоторого множества действий (области решения задач), языка  представления методов решения задач и т.д.}
\scnidtftext{explanation}{\uline{семантическая} \textit{спецификация} некоторого достаточно информативного ресурса (\textit{знания})}
\scnsubset{спецификация}
\scntext{note}{Если \textit{спецификация} может специфицировать (описывать) любую \textit{сущность}, то \textit{отнология} специфицирует только различные \textit{знания}. При этом наиболее важными объектами такой спецификации являются \textit{предметные области}}\scnsubset{метазнание}
\scniselement{вид знаний}
\scntext{note}{\textit{онтологии} являются важнейшим \textit{видом знаний} (точнее, метазнаний), обеспечивающих семантическую систематизацию \textit{знаний}, хранимых в памяти \textit{интеллектуальных компьютерных систем} (в т.ч. \textit{ostis-систем}), и, соответственно, семантическую структуризацию \textit{баз знаний}}\scnidtf{важнейший вид \textit{метазнаний}, входящих в состав базы знаний}
\scnidtf{спецификация (уточнение) системы \textit{понятий}, используемых в соответствующем (специфицируемом) \textit{знании}}
\scntext{эпиграф}{Определив точно значения слов, вы избавите человечество от половины заблуждений}
\scnrelfrom{автор}{Рене Декарт}
\scntext{explanation}{\textit{онтология} включает в себя: \begin{scnitemize}
\item типологию специфицируемого \textit{знания};\item связи специфицируемого \textit{знания} с другими \textit{знаниями};\item спецификацию ключевых \textit{понятий}, используемых в специфицируемом \textit{знании}, а также ключевых экземпляров некоторых таких \textit{понятий}.\end{scnitemize}
}\scntext{explanation}{Основная \textit{цель} построения \textit{онтологии} -- семантическое уточнение (пояснение, а в идеале -- определение) такого семейства \textit{знаков}, используемых в заданном \textit{знании}, которых достаточно для понимания смысла всего специфицируемого \textit{знания}. Как выясняется, количество \textit{знаков}, смысл которых определяет смысл всего специфицируемого \textit{знания}, \uline{не является большим}.}\begin{scnsubdividing}
\scnitem{неформальная онтология}
\scnitem{формальная онтология\scnidtf{онтология, представленная на формальном языке}
\scnrelto{ключевой знак}{\cite{Loukashevich2011}}
}
\end{scnsubdividing}
\scnheader{формальная онтология}
\scnidtf{формальное описание \uline{денотационной семантики} (семантической интерпретации) специфицируемого знания}
\scntext{note}{Очевидно, что при отсутствии достаточно полных формальных онтологий невозможно обеспечить семантическую совместимость (интегрируемость) различных знаний, хранимых в базе знаний, а также приобретаемых извне.}\scnheader{онтология предметной области}
\scntext{note}{\textit{онтология} чаще всего трактуется как спецификация концептуализации (спецификация системы \textit{понятий}) заданной \textit{предметной области}. Здесь имеется в виду описание теоретико-множественных связей (прежде всего, классификации) используемых \textit{понятий}, а также описание различных закономерностей для сущностей, принадлежащих этим \textit{понятиям}. Тем не менее, важными видами спецификации \textit{предметной области} являются также: \begin{scnitemize}
\item описание связей специфицируемой \textit{предметной области} с другими \textit{предметными областями};\item описание терминологии специфицируемой \textit{предметной области}.\end{scnitemize}
}\scntext{note}{\textit{онтологию предметной области} можно трактовать, с одной стороны, как \textit{семантическую окрестность} соответствующей \textit{предметной области}, с другой стороны, как \textit{объединение} определённого вида \textit{семантических окрестностей} всех \textit{понятий}, используемых в рамках указанной \textit{предметной области}, а также, возможно, ключевых экземпляров указанных \textit{понятий}, если таковые экземпляры имеются}\scntext{explanation}{Каждая конкретная онтология заданного вида представляет собой семантическую окрестность соответствующей (специфицируемой) предметной области.Каждому \textit{виду онтологий} однозначно соответствует \textit{предметная область}, фрагментами которые являются конкретные \textit{онтологии} этого вида. Следовательно, каждому \textit{виду онтологий} соответствует свой специализированный sc-язык, обеспечивающий представление \textit{онтологий} этого вида.}\scnidtf{описание \textit{денотационной семантики} языка, определяемого (задаваемого) соответствующей (специфицируемой) \textit{предметной области}}
\scnidtf{информационная надстройка (метаинформация) над соответствующей (специфицируемой) \textit{предметной областью}, описывающая различные аспекты этой \textit{предметной области} как достаточно крупного, самодостаточного и семантически целостного фрагмента \textit{база знаний}}
\end{scnsubstruct}
\scnsegmentheader{Типология онтологий предметной области}
\begin{scnsubstruct}
\scnheader{онтология предметной области}
\begin{scnsubdividing}
\scnitem{частная онтология предметной области\scnidtf{\textit{онтология}, представляющая спецификацию соответствующей \textit{предметной области} в том или ином аспекте}
}
\scnitem{объединённая онтология предметной области\scnidtf{онтология \textit{предметной области}, являющаяся результатом объединения всех известных \textit{частных онтологий} этой предметной области}
}
\end{scnsubdividing}
\scnheader{частная онтология предметной области}
\scntext{note}{Каждая \textit{частная онтология} является фрагментом \textit{предметной области}, включающей в себя \uline{все}(!) частные онтологии, принадлежащие соответствующему \textit{виду онтологии}. При этом указанная \textit{предметная область}, в свою очередь, также имеет соответствующую ей \textit{онтологию}, которая уже является не метазнанием (как любая онтология), а метаметазнанием (спецификацией метазнания).}\scnrelfrom{разбиение}{вид онтологий предметных областей}
\begin{scneqtoset}
\scnitem{структурная спецификация предметной области\\\scnidtf{sc-окрестность (sc-спецификация) заданной предметной области в рамках \textit{Предметной области предметных областей}}
\scnidtf{схема предметной области}
}
\scnitem{теоретико-множественная онтология предметной области\scnidtf{sc-спецификация заданной предметной области в рамках \textit{Предметной области множеств}}
}
\scnitem{логическая онтология предметной области \scnidtf{sc-текст формальной теории заданной предметной области}
}
\scnitem{терминологическая онтология предметной области}
\end{scneqtoset}
\scnheader{вид онтологий предметных областей}
\scnidtf{вид спецификаций \textit{предметных областей}}
\scnidtf{вид \textit{метазнаний}, описывающих соответствующие этому виду \textit{метазнаний} свойства \textit{предметных областей}}
\scntext{note}{Каждому виду \textit{онтологий предметных областей}, представленных в \textit{SC-коде} (здесь представленными в \textit{SC-коде} предполагаются не только сами \textit{онтологии}, но и специфицируемые ими \textit{предметные области}) ставится в соответствие \textit{sc-метаязык}, обеспечивающий представление \textit{метазнаний}, входящих в состав указанного \textit{вида онтологий}. Кроме того, обычное теоретико-множественное \textit{объединение} всех \textit{sc-текстов} указанного \textit{sc-метаязыка} означает построение \textit{предметной области} (предметной метаобласти), которая формально задаёт соответствующий \textit{вид онтологии} и которая будет иметь свою \textit{онтологию}. При этом каждую частную онтологию можно связать с той предметной областью, фрагментом которой эта онтология является.}\scnheader{логическая онтология предметной области}
\scnrelto{семейство подмножеств}{Предметная область логических формул, высказываний и формальных теорий}
\scnheader{теоретико-множественная онтология предметной области}
\scnrelto{семейство подмножеств}{Предметная область множеств}
\scnheader{структурная спецификация предметной области}
\scnrelto{семейство подмножеств}{Предметная область предметных областей}
\scnheader{структурная спецификация предметной области}
\scnidtf{структурная онтология предметной области}
\scnidtf{ролевая структура ключевых элементов предметной области}
\scnidtf{схема ролей понятий предметной области и её связи со смежными предметными областями}
\scnidtf{схема предметной области}
\scnidtf{спецификация предметной области с точки зрения теории графов и теории \textit{алгебраических систем}}
\scnidtf{описание внутренней (ролевой) структуры \textit{предметной области}, а также её внешних связей с другими \textit{предметными областями}}
\scnidtf{описание ролей ключевых элементов предметной области (прежде всего, понятий -- концептов), а также место специфицируемой предметной области в множестве себе подобных}
\scnidtf{\textit{семантическая окрестность} знака \textit{предметной области} в рамках самой этой \textit{предметной области}, включающая в себя все \textit{ключевые знаки}, входящие в состав \textit{предметной области} (ключевые понятия и ключевые объекты исследования предметной области) с указанием их ролей (свойств) в рамках этой \textit{предметной области} и \textit{семантическая окрестность} знака специфицируемой \textit{предметной области} в рамках \textit{Предметной области предметных областей}, включающая в себя связи специфицируемой \textit{предметной области} с другими семантически близкими ей \textit{предметными областями} (дочерними и родительскими, аналогичными в том или ином смысле (например, изоморфными), имеющими одинаковые \textit{классы объектов исследования} или одинаковые наборы \textit{исследуемых отношений})}
\scnheader{теоретико-множественная онтология предметной области}
\scnidtf{\textit{семантическая окрестность} специфицируемой \textit{предметной области} в рамках \textit{Предметной области множеств}, описывающая теоретико-множественные связи между \textit{понятиями} специфицируемой \textit{предметной области}, включая связи \textit{отношений} с их \textit{областями определения} и \textit{доменами}, связи используемых \textit{параметров} и классов структур их \textit{областями определения}}
\scnidtf{онтология описывающая:\begin{scnitemize}
\item классификацию объектов исследования специфицируемой предметной области;\item соотношение областей определения и доменов используемых отношений с выделенным классами объектов исследования, а также с выделенными классами вспомогательных (смежных) объектов, не являющихся объектами исследования в специфицируемой предметной области;\item спецификацию используемых отношений и, в том числе, указание того, все ли связки этих отношений входят в состав специфицируемой предметной области\end{scnitemize}
}
\scntext{explanation}{теоретико-множественная онтология предметной области включает в себя:\begin{scnitemize}
\item теоретико-множественные связи (в т.ч. таксономию) между всеми используемыми понятиями, входящими в состав специфицируемой предметной области;\item теоретико-множественную спецификацию всех \textit{отношений}, входящих в состав специфицируемой предметной области (ориентированность, арность, область определения, домены и т.д.);\item теоретико-множественную спецификацию всех параметров, используемых в предметной области (области определения параметров, шкалы, единицы измерения, точки отсчета);\item теоретико-множественную спецификацию всех используемых классов структур\end{scnitemize}
}\scnheader{логическая онтология предметной области}
\scnidtf{формальная теория заданной (специфицируемой) предметной области, описывающая с помощью переменных, кванторов, логических связок, формул различные свойства экземпляров понятий, используемых в специфицируемой предметной области}
\scnidtftext{explanation}{онтология предметной области, которая включает в себя:\begin{scnitemize}
\item формальные определения всех понятий, которые в рамках специфицируемой предметной области являются определяемыми;\item неформальные пояснения и некоторые формальные спецификации (как минимум, примеры) для всех понятий, которые в рамках специфицируемой предметной области являются неопределяемыми;\item иерархическую систему понятий, в которой для каждого понятия, исследуемого в специфицируемой предметной области либо указывается факт неопределяемости этого понятия, либо указываются все понятия, на основе которых даётся определение данному понятию.В результате этого множество исследуемых понятий разбивается на ряд уровней: \begin{scnitemizeii}
\item неопределяемые понятия;\item понятия 1-го уровня, определяемые только на основе неопределяемых понятий;\item понятия 2-го уровня, определяемые на основе понятий, изменяющих 1-й уровень и ниже;\item и т.д.\end{scnitemizeii}
\item формальную запись всех аксиом, т.е. высказываний, которые не требуют доказательств;\item формальную запись высказываний, истинность которых требует обоснования (доказательства);\item формальные тексты доказательства истинности высказываний, представляющие собой спецификацию последовательности шагов соответствующих рассуждений (шагов логического вывода, применения различных правил логического вывода);\item иерархическую систему высказываний, в которой для каждого высказывания, истинного по отношению к специфицируемой предметной области, либо указывается аксиоматичность этого высказывания, либо перечисляются \uline{все} высказывания, на основе которых доказывается данное высказывание. В результате этого множество высказываний, истинных по отношению к специфицируемой предметной области, разбивается на ряд уровней:\begin{scnitemizeii}
\item аксиомы;\item высказывания 1-го уровня, доказываемые только на основе аксиом;\item высказывания 2-го уровня, доказываемые на основе высказываний, находящихся на 1-м уровне и ниже.\end{scnitemizeii}
\item формальная запись гипотетических высказываний;\item формальное описание логико-семантической типологии высказываний -- высказываний о существовании, о несуществовании, об однозначности, высказывания определяющего  типа (которые можно использовать в качестве определений соответствующих понятий);\item формальное описание различного вида логико-семантических связей между высказываниями (например, между высказыванием и его обобщением);\item формальное описание аналогии\begin{scnitemizeii}
\item между определениями;\item между высказываниями любого вида;\item между доказательствами различных высказываний.\end{scnitemizeii}
\end{scnitemize}
}
\scnheader{терминологическая онтология предметной области}
\scnidtf{онтология, описывающая \uline{правила построения} терминов (sc-идентификаторов), соответствующих \mbox{sc-элементам}, принадлежащим специфицируемой предметной области, а также описывающая различного рода терминологические связи между используемыми терминами, характеризующие происхождение этих терминов}
\scnidtf{система терминов заданной предметной области}
\scnidtf{тезаурус соответствующей предметной области}
\scnidtf{словарь соответствующей (специфицируемой) предметной области}
\scnidtf{фрагмент глобальной \textit{Предметной области sc-идентификаторов} (внешних идентификаторов sc-элементов), обеспечивающий терминологическую спецификацию некоторой предметной области}
\bigskip
\end{scnsubstruct}
\end{SCn}

\scsubsection[
    \protect\scneditors{Василевская А.П.;Зотов Н.В.;Орлов М.К.}
    \protect\scnmonographychapter{Глава 2.5. Смысловое представление логических формул и высказываний в различного вида логиках}
    ]{Предметная область и онтология логических формул, высказываний и логических sc-языков}
\label{sd_logics}
\begin{SCn}
\scnsectionheader{\currentname}
\begin{scnsubstruct}
\scnheader{Предметная область логических формул, высказываний и формальных теорий}
\scniselement{предметная область}
\begin{scnhaselementrole}{класс объектов исследования}
формальная теория\end{scnhaselementrole}
\begin{scnhaselementrolelist}{класс объектов исследования}
\begin{itemize}
\item Высказывание
\item Атомарное высказывание
\item Неатомарное высказывание
\item Фактографическое высказывание
\item Логическая формула
\item Атомарная логическая формула
\item Неатомарная логическая формула
\item Утверждение
\item Определение
\item Общезначимая логическая формула
\item Противоречивая логическая формула
\item Нейтральная логическая формула
\item Выполнимая логическая формула
\item Невыполнимая логическая формула
\item Тавтология
\item Квантор
\item Формула существования
\item Число значений переменной
\item Кратность существования
\item Единственное существование
\item Логическая формула и единственность
\item Открытая логическая формула
\item Замкнутая логическая формула
\end{itemize}
\end{scnhaselementrolelist}
\begin{scnhaselementrolelist}{исследуемое отношение}
\begin{itemize}
\item предметная область\scnrolesign
\item аксиома\scnrolesign
\item теорема\scnrolesign
\item подформула*\scnrolesign
\item логическая связка*\scnrolesign
\item импликация*\scnrolesign
\item если\scnrolesign
\item то\scnrolesign
\item эквиваленция*\scnrolesign
\item конъюнкция*\scnrolesign
\item дизъюнкция*\scnrolesign
\item строгая дизъюнкция*\scnrolesign
\item отрицание*\scnrolesign
\item всеобщность*\scnrolesign
\item неатомарное существование*\scnrolesign
\item связываемые переменные\scnrolesign
\end{itemize}
\end{scnhaselementrolelist}
\scnheader{формальная теория}
\scntext{explanation}{\textbf{\textit{формальная теория}}  это множество высказываний, которые считаются истинными в рамках данной \textbf{\textit{формальной теории}}.}\scntext{explanation}{Высказывания могут быть как фактографическими, так и логическими формулами. Некоторые высказывания считаются аксиомами, а другие доказываются на основе других высказываний в рамках этой же \textbf{\textit{формальной теории}}.}\scntext{explanation}{Каждая формальная теория интерпретируется (т.е. ее высказывания являются истинными) на какой-либо \textit{предметной области}, которая является максимальным из \textit{фактографических высказываний} (их \textit{объединением*}),  входящих в состав этой \textbf{\textit{формальной теории}}.}\scntext{explanation}{Каждой \textbf{\textit{формальной теории}} соответствует одна \textit{предметная область}, которая входит в нее под атрибутом \textit{предметная область\scnrolesign}.}\scntext{explanation}{Каждая \textbf{\textit{формальная теория}} может рассматриваться как конъюнктивное высказывание, априори истинное (с чьей-то точки зрения) при интерпретации на соответствующей \textit{предметной области}.}\scntext{explanation}{Каждая \textbf{\textit{формальная теория}} задаётся алфавитом, формулами, аксиомами, правилами вывода.}\scnrelfrom{источник}{\cite{Serhievskaya2004}}
\scnheader{предметная область\scnrolesign}
\scniselement{ролевое отношение}
\scntext{explanation}{\textbf{\textit{предметная область\scnrolesign}} -- это \textit{ролевое отношение}, связывающее \textit{формальную теорию} с \textit{предметной областью}, на которой данная \textit{формальная теория} интерпретируется (в рамках которой истинны \textit{высказывания}, входящие в состав этой \textit{формальной теории}).}\scntext{explanation}{\textit{Предметная область} является максимальным фактографическим высказыванием \textit{формальной теории}, которая интерпретируется на данной \textit{предметной области}.}\scnrelfrom{смотрите}{\nameref{sd_sd}}
\scnheader{аксиома\scnrolesign}
\scniselement{ролевое отношение}
\scntext{explanation}{\textbf{\textit{аксиома\scnrolesign}} -- это \textit{ролевое отношение}, связывающее \textit{формальную теорию} с \textit{высказыванием}, истинность которого не  требует доказательства в рамках этой \textit{формальной теории}.}\scnheader{теорема\scnrolesign}
\scniselement{ролевое отношение}
\scntext{explanation}{\textbf{\textit{теорема\scnrolesign}} -- это \textit{ролевое отношение}, связывающее \textit{формальную теорию} с \textit{высказыванием}, истинность которого доказывается в рамках этой \textit{формальной теории}.}\scnheader{высказывание}
\begin{scnsubdividing}
\scnitem{атомарное высказывание}
\scnitem{неатомарное высказывание}
\end{scnsubdividing}
\begin{scnsubdividing}
\scnitem{фактографическое высказывание}
\scnitem{логическая формула}
\end{scnsubdividing}
\scntext{explanation}{Под \textbf{\textit{высказыванием}} понимается некоторая \textit{структура} (в которую входят \textit{sc-константы} из некоторой предметной области и/или \textit{sc-переменные}) или \textit{логическая связка}, которая может трактоваться как истинная или ложная в рамках какой-либо \textit{предметной области}.}\scntext{note}{Истинность \textbf{\textit{высказывания}} задается путем указания принадлежности знака этого высказывания \textit{формальной теории}, соответствующей данной \textit{предметной области}. Ложность высказывания задается путем указания принадлежности знака \textit{отрицания*} этого высказывания данной \textit{формальной теории}.}\scntext{note}{Явно указанная непринадлежность \textbf{\textit{высказывания}} \textit{формальной теории} может говорить как о его ложности в рамках данной теории (если это указано рассмотренным выше образом), так и о том, что данное  \textbf{\textit{высказывание}} вообще не рассматривается в данной \textit{формальной теории} (например, использует понятия, не принадлежащие данной \textit{предметной области}).}\scntext{note}{Одно и то же \textbf{\textit{высказывание}} может быть истинно в рамках одной \textit{формальной теории} и ложно в рамках другой.}\scnheader{высказывание формальной теории\scnrolesign}
\scniselement{неосновное понятие}
\begin{scnsubdividing}
\scnitem{истинное высказывание\scnrolesign\\\scnidtf{высказывание, истинное в рамках данной формальной теории\scnrolesign}
\scnidtf{высказывание, знак которого принадлежит данной формальной теории\scnrolesign}
}
\scnitem{ложное высказывание\scnrolesign\\\scnidtf{высказывание, ложное в рамках данной формальной теории\scnrolesign}
\scnidtf{высказывание, знак отрицания которого принадлежит данной формальной теории\scnrolesign}
}
\scnitem{нечеткое высказывание\scnrolesign\\\scnidtf{гипотетическое высказывание\scnrolesign}
\scnidtf{высказывание, возможно истинное или ложное в рамках данной формальной теории\scnrolesign}
\scnidtf{высказывание, истинное или ложное в рамках данной формальной теории с некоторой вероятностью\scnrolesign}
}
\scnitem{бессмысленное высказывание\scnrolesign\\\scnidtf{высказывание, бессмысленное в рамках данной формальной теории\scnrolesign}
\scnidtf{высказывание, не рассматриваемое в рамках данной формальной теории\scnrolesign}
\scntext{explanation}{Высказывание является бессмысленным в рамках заданной формальной теории, если в какое-либо \textit{атомарное высказывание} в его составе (или в само это высказывание, если оно является атомарным) входит какая-либо \textit{sc-константа}, не являющаяся элементом предметной области, описываемой указанной \textit{формальной теорией}.}}
\end{scnsubdividing}
\scnheader{атомарное высказывание}
\scnsubset{структура}
\begin{scnsubdividing}
\scnitem{атомарное фактографическое высказывание}
\scnitem{атомарная логическая формула}
\end{scnsubdividing}
\scntext{definition}{\textbf{\textit{атомарное высказывание}} -- это \textit{высказывание}, которое содержит хотя бы один \textit{sc-элемент}, не являющийся знаком другого \textit{высказывания}.}\scnheader{неатомарное высказывание}
\scntext{definition}{\textbf{\textit{неатомарное высказывание}} -- это \textit{высказывание}, в состав которого входят только знаки других \textit{высказываний}.}\scntext{note}{Следует отметить, что мы не можем говорить об истинности либо ложности \textbf{\textit{неатомарного высказывания}} в рамках какой-либо \textit{формальной теории}, в случае, когда невозможно установить истинность либо ложность любого из его элементов в рамках этой же \textit{формальной теории}.}\scnheader{фактографическое высказывание}
\scnsuperset{атомарное фактографическое высказывание}
\scntext{explanation}{Под \textit{фактографическим высказыванием} понимается:\begin{scnitemize}
\item \textit{атомарное высказывание}, в состав которого не входит ни одна \textit{sc-переменная};\item \textit{неатомарное высказывание}, все элементы которого также являются \textbf{\textit{фактографическими высказываниями}}.\end{scnitemize}
}\scnheader{логическая формула}
\scntext{explanation}{Под \textit{логической формулой} понимается:\begin{scnitemize}
\item \textit{атомарное высказывание}, в состав которого входит хотя бы одна \textit{sc-переменная};\item \textit{неатомарное высказывание}, хотя бы один элемент которого является \textbf{\textit{логической формулой}}.\end{scnitemize}
}\begin{scnsubdividing}
\scnitem{атомарная логическая формула}
\scnitem{неатомарная логическая формула}
\end{scnsubdividing}
\begin{scnsubdividing}
\scnitem{открытая логическая формула}
\scnitem{замкнутая логическая формула}
\end{scnsubdividing}
\scnheader{атомарная логическая формула}
\scnidtf{обобщенная структура}
\scnidtf{атомарная формула существования}
\scntext{explanation}{Под \textbf{\textit{атомарной логической формулой}} понимается \textit{атомарное высказывание}, которое является \textit{логической формулой}.}\scntext{explanation}{\textbf{\textit{Атомарная логическая формула}} -- это  логическая формула, которая не содержит логических связок.}\scntext{note}{По умолчанию \textbf{\textit{атомарная логическая формула}} трактуется как \textit{высказывание} о существовании, то есть наличия в памяти значений, соответствующих всем \textit{sc-переменным}, входящим в состав данной формулы и не попадающих под действие какого-либо другого \textit{квантора} (указанного явно или по умолчанию). Таким образом, на все \textit{sc-переменные}, входящие в состав \textbf{\textit{атомарной логической формулы}} и не попадающие под действие какого-либо другого \textit{квантора}, неявно накладывается квантор \textit{существования*}.}\scnrelfrom{основной sc-идентификатор}{\scnfilelong{Примечание про высказывание о существовании}
}
\scnheader{неатомарная логическая формула}
\begin{scnsubdividing}
\scnitem{общезначимая логическая формула}
\scnitem{противоречивая логическая формула}
\scnitem{нейтральная логическая формула}
\end{scnsubdividing}
\begin{scnsubdividing}
\scnitem{выполнимая логическая формула}
\scnitem{невыполнимая логическая формула}
\end{scnsubdividing}
\scnsuperset{тавтология}
\scntext{explanation}{Под \textbf{\textit{неатомарной логической формулой}} понимается \textit{неатомарное высказывание}, которое является \textit{логической формулой}.}\scntext{note}{Для того, чтобы рассмотреть типологию \textbf{\textit{неатомарных логических формул}}, будем говорить, что исследуется истинность самой \textbf{\textit{неатомарной логической формулы}} и всех ее \textit{подформул*} в рамках одной и той же \textit{формальной теории}, при этом не важно, какой именно. Также считается, что в рассматриваемой \textit{формальной теории} каждая \textit{подформула*} рассматриваемой \textbf{\textit{неатомарной логической формулы}} в рамках этой \textit{формальной теории} может однозначно трактоваться как либо истинная, либо ложная. В противном случае мы не можем говорить об истинности либо ложности исходной \textbf{\textit{неатомарной логической формулы}} в рамках этой \textit{формальной теории}.}\scnrelfrom{описание примера}{Примеры неатомарных логических формул}
\scnheader{подформула*}
\scnidtf{частная формула*}
\scniselement{бинарное отношение}
\scniselement{ориентированное отношение}
\scniselement{транзитивное отношение}
\scntext{definition}{Будем называть \textbf{\textit{подформулой*}} \textit{неатомарной логической формулы} \textbf{\textit{fi}} любую \textit{логическую формулу} \textbf{\textit{fj}}, являющуюся элементом исходной формулы \textbf{\textit{fi}}, а также любую \textbf{\textit{подформулу*}} формулы \textbf{\textit{fj}}.}\scnrelfrom{описание примера}{\scnfileimage[20em]{figures/sd_logical_formulas/subformula.png}
}
\scniselement{sc.g-текст}
\scnheader{утверждение}
\scnidtf{текст логической формулы}
\scntext{definition}{\textbf{\textit{утверждение}} -- это \textit{семантическая окрестность} некоторой \textit{логической формулы}, в которую входит полный текст этой \textit{логической формулы}, а также факт принадлежности этой \textit{логической формулы} некоторой \textit{формальной теории}.}\scntext{explanation}{Знак \textit{логической формулы}, семантическая окрестность которой представляет собой утверждение, является \textit{главным ключевым sc-элементом\scnrolesign} в рамках этого \textbf{\textit{утверждения}}. Знаки понятий соответствующей \textit{предметной области}, которые входят в состав какой-либо \textit{подформулы*} указанной \textit{логической формулы}, будут \textit{ключевыми sc-элементами\scnrolesign} в рамках этого \textbf{\textit{утверждения}}.Полный текст некоторой \textit{логической формулы} включает в себя:\begin{scnitemize}
\item знак самой этой \textit{логической формулы};\item знаки всех ее \textit{подформул*};\item элементы всех \textit{логических формул}, знаки которых попали в данную структуру;\item все пары принадлежности, связывающие \textit{логические формулы}, знаки которых попали в данную структуру, с их компонентами.\end{scnitemize}
Таким образом, факт принадлежности (истинности) логической формулы нескольким \textit{формальным теориям} будет порождать новое утверждение для каждой такой \textit{формальной теории}. Текст \textbf{\textit{утверждения}} входит в состав \textit{логической онтологии}, соответствующей \textit{предметной области}, на которой интерпретируется \textit{главный ключевой sc-элемент\scnrolesign} данного утверждения.}\scntext{правило идентификации экземпляров}{\textbf{\textit{утверждения}} в рамках \textit{Русского языка} именуются по следующим правилам:\begin{scnitemize}
\item в начале идентификатора пишется сокращение \textbf{Утв.};\item далее в круглых скобках через точку с запятой перечисляются основные идентификаторы \textit{ключевых \mbox{sc-элементов}\scnrolesign} данного \textbf{\textit{утверждения}}. Порядок определяется в каждом конкретном случае в зависимости от того, свойства каких из этих \textit{понятий} описывает данное \textbf{\textit{утверждение}} в большей или меньшей степени.\end{scnitemize}
}
\scntext{описание примера}{\textit{Утв. (параллельность*; секущая*)}}
\scntext{note}{Могут быть исключения для \textbf{\textit{утверждений}}, названия которых закрепились исторически, например, \textit{Теорема Пифагора}, \textit{Аксиома о прямой и точке}.}\scnrelfrom{описание примера}{\scnfileimage[20em]{figures/sd_logical_formulas/statement.png}
}
\scntext{note}{Утверждение показывает, что соответствующие углы при пересечении параллельных прямых секущей равны.}\scniselement{sc.g-текст}
\scnheader{определение}
\scnidtf{текст определения}
\scnsubset{утверждение}
\scntext{definition}{\textbf{\textit{определение}} -- это \textit{утверждение}, \textit{главным ключевым sc-элементом\scnrolesign} которого является связка \textit{эквиваленции*}, однозначно определяющая некоторое понятие на основе других понятий.}\scntext{note}{Каждое определение имеет ровно один \textit{ключевой sc-элемент\scnrolesign} (не считая \textit{главного ключевого sc-элемента\scnrolesign}).}\scntext{note}{Для одного и того же понятия в рамках одной \textit{формальной теории} может существовать несколько \textit{утверждений об эквиваленции*}, однозначно задающих некоторое понятие на основе других, однако только одно такое \textit{утверждение} в рамках этой \textit{формальной теории} может быть отмечено как \textbf{\textit{определение}}. Остальные \textit{утверждения об эквиваленции*} могут трактоваться как \textit{пояснения} данного понятия.}\scntext{правило идентификации экземпляров}{\textbf{\textit{определения}} в рамках \textit{Русского языка} именуются по следующим правилам:\begin{scnitemize}
\item в начале идентификатора пишется сокращение \textbf{Опр.};\item далее в круглых скобках через точку с запятой записывается основной идентификатор  \textit{ключевого sc-элемента\scnrolesign} данного \textbf{\textit{определения}}.\end{scnitemize}
}
\scntext{описание примера}{\textit{Опр. (ромб)}}
\scnrelfrom{описание примера}{\scnfileimage[20em]{figures/sd_logical_formulas/definition.png}
}
\scntext{note}{Определение показывает, что ромб  это четырёхугольник, у которого все стороны равны.}\scniselement{sc.g-текст}
\scnheader{общезначимая логическая формула}
\scnidtf{тождественно истинная логическая формула}
\scnsubset{выполнимая логическая формула}
\scnsubset{тавтология}
\scntext{definition}{\textbf{\textit{общезначимая логическая формула}} -- это \textit{логическая формула}, для которой не существует \textit{формальной теории}, в рамках которой она была бы ложной с учетом истинности и ложности всех ее \textit{подформул*} в рамках этой же \textit{формальной теории}.}\scnrelfrom{описание примера}{\scnfileimage[20em]{figures/sd_logical_formulas/valid_formula.png}
}
\scnrelfrom{основной sc-идентификатор}{\scnfilelong{закон тождества}
}
\scniselement{sc.g-текст}
\scnheader{противоречивая логическая формула}
\scnidtf{тождественно ложная логическая формула}
\scnsubset{невыполнимая логическая формула}
\scnsubset{тавтология}
\scntext{definition}{\textbf{\textit{противоречивая логическая формула}} -- это \textit{логическая формула}, для которой не существует \textit{формальной теории}, в рамках которой она была бы истинной с учетом истинности и ложности всех ее \textit{подформул*} в рамках этой же \textit{формальной теории}.}\scnrelfrom{описание примера}{\scnfileimage[20em]{figures/sd_logical_formulas/contradiction_formula.png}
}
\scnrelfrom{основной sc-идентификатор}{\scnfilelong{закон противоречия}
}
\scniselement{sc.g-текст}
\scnheader{нейтральная логическая формула}
\scnsubset{выполнимая логическая формула}
\scntext{definition}{\textbf{\textit{нейтральная логическая формула}} -- это \textit{логическая формула}, для которой существует хотя бы одна \textit{формальная теория}, в рамках которой эта формула ложна, и хотя бы одна \textit{формальная теория}, в рамках которой эта формула истинна.}\scnrelfrom{описание примера}{\scnfileimage[20em]{figures/sd_logical_formulas/neutral_formula.png}
}
\scntext{note}{В евклидовой геометрии в плоскости через точку, не лежащую на данной прямой, можно провести одну и только одну прямую, параллельную данной. В геометрии Лобачевского данный постулат является ложным.}\scntext{note}{В сферической геометрии все прямые пересекаются.}\scniselement{sc.g-текст}
\scnheader{непротиворечивая логическая формула}
\scnidtf{выполнимая логическая формула}
\scntext{definition}{\textbf{\textit{непротиворечивая логическая формула}} -- это \textit{логическая формула}, для которой существует хотя бы одна \textit{формальная теория}, в рамках которой эта формула истинна.}\begin{scnreltoset}{объединение}
\scnitem{нейтральная логическая формула}
\scnitem{общезначимая логическая формула}
\end{scnreltoset}
\scnheader{необщезначимая логическая формула}
\scnidtf{невыполнимая логическая формула}
\scntext{definition}{\textbf{\textit{необщезначимая логическая формула}} -- это \textit{логическая формула}, для которой существует хотя бы одна \textit{формальная теория}, в рамках которой эта формула ложна.}\begin{scnreltoset}{объединение}
\scnitem{нейтральная логическая формула}
\scnitem{противоречивая логическая формула}
\end{scnreltoset}
\scnheader{тавтология}
\scntext{definition}{\textbf{\textit{тавтология}} -- это \textit{логическая формула}, которая является либо только истинной, либо только ложной в рамках всех \textit{формальных теорий}, в которых можно установить ее истинность или ложность.}\scntext{explanation}{\textbf{\textit{тавтология}} -- это такая \textit{логическая формула}, которая является либо \textit{общезначимой логической формулой}, либо \textit{противоречивой логической формулой}.}\scnheader{логическая связка*}
\scnidtf{неатомарная логическая формула}
\scnidtf{логический оператор*}
\scnidtf{пропозициональная связка*}
\scniselement{класс связок разной мощности}
\scnrelto{семейство подмножеств}{неатомарное высказывание}
\scntext{definition}{\textbf{\textit{логическая связка*}} -- это отношение (класс связок), связками которого являются \textit{высказывания}.}\scntext{explanation}{\textbf{\textit{логическая связка*}} -- это \textit{отношение}, областью определения которого является множество \textit{высказываний}, при этом само это отношение и некоторые его подмножества могут быть \textit{классами связок разной мощности}.}\scnheader{конъюнкция*}
\scnidtf{логическое и*}
\scnidtf{логическое умножение*}
\scnsubset{логическая связка*}
\scniselement{неориентированное отношение}
\scniselement{класс связок разной мощности}
\scntext{definition}{\textbf{\textit{конъюнкция*}} -- это множество конъюнктивных \textit{высказываний}, каждое из которых истинно в рамках некоторой \textit{формальной теории} только в том случае, когда все его компоненты истинны в рамках этой же \textit{формальной теории}.}\scntext{note}{\textbf{\textit{конъюнкция*}} атомарных формул может быть заменена на атомарную формулу, полученную путём объединения исходных атомарных формул.}\scnrelfrom{описание примера}{\scnfileimage[20em]{figures/sd_logical_formulas/conjunction_triangles.png}
}
\scntext{explanation}{Данные конструкции эквивалентны по принципу $\exists x T(x) \land \exists x PT(x) \ \Longrightarrow \ \exists x (T(x) \land PT(x))$}\scntext{explanation}{Следует помнить про \textbf{\textit{Примечание про высказывание о существовании}}.}\scniselement{sc.g-текст}
\scnrelfrom{описание примера}{\scnfileimage[20em]{figures/sd_logical_formulas/conjunction.png}
}
\scniselement{sc.g-текст}
\scnheader{дизъюнкция*}
\scnidtf{логическое или*}
\scnidtf{логическое сложение*}
\scnidtf{включающее или*}
\scnsubset{логическая связка*}
\scniselement{неориентированное отношение}
\scniselement{класс связок разной мощности}
\scntext{definition}{\textbf{\textit{дизъюнкция*}} -- это множество дизъюнктивных \textit{высказываний}, каждое из которых истинно в рамках некоторой \textit{формальной теории} только в том случае, когда хотя бы один его компонент является истинным в рамках этой же \textit{формальной теории}.}\scnrelfrom{описание примера}{\scnfileimage[20em]{figures/sd_logical_formulas/disjunction.png}
}
\scniselement{sc.g-текст}
\scnheader{отрицание*}
\scnsubset{логическая связка*}
\scnsubset{синглетон}
\scntext{definition}{\textbf{\textit{отрицание*}} -- это множество \textit{высказываний} об отрицании, каждое из которых истинно в рамках некоторой \textit{формальной теории} только в том случае, когда его единственный элемент является ложным в рамках этой же \textit{формальной теории}.}\scnrelfrom{описание примера}{\scnfileimage[20em]{figures/sd_logical_formulas/negation.png}
}
\scniselement{sc.g-текст}
\scnheader{строгая дизъюнкция*}
\scnidtf{сложение по модулю 2*}
\scnidtf{исключающее или*}
\scnidtf{альтернатива*}
\scnsubset{логическая связка*}
\scniselement{неориентированное отношение}
\scniselement{класс связок разной мощности}
\scntext{definition}{\textbf{\textit{строгая дизъюнкция*}} -- это множество строго дизъюнктивных \textit{высказываний}, каждое из которых истинно в рамках некоторой \textit{формальной теории} только в том случае, когда ровно один его компонент является истинным в рамках этой же \textit{формальной теории}.}\scnrelfrom{описание примера}{\scnfileimage[20em]{figures/sd_logical_formulas/strictDisjunction.png}
}
\scniselement{sc.g-текст}
\scnrelfrom{описание примера}{\scnfileimage[20em]{figures/sd_logical_formulas/strict_disjunction_triangle.png}
}
\scntext{explanation}{Данная неатомарная логическая формула содержит следующую информацию: для любых переменных \_triangle если \_triangle является треугольником, то \_triangle является или тупоугольным треугольником, или остроугольным треугольником, или прямоугольным треугольником.}\scniselement{sc.g-текст}
\scntext{note}{\textbf{\textit{строгая дизъюнкция*}} может быть представлена как \textit{дизъюнкция} \textit{конъюнкции} \textit{отрицания} первой логической формулы и второй логической формулы и \textit{конъюнкции} первой логической формулы и \textit{отрицания} второй логической формулы. Также она может быть представлена и ввиде \textit{конъюнкции} \textit{дизъюнкций} двух логических формул и их \textit{отрицаний}.}\scnrelfrom{описание примера}{\scnfileimage[20em]{figures/sd_logical_formulas/strict_disjunction_representation.png}
}
\scniselement{sc.g-текст}
\scnheader{импликация*}
\scnidtf{логическое следование*}
\scnsubset{логическая связка*}
\scniselement{бинарное отношение}
\scniselement{ориентированное отношение}
\scntext{definition}{\textbf{\textit{импликация*}} -- это множество импликативных \textit{неатомарных высказываний}, каждое из которых состоит из посылки (первый компонент \textit{высказывания}) и следствия (второй компонент \textit{высказывания}).}\scntext{note}{Каждое импликативное \textit{высказывание} ложно в рамках некоторой \textit{формальной теории} в том случае, когда его посылка истинна, а заключение ложно в рамках этой же \textit{формальной теории}. В других случаях такое \textit{высказывание} истинно.}\scntext{note}{По умолчанию на все переменные, входящие в обе части высказывания об \textbf{\textit{импликации*}} (или хотя бы одну из \textit{подформул*} каждой части) неявно накладывается квантор \textit{всеобщности*}, при условии, что эти переменные не связаны другим \textit{квантором}, указанным явно.}\scnrelfrom{описание примера}{\scnfileimage[20em]{figures/sd_logical_formulas/implication.png}
}
\scniselement{sc.g-текст}
\scntext{note}{\textbf{\textit{импликация*}} может быть представлена как \textit{дизъюнкция} \textit{отрицания} первой логической формулы и второй логической формулы или же как \textit{отрицание} \textit{конъюнкции} первой логической формулы и \textit{отрицания} второй логической формулы.}\scnrelfrom{описание примера}{\scnfileimage[20em]{figures/sd_logical_formulas/implication_representation.png}
}
\scniselement{sc.g-текст}
\scnrelfrom{описание примера}{\scnfileimage[20em]{figures/sd_logical_formulas/implication_triangle.png}
}
\scntext{explanation}{Данная неатомарная логическая формула содержит следующую информацию: для любых переменных \_triangle и \_angle если \_triangle является прямоугольным треугольником, то синус его внутреннего угла \_angle равен единице.}\scniselement{sc.g-текст}
\scnheader{если\scnrolesign}
\scnidtf{посылка\scnrolesign}
\scnsubset{1\scnrolesign}
\scniselement{ролевое отношение}
\scntext{definition}{\textbf{\textit{если\scnrolesign}} -- это \textit{ролевое отношение}, используемое в связках \textit{импликации*} для указания посылки.}\scnheader{то\scnrolesign}
\scnidtf{следствие\scnrolesign}
\scnsubset{2\scnrolesign}
\scniselement{ролевое отношение}
\scntext{definition}{\textbf{\textit{то\scnrolesign}} -- это \textit{ролевое отношение}, используемое в связках \textit{импликации*} для указания следствия.}\scnheader{эквиваленция*}
\scnidtf{эквивалентность*}
\scnsubset{логическая связка*}
\scniselement{бинарное отношение}
\scniselement{неориентированное отношение}
\scntext{definition}{\textbf{\textit{эквиваленция*}} -- это множество \textit{высказываний} об эквивалентности, каждое из которых истинно в рамках некоторой \textit{формальной теории} только в тех случаях, когда оба его компонента одновременно либо истинны в рамках этой же \textit{формальной теории}, либо ложны.}\scntext{note}{По умолчанию на все переменные, входящие в обе части высказывания об \textbf{\textit{эквиваленции*}} (или хотя бы одну из \textit{подформул*} каждой части) неявно накладывается квантор \textit{всеобщности*}, при условии, что эти переменные не связаны другим \textit{квантором}, указанным явно.}\scnrelfrom{описание примера}{\scnfileimage[20em]{figures/sd_logical_formulas/equivalent.png}
}
\scniselement{sc.g-текст}
\scntext{note}{\textbf{\textit{эквиваленция*}} двух логических формул может быть представлена как \textit{дизъюнкция} \textit{конъюнкции} этих двух логическх формул и \textit{конъюнкции} \textit{отрицаний} этих двух логических формул.}\scnrelfrom{описание примера}{\scnfileimage[20em]{figures/sd_logical_formulas/equivalence_representation.png}
}
\scniselement{sc.g-текст}
\scnheader{квантор}
\scnsubset{логическая связка*}
\scntext{definition}{\textbf{\textit{квантор}}  это \textit{отношение}, каждая связка которой задает истинность множества \textit{логических формул}, входящих в ее состав, при выполнении дополнительных условий, связанных с некоторыми из переменных, входящих в состав этих \textit{логических формул}.}\scntext{note}{Будем говорить, что переменные связаны \textbf{\textit{квантором}} или попадают под область действия данного \textbf{\textit{квантора}} (имея в виду конкретную связку конкретного \textbf{\textit{квантора}}).}\scntext{note}{В состав каждой связки каждого \textbf{\textit{квантора}} входит \textit{атомарная формула}, являющаяся \textit{тривиальной структурой}, в которой перечислены переменные, связанные данным \textbf{\textit{квантором}}.}\scnheader{всеобщность*}
\scnidtf{квантор всеобщности*}
\scnidtf{квантор общности*}
\scniselement{квантор}
\scniselement{ориентированное отношение}
\scniselement{класс связок разной мощности}
\scntext{definition}{\textbf{\textit{всеобщность}} -- это \textit{квантор}, для каждой связки которого, истинной в рамках некоторой \textit{формальной теории}, выполняется следующее утверждение: все формулы, входящие в состав этой связки истинны в рамках этой же \textit{формальной теории} при всех (любых) возможных значениях всех элементов множества \textit{связываемых переменных\scnrolesign} входящего в эту связку.}\scntext{note}{Каждая связка \textit{квантора} \textbf{\textit{всеобщность*}} может быть представлена как \textit{конъюнкция*} (потенциально бесконечная) исходных \textit{логических формул}, входящих в состав этой связки, в каждой из которых все \textit{связанные переменные\scnrolesign} заменены на их возможные значения.}\scntext{note}{Квантор \textbf{\textit{всеобщности*}} зачастую обозначается $\forall$ \ и читается как для всех, для каждого, для любого или все, каждый, любой.}\scnrelfrom{описание примера}{\scnfileimage[20em]{figures/sd_logical_formulas/universality.png}
}
\scniselement{sc.g-текст}
\scnheader{формула существования}
\scnidtf{существование*}
\begin{scnsubdividing}
\scnitem{атомарная логическая формула}
\scnitem{неатомарное существование*}
\end{scnsubdividing}
\scnheader{неатомарное существование*}
\scnidtf{квантор неатомарного существования*}
\scniselement{квантор}
\scniselement{ориентированное отношение}
\scniselement{класс связок разной мощности}
\scntext{definition}{\textbf{\textit{неатомарное существование*}} -- это \textit{квантор}, для каждой связки которого, истинной в рамках некоторой \textit{формальной теории}, выполняется следующее утверждение: существуют значения всех элементов множества \textit{связываемых переменных\scnrolesign} входящего в эту связку, такие, что все формулы, входящие в состав этой связки истинны в рамках этой же \textit{формальной теории}.}\scntext{note}{Каждая связка \textit{квантора} \textbf{\textit{неатомарное существование*}} может быть представлена как \textit{дизъюнкция*} (потенциально бесконечная) исходных \textit{логических формул}, входящих в состав этой связки, в каждой из которых все \textit{связанные переменные\scnrolesign} заменены на их возможные значения.}\scntext{note}{квантор \textbf{\textit{существования*}} зачастую обозначается $\exists$ \ и читается как существует, для некоторого, найдется.}\scnrelfrom{описание примера}{\scnfileimage[20em]{figures/sd_logical_formulas/non_atomicExistence.png}
}
\scniselement{sc.g-текст}
\scnheader{число значений переменной}
\scniselement{параметр}
\scntext{explanation}{Каждый элемент \textit{параметра} \textbf{\textit{число значений переменной}} представляет собой класс ориентированных пар, первым компонентом которых является знак \textit{логической формулы}, вторым -- \textit{sc-переменная}, имеющая в рамках данной \textit{логической формулы} ограниченное известное число значений, при которых данная формула является истинной в рамках соответствующей \textit{формальной теории}.}\scntext{note}{Отметим, что в случае \textit{атомарной логической формулы} каждая такая связка связывает знак формулы и знак принадлежащей ей \textit{sc-переменной}, т.е. является, по сути, частным случаем пары принадлежности. В случае \textit{неатомарной логической формулы} указанная \textit{sc-переменная} может принадлежать любой из \textit{подформул*} исходной формулы.}\scntext{note}{\textit{измерением*} значения параметра \textbf{\textit{число значений переменной}} является некоторое \textit{число}, задающее количество значений \textit{sc-переменных} в рамках \textit{логической формулы}.}\scnheader{кратность существования}
\scniselement{параметр}
\scnrelfrom{область определения параметра}{формула существования}
\scnhaselement{единственное существование}
\scntext{explanation}{Каждый элемент \textit{параметра} \textbf{\textit{кратность существования}} представляет собой класс логических \textit{формул существования}, для которых  при интерпретации на соответствующей \textit{предметной области} существует ограниченное общее для всех таких формул число комбинаций значений переменных, при которых указанные формулы являются истинными в рамках соответствующей \textit{формальной теории}.}\scntext{note}{\textit{измерением*} каждого значения \textbf{\textit{кратности существования}} является некоторое \textit{число}, задающее количество таких комбинаций.}\scnheader{единственное существование}
\scnidtf{однократное существование}
\scnidtf{формула существования и единственности}
\scntext{note}{\textbf{\textit{единственное существование}} зачастую обозначается $\exists!$ \ и читается как существует и единственный.}\scnheader{логическая формула и единственность}
\scnsubset{логическая формула}
\scnsubset{единственное существование}
\scntext{explanation}{Каждый элемент множества \textbf{\textit{логическая формула и единственность}} представляет собой \textit{логическую формулу} (\textit{атомарную} или \textit{неатомарную}), для которой дополнительно уточняется, что при ее интерпретации на некоторой предметной области существует только один набор значений переменных, входящих в эту формулу (или ее \textit{подформулы*}), при котором указанная логическая формула истинна в рамках \textit{формальной теории}, в которую входит данная \textit{предметная область}.}\scnrelfrom{описание примера}{\scnfileimage[20em]{figures/sd_logical_formulas/unique_existance.png}
}
\scntext{note}{Данная формула показывает, что в рамках формальной теории геометрии Евклида существует только один прямоугольный треугольник с некоторым периметром, являющийся равнобедренным.}\scniselement{sc.g-текст}
\scnheader{связываемые переменные\scnrolesign}
\scniselement{ролевое отношение}
\scntext{definition}{\textbf{\textit{связываемые переменные\scnrolesign}} -- это \textit{ролевое отношение}, которое связывает связку конкретного \textit{квантора} с множеством переменных, которые связаны этим квантором.}\scnheader{открытая логическая формула}
\scntext{definition}{\textbf{\textit{открытая логическая формула}} -- это \textit{логическая формула}, в рамках которой (и всех ее \textit{подформул*}) существует хотя бы одна переменная, не связанная никаким \textit{квантором}.}\scnheader{замкнутая логическая формула}
\scntext{definition}{\textbf{\textit{замкнутая логическая формула}} -- это \textit{логическая формула}, в рамках которой (и всех ее \textit{подформул*}) не существует переменных, не связанных каким-либо \textit{квантором}.}\scnheader{Примеры неатомарных логических формул}
\begin{scneqtoset}
\scnitem{\scnfileimage[20em]{figures/sd_logical_formulas/example_line_segment_sum.png}
\\\scnrelfrom{описание примера}{\scnfileimage[20em]{figures/sd_logical_formulas/example_line_segment_sum_note.png}
}
\scntext{explanation}{AB+BC=AC}\scniselement{sc.g-текст}
}
\scnitem{\scnfileimage[20em]{figures/sd_logical_formulas/example_line_segment_diff.png}
\\\scnrelfrom{описание примера}{\scnfileimage[20em]{figures/sd_logical_formulas/example_line_segment_diff_note.png}
}
\scntext{explanation}{AB-AC=CB}\scniselement{sc.g-текст}
}
\end{scneqtoset}
\bigskip
\end{scnsubstruct}
\end{SCn}


\scsubsection[
    \protect\scneditors{Садовский М.Е.;Никифоров С.А.}
    \protect\scnmonographychapter{Глава 2.6. Языковые средства формального описания синтаксиса и денотационной семантики различных языков в интеллектуальных компьютерных системах нового поколения}
    \protect\scnmonographychapter{Глава 4.1. Структура интерфейсов интеллектуальных компьютерных систем нового поколения}]
    {Предметная область и онтология файлов, внешних информационных конструкций и внешних языков ostis-систем}
\label{sd_files}
\begin{SCn}
\scnsectionheader{\currentname}
\begin{scnsubstruct}
\scnheader{Предметная область файлов, внешних информационных конструкций и внешних языков ostis-систем}
\scniselement{предметная область}
\begin{scnhaselementrole}{класс объектов исследования}
файл\end{scnhaselementrole}
\begin{scnhaselementrolelist}{класс объектов исследования}
внешний язык;естественный язык;Русский язык;Английский язык;изображение;класс синтаксически эквивалентных информационных конструкций;максимальный класс синтаксически эквивалентных информационных конструкций
\end{scnhaselementrolelist}
\begin{scnhaselementrolelist}{исследуемое отношение}
трансляция sc-текста*
\end{scnhaselementrolelist}
\scnheader{файл}
\scnidtf{знак файла}
\scnidtf{sc-знак файла}
\scnidtf{знак информационной конструкции, внешней по отношению к sc-памяти}
\scnidtf{sc-ссылка}
\scntext{explanation}{Под \textbf{\textit{файлом}} понимается любая информационная конструкция, внешняя по отношению к \textit{sc-памяти}, т.е., не являющаяся \textit{sc-текстом}. При этом каждому \textbf{\textit{файлу}} может быть поставлен в соответствие семантически эквивалентный \textit{sc-текст}.}\scnheader{внешний язык}
\scnidtf{язык, внешний по отношению к sc-памяти}
\scnrelto{семейство подмножеств}{файл}
\scntext{explanation}{Под \textbf{\textit{внешним языком}} понимается множество \textit{файлов}, имеющих общую синтаксическую структуру.}\scnheader{естественный язык}
\scnidtf{язык диалога с пользователем}
\scnsubset{внешний язык}
\scnhaselement{Русский язык}
\scnhaselement{Английский язык}
\scntext{explanation}{Под конкретным \textbf{\textit{естественным языком}} понимается некоторое множество \textit{файлов} (например, идентификаторов, естественно-языковых пояснений и т.д.), которые используются при диалоге с тем или иным пользователем, режим ведения которого он может выбрать.В этом смысле некоторые фрагменты, такие как, например обозначения \textbf{sin}, \textbf{cos}, \textbf{a.e.} и т.п., могут входить в несколько \textbf{\textit{естественных языков}}, поскольку используются при диалоге, но исторически являться фрагментами другого языка.}\scnheader{Русский язык}
\scnheader{Английский язык}
\scnheader{трансляция sc-текста*}
\scntext{explanation}{Связки отношения \textbf{\textit{трансляция sc-текста*}} связывают некоторый \textit{sc-текст} и \textit{файл}, который является семантическим эквивалентом этого \textit{sc-текста} на некотором внешнем языке (в том числе, например, языке геометрических чертежей, математических формул и т.д.).}\scnheader{изображение}
\scnidtf{графический файл}
\scnidtf{графическая несимвольная информационная конструкция}
\scnsubset{файл}
\scnheader{класс синтаксически эквивалентных информационных конструкций}
\scnsuperset{максимальный класс синтаксически эквивалентных информационных конструкций}
\scntext{explanation}{\textbf{\textit{класс синтаксически эквивалентных информационных конструкций}} - \textit{класс} информационных конструкций имеющих общие синтаксические свойства без учета различий в форматах их кодирования, в используемых шрифтах, в форматировании и размещении.}\scnheader{максимальный класс синтаксически эквивалентных информационных конструкций}
\scntext{explanation}{\textbf{\textit{максимальный класс синтаксически эквивалентных информационных конструкций}} - \textit{класс} всевозможных конструкций, имеющих общие синтаксические свойства без учета различий в форматах их кодирования, в используемых шрифтах, в форматировании и размещении.}
\end{scnsubstruct}
\end{SCn}

\scsubsubsection[
    \protect\scneditors{Никифоров С.А.;Бобёр  Е.С.}
    \protect\scnmonographychapter{Глава 2.6. Языковые средства формального описания синтаксиса и денотационной семантики различных языков в интеллектуальных компьютерных системах нового поколения}
    ]{Предметная область и онтология естественных языков}
\label{sd_natural_languages}
\begin{SCn}
\scnsectionheader{\currentname}
\begin{scnsubstruct}
\begin{scnrelfromlist}{соавтор}
\scnitem{Гордей А.Н.}
\scnitem{Никифоров С.А.}
\scnitem{Бобёр Е.С.}
\scnitem{Святощик М.И.}
\end{scnrelfromlist}
\scnheader{Предметная область естественных языков}
\scniselement{предметная область}
\begin{scnhaselementrole}{класс объектов исследования}
язык\end{scnhaselementrole}
\begin{scnhaselementrolelist}{класс объектов исследования}
плановый язык;язык общения;лексема;номинативная единица;комбинаторный вариант лексемы;естественный язык;тайген;ёген
\end{scnhaselementrolelist}
\begin{scnhaselementrolelist}{исследуемое отношение}
морфологическая парадигма*;член предложения\scnrolesign
\end{scnhaselementrolelist}
\scnheader{язык}
\begin{scnsubdividing}
\scnitem{естественный язык\\\scntext{explanation}{Естественный язык представляет собой язык, который не был создан целенаправленно}}
\scnitem{искусственный язык\\\scntext{explanation}{Искусственный язык представляет собой язык, специально разработанный для достижения определённых целей}\scnhaselement{Эсперанто}
\scnhaselement{Python}
\scnsuperset{сконструированный язык}
\scntext{explanation}{Сконструированный язык представляет собой искусственный язык, предназначенный для общения людей}\scnhaselement{Эсперанто}
}
\end{scnsubdividing}
\scnsuperset{международный язык}
\scntext{explanation}{Международный язык представляет собой естественный или искусственный язык, использующийся для общения людей разных из стран}\scnhaselement{Английский язык}
\scnhaselement{Русский язык}
\scnheader{плановый язык}
\begin{scnreltoset}{пересечение}
\scnitem{сконструированный язык}
\scnitem{международный язык}
\end{scnreltoset}
\scnheader{язык общения}
\begin{scnreltoset}{объединение}
\scnitem{естественный язык}
\scnitem{сконструированный язык}
\end{scnreltoset}
\scnhaselement{Английский язык}
\scnhaselement{Русский язык}
\scnhaselement{Эсперанто}
\begin{scnreltoset}{объединение}
\scnitem{корневой язык\\\scntext{explanation}{Корневой язык представляет собой язык, для которого характерно полное отсутствие словоизменения и наличие грамматической значимости порядка слов, состоящих только из корня.}\scnhaselement{Английский язык}
}
\scnitem{агглютинативный язык\\\scntext{explanation}{Агглютинативный язык характеризуется развитой системой употребления суффиксов, приставок, добавляемых к неизменяемой основе слова, которые используются для выражения категорий числа, падежа, рода и др.}\scnhaselement{Английский язык}
}
\scnitem{флективный язык\\\scntext{explanation}{Для флективного языка характерно развитое употребление окончаний для выражения категорий рода, числа, падежа, сложная система склонения глаголов, чередование гласных в корне, а также строгое различение частей речи.}\scnhaselement{Русский язык}
}
\scnitem{профлективный язык\\\scntext{explanation}{Для профлективного языка характерны агглютинация (в случае именного словоизменения), флексия и чередование гласных (аблаут)(в случае глагольного словоизменения).}}
\end{scnreltoset}
\scnheader{лексема}
\scnsubset{файл}
\scntext{explanation}{\textit{Лексема} -- тайген или ёген конкретного естественного языка.}\scnrelfrom{источник}{\cite{Hardzei2005}}
\scnheader{номинативная единица}
\scnsubset{файл}
\scntext{explanation}{\textit{Номинативная единица} -- устойчивая последовательность комбинторных вариантов лексем, в которой один вариант лексемы (модификатор) определяет другой (актуализатор), например: записная книжка, бежать галопом.}\scnrelfrom{источник}{\cite{Hardzei2005}}
\scnheader{комбинаторный вариант лексемы}
\scnsubset{файл}
\scntext{explanation}{\textit{Комбинторный вариант лексемы } -- вариант лексемы в упорядоченном наборе её вариантов (парадигме).}\scnrelfrom{источник}{\cite{Hardzei2007}}
\scnheader{морфологическая парадигма*}
\scniselement{квазибинарное отношение}
\scntext{explanation}{\textit{Морфологическая парадигма*} -- квазибинарное отношение, связывающее лексему с её комбинторными вариантами.}\scnrelfrom{первый домен}{словоформа}
\scnrelfrom{второй домен}{лексема}
\scnheader{естественный язык}
\begin{scnsubdividing}
\scnitem{часть языка\\\begin{scnsubdividing}
\scnitem{тайген}
\scnitem{ёген}
\end{scnsubdividing}
}
\scnitem{знак алфавита синтаксиса\\\scntext{explanation}{\textit{Знаки алфавита синтаксиса} -- вспомогательные средства синтаксиса (на макроуровне -- предлоги, послелоги, союзы, частицы и др., на микроуровне -- флексии, префиксы, постфиксы, инфиксы и др.), служащие для соединения составных частей языковых структур и образования морфологических парадигм.}\scnrelfrom{источник}{\cite{Hardzei2005}}
}
\end{scnsubdividing}
\scnheader{тайген}
\scntext{explanation}{\textit{Тайген} -- часть языка, обозначающая индивида.}\scnrelfrom{источник}{\cite{Hardzei2006}}
\scnrelfrom{источник}{\cite{Hardzei2015}}
\begin{scnsubdividing}
\scnitem{развёрнутый тайген\\\begin{scnsubdividing}
\scnitem{составной тайген\\}
\scnitem{сложный тайген\\}
\end{scnsubdividing}
}
\scnitem{свёрнутый тайген\\\begin{scnsubdividing}
\scnitem{сокращённый тайген\\}
\scnitem{сжатый тайген\\\begin{scnsubdividing}
\scnitem{информационный тайген\\\scntext{explanation}{\textit{Информационный тайген} -- тайген, обозначающий индивида в информационном фрагменте модели мира.}\scnrelfrom{источник}{\cite{Hardzei2006}}
\scnrelfrom{источник}{\cite{Hardzei2015}}
}
\scnitem{физический тайген\\\scntext{explanation}{\textit{Физический тайген} -- тайген, обозначающий индивида в физическом фрагменте модели мира. }\scnrelfrom{источник}{\cite{Hardzei2006}}
\scnrelfrom{источник}{\cite{Hardzei2015}}
\begin{scnsubdividing}
\scnitem{постоянный тайген\\\scntext{explanation}{\textit{Постоянный тайген} -- физический тайген, обозначающий постоянного индивида.}\scnrelfrom{источник}{\cite{Hardzei2006}}
\scnrelfrom{источник}{\cite{Hardzei2015}}
}
\scnitem{переменный тайген\\\scntext{explanation}{\textit{Переменный тайген} -- физический тайген, обозначающий переменного индивида.}\scnrelfrom{источник}{\cite{Hardzei2006}}
\scnrelfrom{источник}{\cite{Hardzei2015}}
}
\end{scnsubdividing}
\begin{scnsubdividing}
\scnitem{качественный тайген}
\scnitem{количественный тайген\\}
\end{scnsubdividing}
\begin{scnsubdividing}
\scnitem{одноместный тайген}
\scnitem{многоместный тайген\\\scnsuperset{интенсивный тайген}
\scnsuperset{экстенсивный тайген}
}
\end{scnsubdividing}
}
\end{scnsubdividing}
}
\end{scnsubdividing}
}
\end{scnsubdividing}
\scnheader{ёген}
\scntext{explanation}{\textit{Ёген} -- часть языка, обозначающая признак индивида.}\scnrelfrom{источник}{\cite{Hardzei2006}}
\scnrelfrom{источник}{\cite{Hardzei2015}}
\begin{scnsubdividing}
\scnitem{развёрнутый ёген\\\begin{scnsubdividing}
\scnitem{составной ёген}
\scnitem{сложный ёген}
\end{scnsubdividing}
}
\scnitem{свёрнутый ёген\\\begin{scnsubdividing}
\scnitem{сокращённый ёген\\\begin{scnsubdividing}
\scnitem{информационный ёген\\\scntext{explanation}{\textit{Информационный ёген} -- еген, обозначающий признак индивида в информационном фрагменте модели мира.}\scnrelfrom{источник}{\cite{Hardzei2006}}
\scnrelfrom{источник}{\cite{Hardzei2007a}}
}
\scnitem{физический ёген\\\scntext{explanation}{\textit{Физический ёген} -- еген, обозначающий признак индивида в физическом фрагменте модели мира.}\scnrelfrom{источник}{\cite{Hardzei2006}}
\scnrelfrom{источник}{\cite{Hardzei2007a}}
\begin{scnsubdividing}
\scnitem{постоянный ёген\\\scntext{explanation}{\textit{Постоянный ёген} - физический ёген, обозначающий постоянный признак индивида.}\scnrelfrom{источник}{\cite{Hardzei2006}}
\scnrelfrom{источник}{\cite{Hardzei2007a}}
}
\scnitem{переменный ёген\\\scntext{explanation}{\textit{Переменный ёген} -- физический ёген, обозначающий переменный признак индивида.}\scnrelfrom{источник}{\cite{Hardzei2006}}
\scnrelfrom{источник}{\cite{Hardzei2007a}}
}
\end{scnsubdividing}
\begin{scnsubdividing}
\scnitem{качественный ёген}
\scnitem{количественный ёген}
\end{scnsubdividing}
\begin{scnsubdividing}
\scnitem{одноместный ёген}
\scnitem{многоместный ёген\\\begin{scnsubdividing}
\scnitem{интенсивный ёген}
\scnitem{экстенсивный ёген}
\end{scnsubdividing}
}
\end{scnsubdividing}
}
\end{scnsubdividing}
}
\scnitem{сжатый ёген}
\end{scnsubdividing}
}
\end{scnsubdividing}
\scnheader{член предложения\scnrolesign}
\scniselement{ролевое отношение}
\scntext{explanation}{\textit{Член предложения\scnrolesign} -- это отношение, связывающее декомпозицию текста с файлом, содержимое которого (часть языка) играет в декомпозируемом тексте определенную синтаксическую роль.}\scnrelfrom{источник}{\cite{Hardzei2005}}
\begin{scnsubdividing}
\scnitem{главный член предложения\scnrolesign\\\begin{scnsubdividing}
\scnitem{подлежащее\scnrolesign\\\scntext{explanation}{\textit{Подлежащее\scnrolesign} -- это одно из главных ролевых отношений, связывающее декомпозицию текста с файлом, содержимое которого обозначает исходный пункт описания события, выбранный наблюдателем. }\scnrelfrom{источник}{\cite{Hardzei2020}}
}
\scnitem{сказуемое\scnrolesign\\\scntext{explanation}{\textit{Сказуемое\scnrolesign} -- это одно из главных ролевых отношений, связывающее декомпозицию текста с файлом, содержимое которого обозначает отображение наблюдателем исходного пункта описания события в конечный.}\scnrelfrom{источник}{\cite{Hardzei2020}}
}
\scnitem{прямое дополнение\scnrolesign\\\scntext{explanation}{\textit{Прямое дополнение\scnrolesign} -- это одно из главных ролевых отношений, связывающее декомпозицию текста с файлом, содержимое которого обозначает конечный пункт описания события, выбранный наблюдателем.}\scnrelfrom{источник}{\cite{Hardzei2020}}
}
\end{scnsubdividing}
}
\scnitem{второстепенный член предложения\scnrolesign\\\begin{scnsubdividing}
\scnitem{косвенное дополнение\scnrolesign}
\scnitem{определение\scnrolesign\\\scntext{explanation}{\textit{Определение\scnrolesign} -- это одно из второстепенных ролевых отношений, связывающее декомпозицию текста с файлом, содержимое которого обозначает модификацию подлежащего, дополнения, обстоятельства места и времени.}\scnrelfrom{источник}{\cite{Hardzei2007a}}
\scnrelfrom{источник}{\cite{Hardzei2017a}}
\scnrelfrom{источник}{\cite{Hardzei2007b}}
}
\scnitem{обстоятельство\scnrolesign\\\scntext{explanation}{\textit{Обстоятельство\scnrolesign} -- это одно из второстепенных ролевых отношений, связывающее декомпозицию текста с файлом, содержимое которого обозначает либо модификацию, либо локализацию сказуемого.}\scnrelfrom{источник}{\cite{Hardzei2007a}}
\scnrelfrom{источник}{\cite{Hardzei2017a}}
\scnrelfrom{источник}{\cite{Hardzei2007b}}
\begin{scnsubdividing}
\scnitem{обстоятельство степени\scnrolesign\\\scntext{explanation}{Обстоятельство степени -- обстоятельство, обозначающее модификацию сказуемого.}}
\scnitem{обстоятельство образа действия\scnrolesign\\\scntext{explanation}{Обстоятельство образа действия -- обстоятельство, обозначающее модификацию сказуемого.}}
\scnitem{обстоятельство места\scnrolesign\\\scntext{explanation}{Обстоятельства места -- обстоятельство, обозначающее пространственную локализацию сказуемого.}\begin{scnsubdividing}
\scnitem{динамическое обстоятельство места\scnrolesign}
\scnitem{статическое обстоятельство места\scnrolesign}
\end{scnsubdividing}
}
\scnitem{обстоятельство времени\scnrolesign\\\scntext{explanation}{Обстоятельство времени -- обстоятельство, обозначающее временную локализацию сказуемого.}\begin{scnsubdividing}
\scnitem{динамическое обстоятельство времени\scnrolesign}
\scnitem{статическое обстоятельство времени\scnrolesign}
\end{scnsubdividing}
}
\end{scnsubdividing}
}
\end{scnsubdividing}
}
\end{scnsubdividing}
\bigskip\bigskip\scnheader{Пример sc.g-текста, описывающего лексему}
\scniselement{sc.g-текст}
\scntext{explanation}{Здесь представлено описание лексемы с указанием ее принадлежности определённой части речи. Также описание содержит морфологическую парадигму данной лексемы, связывающую ее с ее словоформами.}\scneq{\scnfileimage[20em]{figures/sd_natural_languages/lexeme_example.png}}
\scnheader{Пример этапов разбора текста естественного языка}
\begin{scnsubstruct}
\scnfileimage[20em]{figures/sd_natural_languages/nl_text.png}
\scntext{explanation}{с точки зрения ostis-системы, любой естественно-языкой текст является \textit{файлом.}}\scnrelfrom{лексическая структура}{\scnfileimage[20em]{figures/sd_natural_languages/nl_lexical.png}
}
\scntext{explanation}{Данная конструкция описывает декомпозицию исходного текста на фрагменты с указанием их принадлежности определённой \textit{номинативной единице} или \textit{знаку алфавита синтаксиса}.}\scnrelfrom{синтаксическая структура}{\scnfileimage[20em]{figures/sd_natural_languages/nl_synactical.png}
}
\bigskip\bigskip\scntext{explanation}{Здесь приведена только частью синтаксической структуры. Оставшаяся часть записывается аналогично.}
\end{scnsubstruct}
\end{scnsubstruct}
\end{SCn}

\scparagraph[
    \protect\scneditors{Никифоров С.А.;Бобёр  Е.С.}
    \protect\scnmonographychapter{Глава 2.6. Языковые средства формального описания синтаксиса и денотационной семантики различных языков в интеллектуальных компьютерных системах нового поколения}
    ]{Предметная область и онтология синтаксиса естественных языков}
\label{sd_syntax_natural_lang}

\scparagraph[
    \protect\scneditors{Никифоров С.А.;Бобёр  Е.С.}
    \protect\scnmonographychapter{Глава 2.6. Языковые средства формального описания синтаксиса и денотационной семантики различных языков в интеллектуальных компьютерных системах нового поколения}
    ]{Предметная область и онтология денотационной семантики естественных языков}
\label{sd_sem_natural_lang}

\scsubsection[
    \protect\scneditors{Никифоров С.А.;Шункевич Д.В.}
    \protect\scnmonographychapter{Глава 3.1. Формализация понятий действия, задачи, метода, средства, навыка и технологии}
    ]{Глобальная предметная область и онтология, описывающая воздействия, действия, методы, средства и технологии}
\label{sd_actions}
\begin{SCn}
\scnsectionheader{\currentname}
\begin{scnsubstruct}
\begin{SCn}
\scniselement{раздел}
\scniselement{предметная область и онтология}
\begin{scnreltovector}{конкатенация сегментов}
\scnitem{Уточнение понятия воздействия и понятия действия. Типология воздействий и действий}
\scnitem{Уточнение понятия задачи. Типология задач}
\scnitem{Уточнение семейства параметров и отношений, заданных на множестве воздействий, действий и задач}
\scnitem{Предметная область и онтология субъектно-объектных спецификаций воздействий}
\scnitem{Уточнение понятий плана сложного действия, класса задач, метода}
\scnitem{ Уточнение понятия навыка, понятия класса методов и понятия модели решения задач}
\scnitem{Уточнение понятия деятельности, понятия вида деятельности и понятия технологии}
\end{scnreltovector}
\begin{scnhaselementrolelist}
\scnitem{исследуемый класс первичных объектов исследования}
\end{scnhaselementrolelist}
\bigskip\begin{scnhaselementrolelist}
\scnitem{исследуемый класс классов первичных объектов исследования}
\end{scnhaselementrolelist}
\begin{scnhaselementrolelist}
\scnitem{исследуемый класс классов}
\end{scnhaselementrolelist}
\scnsourcecommentpar{Здесь указаны классы классов, которые не являются классами классов \uline{первичных} объектов исследования}\bigskip\begin{scnhaselementrolelist}
\scnitem{исследуемое отношение, заданное на множестве первичных объектов исследования}
\end{scnhaselementrolelist}
\bigskip\begin{scnhaselementrolelist}
\scnitem{исследуемое отношение}
\end{scnhaselementrolelist}
\scnsourcecommentpar{Здесь указаны исследуемые отношения, которые заданы не на множестве первичных объектов исследования}\bigskip\begin{scnhaselementrolelist}
\scnitem{исследуемый класс структур, специфицирующих первичные объекты исследования}
\end{scnhaselementrolelist}
\bigskip\begin{scnhaselementrolelist}
\scnitem{исследуемый класс структур}
\end{scnhaselementrolelist}
\scnsourcecommentpar{Здесь указаны классы структур, не являющихся спецификациями первичных объектов исследования}\bigskip\begin{scnhaselementrolelist}
\scnitem{вводимое, но не исследуемое понятие}
\end{scnhaselementrolelist}
\scnsourcecommentpar{Здесь указаны понятия, исследуемые в предметной области (и соответствующей онтологии), которая является \uline{частной} по отношению к заданной и которая наследует все свойства заданной предметной области и онтологии}\bigskip\begin{scnhaselementrolelist}
\scnitem{используемое понятие, исследуемое в другой предметной области и онтологии}
\end{scnhaselementrolelist}
\scnheader{следует отличать*}
\begin{itemize}
    \item Класс действий
    \item Класс методов
    \item Вид деятельности
\end{itemize}
\scnsubset{семейство подклассов*}
\scntext{note}{Все сущности, принадлежащие рассмотренным \textit{понятиям}, требуют достаточно детальной \textit{спецификации}. При этом не следует путать сами сущности и их \textit{спецификации}. Так, например, не следует путать \textit{действие} и \textit{задачу}, которая специфицирует (уточняет) это \textit{действие}. Особое место среди указанных понятий занимает понятие \textit{метода}, т.к. каждый конкретный \textit{метод}, с одной стороны, является \textit{спецификацией} соответствующего \textit{класса действий}, а, с другой стороны, сам нуждается в \textit{спецификации}, которая уточняет либо \textit{декларативную семантику} этого \textit{метода} (т.е. обобщенную декларативную формулировку класса задач, решаемых с помощью этого \textit{метода}), либо \textit{операционную семантику} этого \textit{метода}, (т.е. множество \textit{методов}, обеспечивающих \textit{интерпретацию} данного специфицируемого \textit{метода}) и тем самым преобразует специфицируемый \textit{метод} в \textit{навык}.}\scnheader{следует отличать*}
\begin{scnhaselementvector}
\scnitem{первый домен*(спецификация*)\\\scnidtf{специфицируемая сущность}
\scnidtf{сущность, использование которой требует вполне определенной ее спецификации}
\scnsuperset{действие}
\scnsuperset{класс действий}
\scnsuperset{метод}
\scnsuperset{класс методов}
\scnsuperset{деятельность}
\scnsuperset{вид деятельности}
}
\scnitem{второй домен*(спецификация*)\\\scnidtf{спецификация}
\scnsuperset{задача}
\scnsuperset{декларативная формулировка задачи}
\scnidtf{семантическая формулировка задачи}
\scnsuperset{процедурная формулировка задачи}
\scnidtf{функциональная формулировка задачи}
\scnsuperset{план действия}
\scnidtf{план}
\scnidtf{план выполнения действия}
\scnsuperset{декларативная спецификация выполнения действий}
\scnidtf{иерархическая система подзадач}
\scnsuperset{протокол}
\scnsuperset{результативная часть протокола}
\scnsuperset{обобщенная декларативная формулировка класса задач}
\scnsuperset{метод}
\scnsuperset{декларативная семантика метода}
\scnsuperset{операционная семантика метода}
\scnsuperset{модель решения задач}
}
\end{scnhaselementvector}
\scntext{note}{При этом следует отличать:\begin{scnitemize}
\item спецификацию конкретного \textit{действия} (\textit{задачу}, \textit{план}, \textit{декларативную спецификацию выполнения действия}, \textit{протокол}, \textit{результативную часть протокола});\item спецификацию конкретной \textit{деятельности} (\textit{контекст}*, \textit{множество используемых методов}*);\item спецификацию \textit{класса действий} (\textit{обобщенную декларативную формулировку класса задач}, \textit{метод});\item спецификацию \textit{вида деятельности} (\textit{технологию});\item спецификацию \textit{метода} (\textit{декларативную семантику метода}, \textit{операционную семантику метода});\item спецификацию \textit{класса методов} (\textit{модель решения задач}).\end{scnitemize}
}\scnheader{следует отличать*}

\bigskip\begin{scnrelfromlist}{библиографический источник}
\scnitem{\cite{Martynov1984}\\\scnciteannotation{Martynov1984}}
\scnitem{\cite{Ikeda1998}\\\scnciteannotation{Ikeda1998}\scnrelfrom{ключевой знак}{онтология классов задач}
\scnidtf{задачная онтология}
\scnidtf{онтология классов задач, решаемых в данной предметной области}
}
\scnitem{\cite{Studer1996}\\\scnciteannotation{Studer1996}}
\scnitem{\cite{Benjamins1999}\\\scnciteannotation{Benjamins1999}}
\scnitem{\cite{Chandrasekaran1999}\\\scnciteannotation{Chandrasekaran1999}}
\scnitem{\cite{Chandrasekaran1998}\\\scnciteannotation{Chandrasekaran1998}}
\scnitem{\cite{Fensel1998Reuse}\\\scnciteannotation{Fensel1998Reuse}}
\scnitem{\cite{Kemke2001}\\\scnciteannotation{Kemke2001}}
\scnitem{\cite{Tu1995}\\\scnciteannotation{Tu1995}}
\scnitem{\cite{Trypuz2007}\\\scnciteannotation{Trypuz2007}}
\scnitem{\cite{Fang2019}\\\scnciteannotation{Fang2019}}
\scnitem{\cite{Fensel1997}}
\scnitem{\cite{McBride2021}\\\scnciteannotation{McBride2021}}
\scnitem{\cite{Crowther2020}\\\scnciteannotation{Crowther2020}}
\scnitem{\cite{McCann1998}\\\scnciteannotation{McCann1998}}
\scnitem{\cite{Yan2014}\\\scnciteannotation{Yan2014}}
\scnitem{\cite{Ansari2018}}
\scnitem{\cite{Crubezy2004}\\\scnciteannotation{Crubezy2004}}
\scnitem{\cite{Coelho1996}}
\end{scnrelfromlist}
\end{SCn}
\begin{SCn}
\scnsegmentheader{Уточнение понятия воздействия и понятия действия. Типология воздействий и действий}
\begin{scnsubstruct}
\scniselement{сегмент базы знаний}
\scnheader{воздействие}
\scnidtf{\textit{процесс} воздействия одной сущности (или некоторого множества \textit{сущностей}) на другую \textit{сущность} (или на некоторое множество других \textit{сущностей})}
\scnidtf{\textit{процесс}, в котором могут быть явно выделены хотя бы одна воздействующая сущность (\textit{субъект воздействия\scnrolesign}) и хотя бы одна \textit{сущность}, на которую осуществляется воздействие (\textit{субъект воздействия\scnrolesign})}
\scnheader{воздействие}
\scnsubset{процесс}
\scnidtf{динамическая структура}
\scntext{note}{Поскольку \textit{воздействия} являются частным видом \textit{процессов}, воздействиями наследуются все свойства \textit{процессов}. Смотрите Раздел \textit{Предметная область и онтология структур}). В частности, используются все \textit{параметры}, заданные на множестве \textit{процессов}, например, \textit{длительность*}, \textit{момент начала процесса*}, \textit{момент завершения процесса\scnsupergroupsign}}\scnheader{процесс}
\scnrelfrom{покрытие}{длительность\scnsupergroupsign\\\begin{scneqtoset}
\scnitem{краткосрочный процесс}
\scnitem{среднесрочный процесс}
\scnitem{долгосрочный процесс}
\scnitem{ перманентный процесс}
\end{scneqtoset}
\scntext{note}{Длительность различных процессов можно уточнять до любой необходимой точности, используя различные единицы измерения длительности (с точностью до секунд, минут, часов, дней, месяцев, лет, столетий и т.д.). Кроме того, можно ссылаться на процессы, длительность (время жизни) которых соизмерима с рассматриваемым процессом.}}
\scnheader{действие}
\scnsubset{воздействие}
\scnsubset{процесс}
\scnidtf{\textit{воздействие}, в котором \textit{воздействующая сущность\scnrolesign} осуществляет \textit{воздействие} осознанно}
\end{scnsubstruct}
\end{SCn}\begin{SCn}
\scnsegmentheader{Уточнение понятия задачи. Типология задач}
\begin{scnsubstruct}
\scniselement{сегмент базы знаний}
\scnheader{задача}
\scnidtf{описание желаемого состояния или события в рамках внешней среды кибернетической системы либо в рамках её базы знаний}
\scnidtf{формулировка задачи}
\scnidtf{задание на выполнение некоторого действия}
\scnidtf{постановка задачи}
\scnidtf{описание задачной ситуации}
\scnidtf{спецификация некоторого действия, обладающая достаточной полнотой для выполнения этого действия}
\scnidtf{цель плюс дополнительные условия (ограничения) накладываемые на результат или процесс получения этого результата}
\scnidtf{описание того, что требуется сделать}
\scntext{explanation}{\textbf{\textit{Задача}}, т.е. формальное описание условия некоторой задачи есть, по сути, формальная \textit{спецификация} некоторого \textit{действия}, направленного на решение данной \textit{задачи}, достаточная для выполнения данного \textit{действия} каким-либо \textit{субъектом}. В зависимости от конкретного \textit{класса задач}, описываться может как внутреннее состояние самой интеллектуальной системы, так и требуемое состояние \textit{внешней среды}. \textit{sc-элемент}, обозначающий \textit{действие} входит в \textit{задачу} под атрибутом \textit{ключевой знак\scnrolesign}.Каждая \textbf{\textit{задача}} представляет собой спецификацию \textit{действия}, которое либо уже выполнено, либо выполняется в текущий момент (в настоящее время), либо планируется (должно) быть выполненным, либо может быть выполнено (но не обязательно).Классификация \textit{задач} может осуществляться по дидактическому признаку в рамках каждой предметной области, например, задачи на треугольники, задачи на системы уравнений и т.п.Каждая \textit{задача} может включать:\begin{scnitemize}
\item факт принадлежности \textit{действия} какому-либо частному классу \textit{действий} (например,\textit{ действие. сформировать полную семантическую окрестность указываемой сущности}), в том числе состояние \textit{действия} с точки зрения жизненного цикла (инициированное, выполняемое и т.д.);\item описание \textit{цели*} (\textit{результата*}) \textit{действия}, если она точно известна;\item указание \textit{заказчика*} действия;\item указание \textit{исполнителя* действия} (в том числе, коллективного);\item указание \textit{аргумента(ов) действия\scnrolesign};\item указание инструмента или посредника \textit{действия};\item описание \textit{декомпозиции действия*};\item указание \textit{последовательности действий*} в рамках \textit{декомпозиции действия*}, т.е построение плана решения задачи. Другими словами, построение плана решения представляет собой декомпозицию соответствующего \textit{действия} на систему взаимосвязанных между собой поддействий;\item указание области \textit{действия};\item указание условия инициирования \textit{действия};\item момент начала и завершения \textit{действия}, в том числе планируемый и фактический, предполагаемая и/или фактическая длительность выполнения;\end{scnitemize}
Некоторые \textit{задачи} могут быть дополнительно уточнены контекстом -- дополнительной информацией о сущностях, рассматриваемых в формулировке \textit{задачи}, т.е. описанием того, что дано, что известно об указанных сущностях.Кроме этого, \textit{задача} может включать любую дополнительную информацию о действии, например:\begin{scnitemize}
\item перечень ресурсов и средств, которые предполагается использовать при решении задачи, например список доступных исполнителей, временные сроки, объем имеющихся финансов и т.д.;\item ограничение области, в которой выполняется \textit{действие}, например, необходимо заменить одну \textit{\mbox{sc-конструкцию}} на другую по некоторому правилу, но только в пределах некоторого \textit{раздела базы знаний};\item ограничение знаний, которые можно использовать для решения той или иной задачи, например, необходимо решить задачу по алгебре используя только те утверждения, которые входят в курс школьной программы до седьмого класса включительно, и не используя утверждения, изучаемые в старших классах;\item и прочее\end{scnitemize}
С одной стороны, решаемые системой \textit{задачи}, можно классифицировать на \textit{информационные задачи} и \textit{поведенческие задачи}.С точки зрения формулировки поставленной задачи можно выделить \textit{декларативные формулировки задачи} и \textit{процедурные формулировки задачи}. Следует отметить, что данные классы задач не противопоставляются, и могут существовать формулировки задач, использующие оба подхода.}\scntext{правило идентификации экземпляров}{Экземпляры класса \textbf{\textit{задач}} в рамках \textit{Русского языка} именуются по следующим правилам:\begin{scnitemize}
\item в начале идентификатора пишется слово ``\textit{Задача} и ставится точка;\item далее с прописной буквы идет либо содержащее глагол совершенного вида в инфинитиве описание сути того, что требуется получить в результате выполнения действия, либо вопросительное предложение, являющееся спецификацией запрашиваемой (ответной) информации.\end{scnitemize}
Например:\\\textit{Задача. Сформировать полную семантическую окрестность понятия треугольник}\\\textit{Задача. Верифицировать Раздел. Предметная область sc-элементов}}
\scnsubset{семантическая окрестность}
\scnsuperset{вопрос}
\scnsuperset{команда}
\scnidtf{спецификация действия, которое выполнилось, выполняется или может быть выполнено соответствующей кибернетической системой}
\scntext{note}{Каждой задаче и, соответственно, каждому специфицируемому действию соответствует определенная кибернетическая система, являющаяся субъектом, выполняющим это действие.}\scnsubset{знание}
\scntext{note}{Каждая \textit{задача} -- это \textit{знание}, описывающее то какое действие возможно потребуется выполнить.}\scnsuperset{инициированная задача}
\scnidtf{формулировка задачи, которая подлежит выполнению}
\scnidtf{спецификация (описание) соответствующего действия}
\scnsuperset{декларативная формулировка задачи}
\scnidtf{задача, в формулировке которой явно указывается (описывается) целевая ситуация, т.е. то, что является результатом выполнения (решения) данной задачи}
\scnsuperset{процедурная формулировка задачи}
\scnidtf{задача, в формулировке которой явно указывается характеристика действия, специфицируемого этой задачей, а именно, например, указывается:\begin{scnitemize}
\item субъект или субъекты, выполняющие это действие,\item объекты, над которыми действие выполняется, -- аргументы действия,\item инструменты, с помощью которых выполняется действие,\item момент и, возможно, дополнительные условия начала и завершения выполнения действия\end{scnitemize}
}
\scnsuperset{декларативно-процедурная формулировка задачи}
\scnidtf{задача, в формулировке которой присутствуют как декларативные (целевые), так и процедурные аспекты}
\scntext{note}{От качества (корректности и полноты) формулировки задачи, т.е. спецификации соответствующего действия, во многом зависит качество (эффективность) выполнения этого действия, т.е. качество процесса решения указанной задачи.}\scnsuperset{проблема}
\scnidtf{проблемная задача}
\scnidtf{сложная, трудно решаемая задача}
\scnsuperset{изобретательская задача}
\scnheader{процедурная формулировка задачи}
\scnidtf{спецификация действия, которое планируется быть выполненным}
\scntext{explanation}{В случае \textbf{\textit{процедурной формулировки задачи}}, в формулировке задачи явно указываются аргументы соответствующего задаче \textit{действия}, и в частности, вводится семантическая типология \textit{действий}. При этом явно не уточняется, что должно быть результатом выполнения данного действия. Заметим, что, при необходимости, \textit{процедурная формулировка задачи} может быть сведена к \textit{декларативной формулировке задачи} путем трансляции на основе некоторого правила, например определения класса действия через более общий класс.}\scnheader{декларативная формулировка задачи}
\scnidtf{описание ситуации (состояния), которое должно быть достигнуто в результате выполнения планируемого действия}
\scntext{explanation}{В случае \textit{декларативной формулировки задачи}, при описании условия задачи специфицируется цель \textit{действия}, т.е. результат, который должен быть получен при успешном выполнении \textit{действия}.}\scnrelto{второй домен}{декларативная формулировка задачи*}
\scnidtf{описание исходной (начальной) ситуации, являющейся условием выполнения соответствующего действия и целевой (конечной) ситуации, являющейся результатом выполнения этого действия}
\scnidtf{семантическая спецификация действия}
\scntext{note}{Формулировка \textit{задачи} может не содержать указания контекста (области решения) \textit{задачи} (в этом случае областью решения \textit{задачи} считается либо вся \textit{база знаний}, либо ее согласованная часть), а также может не содержать либо описания исходной ситуации, либо описания целевой ситуации. Так, например, описания целевой ситуации для явно специфицированного противоречия, обнаруженного в \textit{базе знаний} не требуется.}\scnidtf{формулировка (описание) задачной ситуации с явным или неявным описанием контекста (условий) выполнения специфицируемого действия, а также результата выполнения этого действия}
\scnidtf{явное или неявное описание\begin{scnitemize}
\item того, что \uline{дано} -- исходные данные, условия выполнения специфируемого действия,\item того, что \uline{требуется} -- формулировка цели, результата выполнения указанного действия\end{scnitemize}
}
\scnhaselementrole{пример}{\scnfileimage[20em]{figures/sd_task/declarative_task_statement.png}
}
\scntext{explanation}{Выполнение данного действия сведется к следующим \uline{событиям}:\begin{scnitemize}
\item для числа \textit{с} будет сгенерирован уникальный идентификатор, являющийся его представлением в соответствующей системе счисления\item будет сгенерирована константная настоящая позитивная пара принадлежности, соединяющая узел \textit{вычислено}{} с узлом \textit{с}{}\item удалится константная будущая позитивная пара принадлежности, а также константная настоящая нечеткая пара принадлежности, выходящие из узла \textit{вычислено}.\end{scnitemize}
Таким образом, после выполнения действия \uline{все} \uline{будущие} сущности, входящие в целевую ситуацию, становятся \uline{настоящими} сущностями, а некоторые \uline{настоящие} сущности, входящие в исходную ситуацию, становятся \uline{прошлыми}.}\scnheader{задача}
\scnsuperset{задача, решаемая в памяти кибернетической системы}
\scnsuperset{задача, решаемая в памяти индивидуальной кибернетической системы}
\scnsuperset{задача, решаемая в общей памяти многоагентной системы}
\scnidtf{информационная задача}
\scnidtf{задача, направленная либо на \uline{генерацию} или поиск информации, удовлетворяющей заданным требованиям, либо на некоторое \uline{преобразование} заданной информации}
\scnsuperset{математическая  задача}
\scnsuperset{элементарная информационная задача}
\scnsuperset{простая информационная задача}
\scnsuperset{проблемная информационная задача}
\scnidtf{интеллектуальная информационная задача}
\scnsuperset{проблема Гильберта}
\scnheader{вопрос}
\scnidtf{запрос}
\scnsubset{задача, решаемая в памяти кибернетической системы}
\scnidtf{непроцедурная формулировка задачи на поиск (в текущем состоянии базы знаний) или на генерацию знания, удовлетворяющего заданным требованиям}
\scnsuperset{вопрос -- что это такое}
\scnsuperset{вопрос -- почему}
\scnsuperset{вопрос -- зачем}
\scnsuperset{вопрос -- как}
\scnidtf{каким способом}
\scnidtf{запрос метода (способа) решения заданного (указываемого) вида задач или класса задач либо, плана решения конкретной указываемой задачи}
\scnidtf{задача, направленная на удовлетворение информационной потребности некоторого субъекта-заказчика}
\scnheader{команда}
\scnidtf{инициированная задача}
\scnidtf{спецификация инициированного действия}
\scntext{explanation}{Идентификатор экземпляров конкретного класса \textbf{\textit{команд}} в рамках \textit{Русского языка} пишется с прописной буквы и представляет собой либо содержащее глагол совершенного вида в инфинитиве описание сути того, что требуется получить в результате выполнения действия, соответствующего данной \textbf{\textit{команде}}, либо вопросительное предложение, являющееся спецификацией запрашиваемой (ответной) информации.Например:\\\textit{Сформировать полную семантическую окрестность понятия треугольник}\\\textit{Верифицировать Раздел. Предметная область sc-элементов}}\scnheader{задача}
\scntext{note}{Сужение бинарного ориентированного отношения \textit{спецификация*} (быть спецификацией*), связывающее \textit{действия} с \textit{задачами}, которые решаются в результате выполнения этих \textit{действий}, не является взаимно однозначным.Каждому \textit{действию} может соответствовать несколько формулировок \textit{задач}, которые с разной степенью детализации или с разных аспектов специфицируют указанное \textit{действие}.Кроме того, интерпретация \uline{разных} формулировок семантически одной и той же \textit{задачи} в общем случае приводит к \uline{разным} \textit{действиям}, решающим эту \textit{задачу}.Подчеркнем, что \uline{разные}, но семантически эквивалентные формулировки \textit{задач} считаются формулировками формально \uline{разных} \textit{задач}.}\newpage\scnheader{отношение, заданное на множестве*(задача)}
\scnhaselement{\scnkeyword{задача}
*}
\scniselement{неосновное понятие}
\scnidtf{формулировка задачи*}
\scnidtf{спецификация действия, уточняющая то, \uline{что} должно быть сделано*}
\begin{scnsubdividing}
\scnitem{декларативная формулировка задачи*}
\scnitem{процедурная формулировка задачи*}
\end{scnsubdividing}
\scnrelfrom{второй домен}{\scnkeyword{задача}
}
\scniselement{основное понятие}
\scnsuperset{задача обработки базы знаний}
\scnsuperset{задача обработки файлов}
\scnsuperset{задача, решаемая кибернетической системой во внешней среде}
\scnsuperset{задача, решаемая кибернетической системой в собственной физической оболочке}
\scnhaselement{\scnkeyword{декларативная формулировка задачи}
*}
\scniselement{неосновное понятие}
\scnidtf{описание исходной ситуации и целевой ситуации специфицируемого действия*}
\scntext{explanation}{декларативная формулировка задачи включает в себя:\begin{scnitemize}
\item связку отношения \textit{цель}*, связывающую специфицируемое действие с описанием целевой ситуации;\item само описание целевой ситуации;\item связку отношения \textit{исходная ситуация*}, связывающую специфицируемое действие с описанием исходной ситуации;\item непосредственно описание исходной ситуации;\item указание контекста (области решения) задачи.\end{scnitemize}
При этом указание и описание исходной ситуации может отсутствовать.}\scnhaselement{\scnkeyword{процедурная формулировка задачи}
*}
\scniselement{неосновное понятие}
\scnidtftext{explanation}{указание\begin{scnitemize}
\item \textit{класса действий}, которому принадлежит специфицируемое \textit{действие}, а также\item \textit{субъекта} или субъектов, выполняющих это действие (с дополнительным указанием роли каждого участвующего субъекта);\item \textit{объекта} или объектов, над которыми осуществляется действие (с указанием роли
\end{scnitemize}
\end{scnsubstruct}
\end{SCn}\scnsegmentheader{Предметная область и онтология субъектно-объектных спецификаций воздействий}
\begin{scnsubstruct}
\begin{scnrelfromlist}{соавтор}
\scnitem{Гордей А.Н.}
\scnitem{Никифоров С.А.}
\scnitem{Бобёр Е.С.}
\scnitem{Святощик М.И.}
\end{scnrelfromlist}
\scniselement{предметная область и онтология}
\scnheader{индивид}
\scnidtftext{часто используемый sc-идентификатор}{субъект}
\scnrelfrom{источник}{\cite{Hardzei2005}}
\scnheader{участник воздействия\scnrolesign}
\scnidtf{участник акции\scnrolesign}
\scniselement{ролевое отношение}
\scnrelfrom{первый домен}{индивид}
\scnrelfrom{второй домен}{воздействие}
\scntext{explanation}{\textit{участник акции\scnrolesign} -- это ролевое отношение, которое связывает акцию с участвующим в ней индивидом.}\scnrelfrom{источник}{\cite{Hardzei2021}}
\scnrelfrom{источник}{\cite{Fillmore1977}}
\scnrelfrom{источник}{\cite{Fillmore1982}}
\begin{scnsubdividing}
\scnitem{субъект\scnrolesign\\\scntext{explanation}{\textit{субъект\scnrolesign} -- инициатор акции.}\begin{scnsubdividing}
\scnitem{инициатор\scnrolesign\\}
\scnitem{вдохновитель\scnrolesign\\}
\scnitem{распространитель\scnrolesign\\}
\scnitem{вершитель\scnrolesign\\\scntext{explanation}{\textit{вершитель\scnrolesign} завершает акцию производством из объекта продукта.}}
\end{scnsubdividing}
}
\scnitem{инструмент\scnrolesign\\\scntext{explanation}{\textit{инструмент\scnrolesign} -- исполнитель акции.}\begin{scnsubdividing}
\scnitem{активатор\scnrolesign\\\scntext{explanation}{\textit{активатор\scnrolesign} непосредственно воздействует на медиатор.}}
\scnitem{супрессор\scnrolesign\\\scntext{explanation}{\textit{супрессор\scnrolesign} подавляет сопротивление медиатора.}}
\scnitem{усилитель\scnrolesign\\\scntext{explanation}{\textit{усилитель\scnrolesign} наращивает воздействие на медиатор.}}
\scnitem{преобразователь\scnrolesign\\\scntext{explanation}{\textit{преобразователь\scnrolesign} преобразует медиатор в инструмент.}}
\end{scnsubdividing}
}
\scnitem{медиатор\scnrolesign\\\scntext{explanation}{\textit{медиатор\scnrolesign} -- посредник акции.}\begin{scnsubdividing}
\scnitem{ориентир\scnrolesign\\}
\scnitem{локус\scnrolesign\\\scntext{explanation}{\textit{локус\scnrolesign} частично или полностью окружает объект и тем самым локализует его в пространстве.}}
\scnitem{транспортёр\scnrolesign\\\scntext{explanation}{\textit{транспортёр\scnrolesign} перемещает объект.}}
\scnitem{адаптер\scnrolesign\\\scntext{explanation}{\textit{адаптер\scnrolesign} приспосабливает  инструмент к воздействию на объект.}}
\scnitem{материал\scnrolesign\\\scntext{explanation}{\textit{материал\scnrolesign} используется в качестве объекта-сырья для производства продукта.}}
\scnitem{макет\scnrolesign\\\scntext{explanation}{\textit{макет\scnrolesign} является исходным образцом для производства из объекта продукта.}}
\scnitem{фиксатор\scnrolesign\\\scntext{explanation}{\textit{фиксатор\scnrolesign} превращает переменный локус объекта в постоянный.}}
\scnitem{ресурс\scnrolesign\\\scntext{explanation}{\textit{ресурс\scnrolesign} питает инструмент.}}
\scnitem{стимул\scnrolesign\\\scntext{explanation}{\textit{стимул\scnrolesign} проявляет параметр объекта.}}
\scnitem{регулятор\scnrolesign\\\scntext{explanation}{\textit{регулятор\scnrolesign} служит инструкцией в производстве из объекта продукта.}}
\scnitem{хронотоп\scnrolesign\\\scntext{explanation}{\textit{хронотоп\scnrolesign} локализует объект во времени.}}
\scnitem{источник\scnrolesign\\\scntext{explanation}{\textit{источник\scnrolesign} обеспечивает инструкциями инструмент.}}
\scnitem{индикатор\scnrolesign\\\scntext{explanation}{\textit{индикатор\scnrolesign} отображает параметр воздействия на объект или параметр продукта как результата воздействия на объект.}}
\end{scnsubdividing}
}
\scnitem{объект\scnrolesign\\\scntext{explanation}{\textit{объект\scnrolesign} -- реципиент акции.}\begin{scnsubdividing}
\scnitem{покрытие\scnrolesign\\\scntext{explanation}{\textit{покрытие\scnrolesign} -- внешняя изоляция оболочки индивида.}}
\scnitem{корпус\scnrolesign\\\scntext{explanation}{\textit{корпус\scnrolesign} -- оболочка индивида.}}
\scnitem{прослойка\scnrolesign\\\scntext{explanation}{\textit{прослойка\scnrolesign} -- внутренняя изоляция оболочки индивида.}}
\scnitem{сердцевина\scnrolesign\\\scntext{explanation}{\textit{сердцевина\scnrolesign} -- ядро индивида.}}
\end{scnsubdividing}
}
\scnitem{продукт\scnrolesign\\\scntext{explanation}{\textit{продукт\scnrolesign} -- результат воздействия субъекта на объект.}\begin{scnsubdividing}
\scnitem{заготовка\scnrolesign\\\scntext{explanation}{\textit{заготовка\scnrolesign} -- превращённый в сырьё объект.}}
\scnitem{полуфабрикат\scnrolesign\\\scntext{explanation}{\textit{полуфабрикат\scnrolesign} -- наполовину изготовленный из сырья продукт.}}
\scnitem{прототип\scnrolesign\\\scntext{explanation}{\textit{прототип\scnrolesign} -- опытный образец продукта.}}
\scnitem{изделие\scnrolesign\\\scntext{explanation}{\textit{изделие\scnrolesign} -- готовый продукт.}}
\end{scnsubdividing}
}
\end{scnsubdividing}
\scnheader{воздействие}
\scnidtf{акция}
\scnrelfrom{источник}{\cite{Hardzei2017}}
\scnrelfrom{разбиение}{\scnkeyword{Типология по характеру взаимодействия участников\scnsupergroupsign}
}
\begin{scneqtoset}
\scnitem{воздействие активизации\\\scntext{explanation}{\textit{воздействие активизации} -- воздействие, в ходе которого взаимодействие осуществляется между субъектом и инструментом.}}
\scnitem{воздействие эксплуатации\\\scntext{explanation}{\textit{воздействие эксплуатации} -- воздействие, в ходе которого взаимодействие осуществляется между инструментом и медиатором.}}
\scnitem{воздействие трансформации\\\scntext{explanation}{\textit{воздействие трансформации} -- воздействие, в ходе которого взаимодействие осуществляется между объектом и продуктом.}}
\scnitem{воздействие нормализации\\\scntext{explanation}{\textit{воздействие нормализации} -- воздействие, в ходе которого взаимодействие осуществляется между объектом и продуктом.}}
\end{scneqtoset}
\scnrelfrom{разбиение}{\scnkeyword{Типология воздействий по виду взаимодействующих подсистем\scnsupergroupsign}
}
\begin{scneqtoset}
\scnitem{воздействие среда-оболочка\\}
\scnitem{воздействие оболочка-ядро\\}
\scnitem{воздействие ядро-оболочка\\}
\scnitem{воздействие оболочка-среда}
\end{scneqtoset}
\scnheader{воздействие}
\scnrelfrom{разбиение}{\scnkeyword{Типология воздействий по фазам наращивания воздействия\scnsupergroupsign}
}
\begin{scneqtoset}
\scnitem{воздействие инициации\\\scntext{explanation}{\textit{воздействие инициации} -- воздействие, в ходе которого оно начинается в каждой подсистеме.}}
\scnitem{воздействие аккумуляции\\\scntext{explanation}{\textit{воздействие аккумуляции} -- воздействие, в ходе которого происходит его накапливание в каждой подсистеме.}}
\scnitem{воздействие амплификации\\\scntext{explanation}{\textit{воздействие амплификации} -- воздействие, в ходе которого происходит его усиление.}}
\scnitem{воздействие генерации\\\scntext{explanation}{\textit{воздействие генерации} -- воздействие, которое представляет собой переход в каждой подсистеме с одного уровня, например, среды-оболочки, на другой, например, оболочки-ядра.}}
\end{scneqtoset}
\scnheader{воздействие}
\scnrelfrom{разбиение}{\scnkeyword{Типология воздействий по виду инструмента\scnsupergroupsign}
}
\begin{scneqtoset}
\scnitem{физическое воздействие\\\scntext{explanation}{\textit{физическое воздействие} -- воздействие, в котором в роли инструмента выступает оболочка субъекта.}}
\scnitem{информационное воздействие\\\scntext{explanation}{\textit{информационное воздействие} -- воздействие, в котором в роли инструмента выступает среда субъекта.}}
\end{scneqtoset}
\scnheader{Рис. Таблица воздействий}
\scneqfile{\\\includegraphics{figures/sd_actions/macroproc_table.png}\\}
\scnrelfrom{источник}{\cite{Hardzei2017}}
\scntext{explanation}{На изображении представлена типология \textit{воздействий}. Любое воздействие характеризуется принадлежностью четырём классам, соответствующим признакам классификации. Заштрихованы \textit{физические воздействия}.}\scnheader{Специфицируемые классы воздействий}
\scnsuperset{\begin{scnset}
формование\\\begin{scnreltoset}{пересечение}
\scnitem{воздействие трансформации}
\scnitem{воздействие среда-оболочка}
\scnitem{воздействие генерации}
\scnitem{физическое воздействие}
\end{scnreltoset}
;притягивание\\\begin{scnreltoset}{пересечение}
\scnitem{воздействие активизации}
\scnitem{воздействие среда-оболочка}
\scnitem{воздействие инициации}
\scnitem{физическое воздействие}
\end{scnreltoset}
;выхолащивание\\\begin{scnreltoset}{пересечение}
\scnitem{воздействие трансформации}
\scnitem{воздействие ядро-оболочка}
\scnitem{воздействие генерации}
\scnitem{физическое воздействие}
\end{scnreltoset}
;аннигилирование\\\begin{scnreltoset}{пересечение}
\scnitem{воздействие трансформации}
\scnitem{воздействие оболочка-среда}
\scnitem{воздействие генерации}
\scnitem{физическое воздействие}
\end{scnreltoset}
;введение\\\begin{scnreltoset}{пересечение}
\scnitem{воздействие эксплуатации}
\scnitem{воздействие оболочка-ядро}
\scnitem{воздействие инициации}
\scnitem{физическое воздействие}
\end{scnreltoset}
;распускание\\\begin{scnreltoset}{пересечение}
\scnitem{воздействие трансформации}
\scnitem{воздействие оболочка-среда}
\scnitem{воздействие амплификации}
\scnitem{физическое воздействие}
\end{scnreltoset}
;разжимание\\\begin{scnreltoset}{пересечение}
\scnitem{воздействие трансформации}
\scnitem{воздействие ядро-оболочка}
\scnitem{воздействие амплификации}
\scnitem{физическое воздействие}
\end{scnreltoset}
;разъединение\\\begin{scnreltoset}{пересечение}
\scnitem{воздействие эксплуатации}
\scnitem{воздействие ядро-оболочка}
\scnitem{воздействие генерации}
\scnitem{физическое воздействие}
\end{scnreltoset}

\end{scnset}
}
\scnheader{Пример sc.g-текста, описывающего спецификацию воздействия}
\scneq{\scnfileimage[20em]{figures/sd_actions/tapaz_description_example.png}}
\scniselement{sc.g-текст}
\scntext{explanation}{Представленный фрагмент базы знаний содержит декомпозицию воздействия во времени, указание принадлежности данного декомпозируемого воздействия и полученных в результате данной декомпозиции воздействий определенному их классу из приведенной выше классификации, а также указание участников данных акций.}\scntext{explanation}{Представленный фрагмент базы знаний можно протранслировать в следующий текст естественного языка: <<Некто принимает молоко, затем окисляет молоко, а именно: нормализует молоко до 15-процентной жирности, затем очищает молоко, затем пастеризует молоко, затем охлаждает молоко до определённой температуры, затем вносит закваску в молоко, затем сквашивает молоко, затем режет сгусток, затем подогревает сгусток, затем обрабатывает сгусток, затем отделяет сыворотку, затем охлаждает сгусток и, в итоге, производит творог>>.}\scnheader{субъект}
\scnidtftext{часто используемый sc-идентификатор}{индивид}
\scnidtf{активная сущность}
\scnidtf{сущность, способная самостоятельно выполнять некоторые виды действий}
\scnidtf{агент деятельности}
\scnsuperset{Собственное Я}
\scnsuperset{внутренний субъект ostis-системы}
\scnsuperset{внешний субъект ostis-системы, с которым осуществляется взаимодействие}
\scnsuperset{внешний субъект ostis-системы, с которым взаимодействие не происходит}
\scnheader{внутренний субъект ostis-системы}
\scnidtf{субъект, входящий в состав той \textit{ostis-системы, в базе знаний} которой он описывается}
\scnsuperset{sc-агент}
\scntext{explanation}{Под \textit{внутренним субъектом ostis-системы} понимается такой \textit{субъект}, который выполняет некоторые \textit{действия} в \uline{той же памяти}, в которой хранится его знак.\newlineК числу \textit{внутренних субъектов ostis-системы} относятся входящие в нее \textit{sc-агенты}, частные sc-машины, целые интеллектуальные подсистемы.}\scnheader{внешний субъект ostis-системы, с которым осуществляется взаимодействие}
\scntext{explanation}{К числу \textit{внешних субъектов ostis-системы, с которыми осуществляется взаимодействие}, относятся конечные пользователи \textit{ostis-системы}, ее разработчики, а также другие компьютерные системы(причем не только интеллектуальные).}\scnheader{субъект действия\scnrolesign}
\scnsubset{субъект\scnrolesign}
\scnidtf{сущность, воздействующая на некоторую другую сущность в процессе заданного действия\scnrolesign}
\scnidtf{сущность, создающая \textit{причину} изменений другой сущности (объекта действия)\scnrolesign}
\scnidtf{быть субъектом данного действия\scnrolesign}
\scnsuperset{субъект неосознанного воздействия\scnrolesign}
\scnsuperset{субъект осознанного воздействия\scnrolesign}
\scnidtf{субъект целенаправленного, активного воздействия\scnrolesign}
\scnheader{исполнитель*}
\scntext{explanation}{Связки отношения \textit{исполнитель*} связывают \textit{sc-элементы}, обозначающие \textit{действие} и \textit{sc-элементы}, обозначающие \textit{субъекта}, который предположительно будет осуществлять, осуществляет или осуществлял выполнение указанного \textit{действия}. Данное отношение может быть использовано при назначении конкретного исполнителя для проектной задачи по развитию баз знаний.В случае, когда заранее неизвестно, какой именно \textit{субъект*} будет исполнителем данного \textit{действия}, отношение \textit{исполнитель*} может отсутствовать в первоначальной формулировке \textit{задачи} и добавляться позже, уже непосредственно при исполнении.Когда действие выполняется (является \textit{настоящей сущностью}) или уже выполнено (является \textit{прошлой сущностью}), то исполнитель этого действия в каждый момент времени уже определён. Но когда действие только инициировано, тогда важно знать:\begin{enumerate}
\item кто \uline{хочет} выполнить это действие и насколько важно для него стать исполнителем данного действия;\item кто \uline{может} выполнить данное действие и каков уровень его квалификации и опыта;\item кто и кому поручает выполнить это действие и каков уровень ответственности за невыполнение (приказ, заказ, официальный договор, просьба...)\end{enumerate}
При этом следует помнить, что связь отношения \textit{исполнитель*} в данном случае также является временной прогнозируемой сущностью.Первым компонентом связок отношений \textit{исполнитель*} является знак \textit{действия}, вторым -- знак \textit{субъекта} исполнителя}\scnheader{объект воздействия\scnrolesign}
\scnsubset{объект\scnrolesign}
\scnidtf{сущность, на которую осуществляется воздействие в рамках заданного действия\scnrolesign}
\scnidtf{сущность, являющаяся в рамках заданного действия исходным условием (аргументом), необходимым для выполнения этого действия\scnrolesign}
\scntext{note}{Для разных действий количество объектов действий может быть различным.}\scntext{note}{Поскольку действие является процессом и, соответственно, представляет собой \textit{динамическую структуру}, то и знак \textit{субъекта действия\scnrolesign}, и знак \textit{объекта действия\scnrolesign} являются элементами данной структуры. В связи с этим можно рассматривать отношения \textit{субъект действия\scnrolesign} и \textit{объект действия\scnrolesign} как \textit{ролевые отношения}. Данный факт не  запрещает вводить аналогичные \textit{неролевые отношения}, однако это нецелесообразно.}\scnheader{продукт\scnrolesign}
\scnidtf{быть продуктом заданного действия\scnrolesign}
\scnsubset{продукт*}
\scnsubset{результат*}
\scnidtf{сухой остаток\scnrolesign}
\scnidtf{то, ради чего может быть выполнено, выполняется или будет выполняться заданное действие\scnrolesign}
\scntext{note}{Продуктом действия может быть некоторая материальная сущность, некоторое множество (тираж) одинаковых материальных сущностей, некоторая информационная конструкция}\scnheader{результат*}
\scntext{explanation}{Связки отношения \textit{результат*} связывают \textit{sc-элемент}, обозначающий \textit{действие}, и \textit{sc-конструкцию}, описывающую результат выполнения рассматриваемого действия, другими словами, цель, которая должна быть достигнута при выполнении \textit{действия}.Результат может специфицироваться как атомарным высказыванием, так и неатомарным, т.е. конъюнктивным, дизъюнктивным, строго дизъюнктивным и т.д.В случае, когда успешное выполнение \textit{действия} приводит к изменению какой-либо конструкции в \textit{\mbox{sc-памяти}}, которое необходимо занести в историю изменений базы знаний или использовать для демонстрации протокола решении задачи, генерируется соответствующая связка отношения \textit{результат*}, связывающая задачу и \textit{sc-конструкцию}, описывающую данное изменение. Конкретный вид указанной \textit{\mbox{sc-конструкции}} зависит от типа действия.}\scnrelboth{следует отличать}{цель*}
\scnidtf{спецификация планируемого результата*}
\scntext{note}{Следует также отмечать то, что является непосредственно результатом (продуктом) некоторого действия, и то, что является предварительной (исходной, стартовой) спецификацией этого  результата. Далеко не всегда результатом действия является именно то, что планировалось (было целью) изначально.}\scnheader{класс выполняемых действий*}
\scnidtf{класс действий, выполняемых классом субъектов*}
\scntext{explanation}{Связки отношения \textit{класс выполняемых действий*} связывают классы \textit{субъектов} и классы действий, при этом предполагается, что каждый субъект указанного класса способен выполнять действия указанного класса действий.}\scnheader{заказчик*}
\scntext{explanation}{Связки отношения \textit{заказчик*} связывают классы \textit{sc-элементы}, обозначающие \textit{действие}, и \textit{sc-элементы}, обозначающие  \textit{субъекта}, который заинтересован в выполнении данного действия и, как правило, инициирует его выполнение. Данное отношение может быть использовано при указании того, кто поставил проектную задачу по развитию баз знаний.Первым компонентов связок отношения \textit{заказчик*} является знак \textit{действия}, вторым -- знак \textit{субъекта}.}\scnheader{инициатор*}
\scntext{explanation}{Связки отношения \textit{инициатор*} связывают \textit{sc-элемент}, обозначающий \textit{инициированное действие}, и знак \textit{субъекта}, который является инициатором данного \textit{действия}, то есть \textit{субъектом}, который инициировал данное \textit{действие} и, как правило, заинтересован в его успешном выполнении.}\scnheader{объект\scnrolesign}
\scnidtf{аргумент действия\scnrolesign}
\scntext{explanation}{Связки отношения \textit{объект\scnrolesign} связывают \textit{sc-элемент}, обозначающий \textit{действие}, и знак той сущности, над которой (по отношению к которой) осуществляется данное \textit{действие}, и, например, знак \textit{структуры}, подлежащий верификации.}\scnsuperset{первый аргумент действия\scnrolesign}
\scnsuperset{второй аргумент действия\scnrolesign}
\scnsuperset{третий аргумент действия\scnrolesign}
\scnheader{класс аргументов*}
\scnidtf{класс аргументов класса команд*}
\scnidtf{быть классом sc-элементов, экземпляры которого являются аргументами для заданного класса команд*}
\scnsuperset{класс первых аргументов*}
\scnsuperset{класс вторых аргументов*}
\scntext{explanation}{Связки отношения \textit{класс аргументов*} связывают \textit{классы команд} (подмножества множества \textit{команд}) и классы \textit{sc-элементов}, которые могут быть аргументами действий, соответствующих данному  \textit{классу команд}. В случае, когда  \textit{команды} данного класса имеют один аргумент, используется собственно отношение  \textit{класс аргументов*}, в случае, когда команды данного класса имеют более одного аргумента, то используются подмножества данного отношения, такие как \textit{класс первых аргументов*}, \textit{класс вторых аргументов*} и т.д.Если для некоторого \textit{класса команд} не указан тип какого-либо из аргументов, то предполагается, что в качестве данного аргумента может выступать любой \textit{sc-элемент}.Первым компонентом связок отношения \textit{класс аргументов*} является знак \textit{класса команд}, вторым -- знак класса \textit{sc-элемента}, которые могут быть \textit{аргументами действий\scnrolesign}, соответствующих данному \textit{классу команд}.}
\end{scnsubstruct}\begin{SCn}
\scnsegmentheader{Уточнение понятий план сложного действия, классы действий, класса задач, метода}
\begin{scnsubstruct}
\scniselement{сегмент базы знаний}
\newpage\scnheader{план сложного действия}
\scnidtf{план}
\scnidtf{план выполнения сложного действия}
\scnidtf{план решения \textit{сложной задачи}}
\scnidtf{план выполнения действия}
\scnidtf{спецификация выполнения действия}
\scnidtf{декомпозиция выполняемого действия на систему последовательно/параллельно выполняемых поддействий*}
\scnidtf{описание того, как может быть выполнено соответствующее сложное действие}
\scnidtf{спецификация соответствующего действия, уточняющая то, \uline{как} предполагается выполнять это действие}
\scnidtf{план решения задачи (выполнения сложного действия) путем описания последовательности выполнения поддействий с описанием того, как передается управление от одних поддействий другим, как осуществляется распараллеливание, как организуется выполнение циклов}
\scntext{definition}{вид спецификации \textit{сложного действия}, представляющий собой систему \textit{задач}, \textit{интерпретация} которой (предполагающая решение указанных \textit{задач} в определенной последовательности) обеспечивает выполнение специфицируемого \textit{сложного действия}}\scnsubset{знание}
\scntext{explanation}{Каждый \textit{план} представляет собой \textit{семантическую окрестность, ключевым sc-элементом\scnrolesign} является \textit{действие}, для которого дополнительно детализируется предполагаемый процесс его выполнения. Основная задача такой детализации -- локализация области базы знаний, в которой предполагается работать, а также набора агентов, необходимого для выполнения  описываемого действия. При этом детализация не обязательно должна быть доведена до уровня элементарных действий, цель составления плана -- уточнение подхода к решению той или иной задачи, не всегда предполагающее составления подробного пошагового решения.При описании \textit{плана} может быть использован как процедурный, так и декларативный подход. В случае процедурного подхода для соответствующего \textit{действия} указывает его декомпозиция на более частные поддействия, а также необходимая спецификация этих поддействий. В случае декларативного подхода указывается набор подцелей (например, при помощи логических утверждений), достижение которых необходимо для выполнения рассматриваемого \textit{действия}. На практике оба рассмотренных подхода можно комбинировать.В общем случае \textit{план} может содержать и переменные, например в случае, когда часть плана задается в виде цикла (многократного повторения некоторого набора действий). Также план может содержать константы, значение которых в настоящий момент не установлено и станет известно, например, только после выполнения предшествующих ему \textit{действий}.Каждый \textit{план} может быть задан заранее как часть спецификации \textit{действия}, т.е. \textit{задачи}, а может формироваться \textit{субъектов} уже собственно в процессе выполнения \textit{действия}, например, в случае использования стратегии разбиения задачи над подзадачи. В первом случае \textit{план} \textit{включается*} в \textit{задачу}, соответствующую тому же действию.}\begin{scnsubdividing}
\scnitem{процедурный план сложного действия\\\scnidtf{декомпозиция \textit{сложного действия} на множество последовательно и/или параллельно выполняемых \textit{поддействий}}
}
\scnitem{непроцедурный план сложного действия\\\scnidtf{декомпозиция исходной \textit{задачи}, соответствующей заданному \textit{сложному действию}, на иерархическую систему и/или подзадач}
}
\end{scnsubdividing}
\scnheader{процедурный план сложного действия}
\scntext{note}{В \textit{процедурном плане выполнения сложного действия} соответствующие \textit{поддействия*} декомпозируемого \textit{сложного действия} представляются специфицирующими их \textit{задачами}. Но, кроме такого рода \textit{задач}, в \textit{процедурный план выполнения сложного действия} входят также \textit{задачи}, которые специфицируют \textit{действия}, обеспечивающие:\begin{scnitemize}
\item синхронизацию выполнения \textit{поддействий*} заданного \textit{сложного действия};\item передачу управления указанным \textit{поддействиям*} (а точнее, соответствующим им \textit{задачам}), т.е. инициирование указанных \textit{поддействий*} (и соответствующих им \textit{задач}).\end{scnitemize}
}\scnheader{действие управления интерпретацией процедурного плана сложного действия}
\scnrelboth{семантически близкий знак}{задача управления интерпретацией процедурного плана сложного действия}
\scnsuperset{безусловная передача управления от одного поддействия к другому}
\scnsuperset{инициирование заданного поддействия при возникновении в базе знаний ситуации или события заданного вида}
\scnsuperset{инициирование заданного множества поддействий при успешном завершении выполнения \uline{всех} поддействий другого заданного множества}
\scnsuperset{инициирование заданного множества поддействий при успешном завершении выполнения \uline{по крайней мере одного} поддействия другого заданного множества}
\scntext{note}{Выделенные классы \textit{действий управления интерпретацией процедурного плана сложного действия} дают возможность реализовать различные виды параллелизма, если это позволяет задача.}\scnheader{класс действий}
\scnrelto{семейство подклассов}{действие}
\scnidtftext{explanation}{\uline{максимальное} множество аналогичных (похожих в определенном смысле) действий, для которого существует (но не обязательно известных в текущий момент) по крайней мере один \textit{метод} (или средство), обеспечивающий выполнение \uline{любого} действия из указанного множества действий}
\scnidtf{множество однотипных действий}
\scnsuperset{класс элементарных действий}
\scnsuperset{класс легковыполнимых сложных действий}
\scntext{note}{Тот факт, что каждому выделяемому \textit{классу действий} соответствует по крайней мере один общий для них \textit{метод} выполнения этих \textit{действий}, означает то, что речь идет о  кластеризации множества \textit{действий}, т.е. выделении \textit{классов действий} по признаку \uline{семантической близости} (сходства) \textit{действий}, входящих в состав выделяемого \textit{класса действий}. При этом прежде всего учитывается аналогичность (сходство) \textit{исходных ситуаций и целевых ситуаций} рассматриваемых \textit{действий}, т.е. аналогичность \textit{задач}, решаемых в результате выполнения соответствующих \textit{действий}. Поскольку одна и та же \textit{задача} может быть решена в результате выполнения нескольких \uline{разных} \textit{действий}, принадлежащих \uline{разным} \textit{классам действий}, следует говорить не только о \textit{классах действий} (множествах аналогичных действий), но и о \textit{классах задач} (о множествах аналогичных задач), решаемых этими \textit{действиями}. Так, например, на множестве \textit{классом действий} заданы следующие \textit{отношения}:\begin{scnitemize}
\item \textit{отношение}, каждая связка которого связывает два разных (непересекающихся) \textit{класса действий}, осуществляющих решение одного и того же \textit{класса задач};\item \textit{отношение}, каждая связка которого связывает два разных \textit{класса действий}, осуществляющих решение разных \textit{классов задач}, один из которых является \textit{надмножеством} другого.\end{scnitemize}
}\scntext{правило идентификации экземпляров}{Конкретные \textit{классы действий} в рамках \textit{Русского языка} именуются по следующим правилам:\begin{scnitemize}
\item в начале идентификатора пишется слово ``\textit{действие} и ставится точка;\item далее со строчной буквы идет либо содержащее глагол совершенного вида в инфинитиве описание сути того, что требуется получить в результате выполнения действий данного класса, либо вопросительное предложение, являющееся спецификацией запрашиваемой (ответной) информации.\end{scnitemize}
Например:\newline\textit{действие, сформировать полную семантическую окрестность указываемой сущности\newlineдействие, верифицировать заданную структуру}Допускается использовать менее строгие идентификаторы, которые, однако, обязаны оперировать словом ``\textit{действие} и достаточно четко специфицировать суть действий описываемого класса.Например:\newline\textit{действие редактирования базы знаний}\newline\textit{действие, направленное на установление темпоральных характеристик указываемой сущности} }
\newpage\scnheader{класс элементарных действий}
\scnidtf{множество элементарных действий, указание принадлежности которому является \uline{необходимым} и достаточным условием для выполнения этого действия}
\scntext{note}{Множество всевозможных элементарных действий, выполняемых каждым субъектом, должно быть \uline{разбито} на классы элементарных действий.}\scntext{explanation}{Принадлежность некоторого \textit{класса действий} множеству \textit{классу элементарных действий}, фиксирует факт того, что при указании всех необходимых аргументов принадлежности \textit{действия} данному классу достаточно для того, чтобы некоторый субъект мог приступить к выполнению этого действия.При этом, даже если \textit{класс действий} принадлежит множеству \textit{класс элементарных действий}, не запрещается вводить более частные \textit{классы действий}, для которых, например, заранее фиксируется один из аргументов.Если конкретный \textit{класс элементарных действий} является более частным по отношению к \textit{действиям в sc-памяти}, то это говорит о наличии в текущей версии системы как минимум одного \textit{sc-агента}, ориентированного на выполнение действий данного класса.}\scnheader{класс легковыполнимых сложных действий}
\scnidtf{множество сложных действий, для которого известен и доступен по крайней мере один \textit{метод}, интерпретация которого позволяет осуществить полную (окончательную, завершающуюся элементарными действиями) декомпозицию на поддействия \uline{каждого} сложного действия из указанного выше множества}
\scnidtf{множество всех сложных действий, выполнимых с помощью известного \textit{метода}, соответствующего этому множеству}
\scntext{explanation}{Принадлежность некоторого \textit{класса действий} множеству \textit{класс легковыполнимых сложных действий} фиксирует факт того, что даже при указании всех необходимых аргументов принадлежности \textit{действия} данному классу недостаточно для того, чтобы некоторый \textit{субъект} приступил к выполнению этого действия, и требуются дополнительные уточнения.}\scnheader{сужение отношения по первому домену*(спецификация*;класс действий*)}
\scnidtftext{часто используемый идентификатор}{спецификация класса действий*}
\begin{scnsubdividing}
\scnitem{обобщенная формулировка задач соответствующего класса*\\\begin{scnsubdividing}
\scnitem{обобщенная декларативная формулировка задач соответствующего класса*\\}
\scnitem{обобщенная процедурная формулировка задач соответствующего класса*\\}
\end{scnsubdividing}
}
\scnitem{метод*\\\scnidtf{метод решения задач заданного класса*}
\scnidtf{метод выполнения действий соответствующего (заданного) класса*}
\begin{scnsubdividing}
\scnitem{процедурный метод выполнения действий соответствующего класса*\\\scnidtf{обобщенный план выполнения действий заданного класса*}
}
\scnitem{декларативный метод выполнения действий соответствующего класса*\\\scnidtf{обобщенная декларативная спецификация выполнения действий заданного класса*}
}
\end{scnsubdividing}
}
\end{scnsubdividing}
\scnheader{класс задач}
\scnidtf{множество аналогичных задач}
\scnidtf{множество задач, для которого можно построить обобщенную формулировку задач, соответствующую всему этому множеству задач}
\scntext{note}{Каждая \textit{обобщенная формулировка задач соответствующего класса} по сути есть не что иное, как строгое логическое определение указанного класса задач.}\scnrelto{семейство подмножеств}{задача}
\scntext{правило идентификации экземпляров}{Конкретные \textit{классы задач} в рамках \textit{Русского языка} именуются по следующим правилам:\begin{scnitemize}
\item в начале идентификатора пишется слово ``\textit{задача} и ставится точка;\item далее с прописной буквы идет либо содержащее глагол совершенного вида в инфинитиве описание сути того, что требуется получить в результате решения данного \textit{класса задач}, либо вопросительное предложение, являющееся спецификацией запрашиваемой (ответной) информации.\end{scnitemize}
Например:\\\textit{задача. сформировать полную семантическую окрестность указываемой сущности}\\\textit{задача. верифицировать заданную структуру}Допускается использовать менее строгие идентификаторы, которые, однако, обязаны оперировать словом ``\textit{задача} и достаточно четко специфицировать суть задач описываемого класса. Например:\\\textit{задача на установление значения величины}\\\textit{задача на доказательство}}
\scnheader{класс команд}
\scnrelto{семейство подмножеств}{задача}
\scnsuperset{класс интерфейсных пользовательских команд}
\scnsuperset{класс интерфейсных команд пользователя ostis-системы}
\scnsuperset{класс команд без аргументов}
\scnsuperset{класс команд с одним аргументом}
\scnsuperset{класс команд с двумя аргументами}
\scnsuperset{класс команд с произвольным числом аргументов}
\scntext{explanation}{Идентификатор конкретного класса \textit{класса команд} в рамках \textit{Русского языка} пишется со строчной буквы и представляет собой либо содержащее глагол совершенного вида в инфинитиве описание сути того, что требуется получить в результате выполнения действий, соответствующих данному \textit{классу команд}, либо вопросительное предложение, являющееся спецификацией запрашиваемой (ответной) информации. Например:\\\textit{сформировать полную семантическую окрестность указываемой сущности}\\\textit{верифицировать заданную структуру}Допускается использовать менее строгие идентификаторы, которые, однако, обязаны оперировать словом ``\textit{команда} и достаточно четко специфицировать суть задач описываемого класса. Например:\\\textit{команда редактирования базы знаний}\\\textit{команда установления темпоральных характеристик указываемой сущности}}\begin{scnsubdividing}
\scnitem{атомарный класс команд}
\scnitem{неатомарный класс команд}
\end{scnsubdividing}
\scnheader{атомарный класс команд}
\scntext{explanation}{Принадлежность некоторого \textit{класса команд} множеству \textit{атомарных классов команд} фиксирует факт того, что данная спецификация является достаточной для того, чтобы некоторый субъект приступил к выполнению соответствующего действия.При этом, даже если \textit{класса команд} принадлежит множеству \textit{атомарных классов команд} не запрещается вводить более частные \textit{классы команд}, в состав которых входит информация, дополнительно специфицирующая соответствующее \textit{действие}.Если соответствующий данному \textit{классу команд класс действий} является более частным по отношению к \textit{действиям в sc-памяти}, то попадание данного класса команд во множество \textit{атомарных классов команд} говорит о наличии в текущей версии системы как минимум одного \textit{sc-агента}, условие инициирования которого соответствует формулировке команд данного класса.}\scnheader{неатомарный класс команд}
\scntext{explanation}{Принадлежность некоторого \textit{класса команд} множеству \textit{неатомарных классов команд} фиксирует факт того, что данная спецификация не является достаточной для того, чтобы некоторый субъект приступил к выполнению соответствующего действия, и требует дополнительных уточнений.}\scnheader{класс действий}
\begin{scnsubdividing}
\scnitem{\textit{класс действий, однозначно задаваемый решаемым классом задач}\\\scnidtf{\textit{класс действий}, обеспечивающих решение соответствующего \textit{класса задач} и использующих при этом любые, самые разные \textit{методы} решения задач этого класса}
}
\scnitem{\textit{класс действий, однозначно задаваемый используемым методом решения задач}}
\end{scnsubdividing}
\scnheader{метод}
\scnrelto{второй домен}{метод*}
\scnidtf{описание того, \uline{как} может быть выполнено любое или почти любое действие, принадлежащее соответствующему классу действий}
\scnidtf{метод решения соответствующего класса задач, обеспечивающий решение любой или большинства задач указанного класса}
\scnidtf{обобщенная спецификация выполнения действий соответствующего класса}
\scnidtf{обобщенная спецификация решения задач соответствующего класса}
\scnidtf{программа решения задач соответствующего класса, которая может быть как процедурной, так и декларативной (непроцедурной)}
\scnidtf{знание о том, как можно решать задачи соответствующего класса}
\scnsubset{знание}
\scniselement{вид знаний}
\scnidtf{способ}
\scnidtf{знание о том, как надо решать задачи соответствующего класса задач (множества эквивалентных (однотипных, похожих) задач)}
\scnidtf{метод (способ) решения некоторого (соответствующего) класса задач}
\scnidtf{информация (знание), достаточная для того, чтобы решить любую \textit{задачу}, принадлежащую соответствующему \textit{классу задач} с помощью соответствующей \textit{модели решения задач}}
\scnidtf{обобщенный план выполнения некоторого класса сложных действий (или обобщенный план решения соответствующего класса сложных задач), привязка которого к конкретной задаче указанного класса и последующая интерпретация обеспечивает решение любой или почти любой задачи этого класса}
\scntext{note}{Очевидно, что трудоемкость разработки метода определяется не столько мощностью класса задач, решаемых с помощью разрабатываемого метода, сколько их семантической близостью, аналогичностью.}\scnidtf{метод решения соответствующего класса задач}
\scnidtf{метод выполнения соответствующего класса действий}
\scnidtf{пассивный метод, хранимый в базе знаний и используемый соответствующими коллективами агентов при соответствующем их инициировании}
\scnidtf{метод выполнения действий некоторого класса или метод решения задач некоторого класса или метод, который может быть использован для выполнения некоторого конкретного действия или для решения некоторой конкретной задачи}
\scnidtf{обобщенное описание того, как можно выполнить действия из соответствующего класса действий или как можно решить задачу из соответствующего класса задач}
\scnsuperset{метод сведения задач к подзадачам}
\scnidtf{класс логически эквивалентных методов, обеспечивающих решение задач путем сведения этих задач к подзадачам и отличающихся только деталями формализации}
\scntext{note}{В состав спецификации каждого \textit{класса задач} входит описание способа привязки \textit{метода} к исходным данным конкретной \textit{задачи}, решаемой с помощью этого \textit{метода}. Описание такого способа привязки включает в себя:\begin{scnitemize}
\item набор переменных, которые входят как в состав \textit{метода}, так и в состав \textit{обобщенной формулировки задач соответствующего класса} и значениями которых являются соответствующие элементы исходных данных каждой конкретной решаемой задачи;\item часть \textit{обобщенной формулировки задач} того класса, которому соответствует рассматриваемый \textit{метод}, являющихся описанием \uline{условия применения} этого \textit{метода}.\end{scnitemize}
\bigskipСама рассматриваемая привязка \textit{метода} к конкретной \textit{задаче}, решаемой с помощью этого \textit{метода} осуществляется путем \uline{поиска} в \textit{базе знаний} такого фрагмента, который удовлетворяет условиям применения указанного \textit{метода}. Одним из результатов такого поиска и является установление соответствия между указанными выше переменными используемого \textit{метода} и значениями этих переменных в рамках конкретной решаемой \textit{задачи}. Другим вариантом установления рассматриваемого соответствия является явное обращение (вызов, call) соответствующего \textit{метода} (программы) с явной передачей соответствующих параметров. Но такое не всегда возможно, т.к. при выполнении процесса решения конкретной \textit{задачи} на основе декларативной спецификации выполнения этого действия нет возможности установить:\begin{scnitemize}
\item когда необходимо инициировать вызов (использование) требуемого \textit{метода};\item какой конкретно \textit{метод} необходимо использовать;\item какие параметры, соответствующие конкретной инициируемой \textit{задаче}, необходимо передать для привязки используемого \textit{метода} к этой \textit{задаче}.\end{scnitemize}
Процесс привязки \textit{метода} решения \textit{задач} к конкретной \textit{задаче}, решаемой с помощью этого \textit{метода}, можно также представить как процесс, состоящий из следующих этапов:\begin{scnitemize}
\item построение копии используемого \textit{метода};\item склеивание основных (ключевых) переменных используемого \textit{метода} с основными параметрами конкретной решаемой \textit{задачи}.\end{scnitemize}
В результате этого на основе рассматриваемого \textit{метода} используемого в качестве образца (шаблона) строится спецификация процесса решения конкретной задачи -- процедурная спецификация (\textit{план}) или декларативная.}\scntext{note}{Заметим, что \textit{методы} могут использоваться даже при построении \textit{планов} решения конкретных \textit{задач}, в случае, когда возникает необходимость многократного повторения неких цепочек \textit{действий} при априори неизвестном количестве таких повторений. Речь идет о различного вида \textit{циклах}, которые являются простейшим видом процедурных \textit{методов} решения задач, многократно используемых (повторяемых) при реализации \textit{планов} решения некоторых \textit{задач}.}\scnidtf{программа}
\scnidtf{программа выполнения действий некоторого класса}
\scntext{note}{Одному \textit{классу действий} может соответствовать несколько \textit{методов} (программ).}\scnsuperset{программа в sc-памяти}
\scnsuperset{процедурная программа}
\scnidtf{обобщенный процедурный план}
\scnidtf{обобщенный процедурный план выполнения некоторого класса действий}
\scnidtf{обобщенный процедурный план решения некоторого класса задач}
\scnidtf{обобщенная спецификация декомпозиции любого действия, принадлежащего заданному классу действий}
\scnidtf{знание о некотором классе действий (и соответствующем классе задач), позволяющее для каждого из указанных действий достаточно легко построить процедурный план его выполнения}
\scnsubset{алгоритм}
\scntext{explanation}{Каждая \textit{процедурная программа} представляет собой обобщенный процедурный план выполнения \textit{действий}, принадлежащих некоторому классу, то есть \textit{семантическую окрестность, ключевым sc-элементом\scnrolesign} является \textit{класс действий}, для элементов которого дополнительно детализируется процесс их выполнения.В остальном описание \textit{процедурной программы} аналогично описанию \textit{плана} выполнения конкретного \textit{действия} из рассматриваемого \textit{класса действий}.Входным параметрам \textit{процедурной программы} в традиционном понимании соответствуют аргументы, соответствующие каждому \textit{действию} из \textit{класса действий}, описываемого \textit{процедурной программой}. При генерации на основе \textit{процедурной программы} \textit{плана} выполнения конкретного \textit{действия} из данного класса эти аргументы принимают конкретные значения.Каждая \textit{процедурная программа} представляет собой систему описанных действий с дополнительным указанием для действия:\begin{scnitemize}
\item либо \textit{последовательности выполнения действий*} (передачи инициирования), когда условием выполнения (инициирования) действий является завершение выполнения одного из указанных или всех указанных действий;\item либо события в базе знаний или внешней среде, являющегося условием его инициирования;\item либо ситуации в базе знаний или внешней среде, являющейся условием его инициирования;\end{scnitemize}
}\scnsuperset{программа sc-агента}
\scnidtf{метод, которому однозначно соответствует некоторыйsc-агент, способный выполнять действия, принадлежащие соответствующему данному методу классу действий}
\scntext{note}{Частным случаем метода является \textit{программа sc-агента}, в этом случае в качестве \textit{операционной семантики метода} выступает коллектив \textit{sc-агентов} более низкого уровня, интерпретирующий соответствующую программу (в  предельном случае это будут \textit{sc-агенты}, являющиеся частью \textit{платформы интерпретации sc-моделей компьютерных систем}, в том числе аппаратной).Таким образом:\begin{scnitemize}
\item можно говорить об иерархии \textit{методов} и о \textit{методах} интерпретации других \textit{методов};\item можно сказать, что понятие метода обобщает понятие \textit{программы sc-агента} для \textit{неатомарных sc-агентов}.\end{scnitemize}
}\scntext{note}{Отметим, что понятие \textit{метода} фактически позволяет локализовать область решения задач соответствующего класса, то есть ограничить множество знаний, которых достаточно для решения задач данного класса определенным способом. Это, в свою очередь, позволяет повысить эффективность работы системы в целом, исключая число лишних действий.}\scntext{note}{Каждый конкретный метод рассматривается нами как важный вид спецификации соответствующего класса задач, но также и как \textit{объект}, который и сам нуждается в спецификации, обеспечивающей непосредственное применение этого метода. Другими словами, метод является не только спецификацией (спецификацией соответствующего класса задач), но и \uline{объектом} спецификации.}\scnheader{следует отличать*}
\begin{scnhaselementset}
\scnitem{план сложного действия}
\scnitem{метод}
\end{scnhaselementset}
\scntext{note}{В отличие от \textit{метода}, \textit{план сложного действия} описывает не \textit{класс действий}, а конкретное \textit{действие}, возможно с учетом особенностей его выполнения в текущем контексте.}\scnheader{эквивалентность задач*}
\scnidtf{быть эквивалентной задачей*}
\scniselement{отношение}
\scntext{определение}{Задачи являются эквивалентными в том и только в том случае, если они могут быть решены путем интерпретации одного и того же \textit{метода} (способа), хранимого в памяти кибернетической системы.}
\scntext{note}{Некоторые \textit{задачи} могут быть решены разными \textit{методами}, один из которых, например, является обобщением другого.}\scnheader{отношение, заданное на множестве*(метод)}
\scnhaselement{подметод*}
\scnidtf{подпрограмма*}
\scnidtf{быть методом, использование которого (обращение к которому) предполагается при реализации заданного метода*}
\scnrelboth{следует отличать}{частный метод*}
\scnidtf{быть методом, обеспечивающим решение класса задач, который является подклассом задач, решаемых с помощью заданного метода*}
\scnheader{стратегия решения задач}
\scnsubset{метод}
\scnidtf{метаметод решения задач, обеспечивающий либо поиск одного релевантного известного метода, либо синтез целенаправленной последовательности акций применения в общем случае различных известных методов}
\scntext{note}{Можно говорить об универсальном метаметоде (универсальной стратегии) решения задач, объясняющем всевозможные частные стратегии.}\scntext{explanation}{Можно говорить о нескольких глобальных \textit{стратегиях решения информационных задач} в базах знаний. Пусть в базе знаний появился знак инициированного действия с формулировкой соответствующей информационной цели, т.е. цели, направленной только на изменение состояния базы знаний. И пусть текущее состояние базы знаний не содержит контекста (исходных данных), достаточного для достижения указанной выше цели, т.е такого контекста, для которого в доступном пакете (наборе) методов (программ) имеется метод (программа), использование которого позволяет достигнуть указанную выше цель. Для достижения такой цели, контекст (исходные данные) которой недостаточен, существует три подхода (три стратегии): \begin{scnitemize}
\item декомпозиция (сведение изначальной цели к иерархической системе и/или подцелей (и/или подзадач) на основе анализа текущего состояния базы знаний и анализа того, чего в базе знаний не хватает для использования того или иного метода.) При этом наибольшее внимание уделяется методам, для создания условий использования которых требуется меньше усилий. В конечном счете мы должны дойти (на самом нижнем уровне иерархии) до подцелей, контекст которых достаточен для применения одного из имеющихся методов (программ) решения задач;\item генерация новых знаний в семантической окрестности формулировки изначальной цели с помощью \uline{любых} доступных методов в надежде получить такое состояние базы знаний, которое будет содержать нужный контекст (достаточные исходные данные) для достижения изначальной цели с помощью какого-либо имеющегося метода решения задач;\item комбинация первого и второго подхода.\end{scnitemize}
Аналогичные стратегии существуют и для поиска пути решения задач, решаемых во внешней среде.}\scnheader{сужение отношения по первому домену(спецификация*, метод)}
\scnidtf{спецификация метода*}
\begin{scnsubdividing}
\scnitem{денотационная семантика метода*\\\scnidtf{обобщенная формулировка класса задач, решаемых с помощью данного метода*}
\scnrelboth{семантически близкий знак*}{обобщенная формулировка задач соответствующего класса*}
\scntext{note}{Данное отношение связывает обобщенную формулировку задач не с методом, а с классом задач}}
\scnitem{операционная семантика метода*\\\scnidtf{перечень обобщенных агентов, обеспечивающих интерпретацию метода*}
\scnidtf{семейство методов интерпретации данного метода*}
\scnidtf{формальное описание интерпретатора заданного метода*}
}
\end{scnsubdividing}
\end{scnsubstruct}
\end{SCn}
\begin{SCn}
\scnsegmentheader{Уточнение понятия навыка, понятия класса методов и понятия модели решения задач}
\begin{scnsubstruct}
\scniselement{сегмент базы знаний}
\scnheader{навык}
\scnidtf{умение}
\scnidtf{объединение \textit{метода} с его исчерпывающей спецификацией -- \textit{полным представлением операционной семантики метода}}
\scnidtf{метод, интерпретация (выполнение, использование) которого полностью может быть осуществлено данной кибернетической системой, в памяти которой указанный метод хранится}
\scnidtf{метод, который данная кибернетическая система умеет (может) применять}
\scnidtf{метод + метод его интерпретации}
\scnidtf{умение решать соответствующий класс эквивалентных задач}
\scnidtf{метод плюс его операционная семантика, описывающая то, как интерпретируется (выполняется, реализуется) этот метод, и являющаяся одновременно операционной семантикой соответствующей модели решения задач}
\begin{scnsubdividing}
\scnitem{активный навык\scnidtf{самоинициирующийся навык}
}
\scnitem{пассивный навык}
\end{scnsubdividing}
\scntext{explanation}{\textit{Навыки} могут быть \textit{пассивными навыками}, то есть такими \textit{навыками}, применение которых должно явно инициироваться каким-либо агентом, либо \textit{активными навыками}, которые инициируются самостоятельно при возникновении соответствующей ситуации в базе знаний. Для этого в состав \textit{активного навыка} помимо \textit{метода} и его операционной семантики включается также \textit{sc-агент}, который реагирует на появление соответствующей ситуации в базе знаний и инициирует интерпретацию \textit{метода} данного \textit{навыка}.Такое разделение позволяет реализовать и комбинировать различные подходы к решению задач, в частности, \textit{пассивные навыки} можно рассматривать в качестве способа реализации концепции интеллектуального пакета программ.}\scnheader{класс методов}
\scnrelto{семейство подклассов}{метод}
\scnidtf{множество методов, для которых можно \uline{унифицировать} представление (спецификацию) этих методов}
\scnidtf{множество всевозможных методов решения задач, имеющих общий язык представления этих методов}
\scnidtf{множество всевозможных методов, представленных на данном языке}
\scnidtf{множество методов, для которых задан язык представления этих методов}
\scnhaselement{процедурный метод решения задач}
\scnsuperset{алгоритмический метод решения задач}
\scnhaselement{логический метод решения задач}
\scnsuperset{продукционный метод решения задач}
\scnsuperset{функциональный метод решения задач}
\scnhaselement{искусственная нейронная сеть}
\scnidtf{класс методов решения задач на основе искусственных нейронных сетей}
\scnhaselement{генетический алгоритм}
\end{scnsubstruct}
\end{SCn}\bigskip
\end{scnsubstruct}
\end{SCn}

\scsubsubsection[
    \protect\scnmonographychapter{Глава 3.1. Формализация понятий действия, задачи, метода, средства, навыка и технологии}
    ]{Предметная область и онтология локальных предметных областей и онтологий действий}
\label{local_sd_actions}

\scsubsubsection[
    \protect\scnidtf{Типология неавтоматизированных ("вручную"{} выполняемых) и автоматически выполняемых \textit{действий}, направленных на управление процессами выполнения различных \textit{сложных действий}, а также система понятий, используемая для \textit{управления сложными действиями}}
    ]{Предметная область и онтология действий по управлению деятельностью многоагентных систем}
\label{local_sd_project_management}
