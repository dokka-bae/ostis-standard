\begin{SCn}
\scnsectionheader{\currentname}
\begin{scnsubstruct}
\begin{scnreltovector}{конкатенация сегментов}

\scnitem{Что такое онтология}
\scnitem{Типология онтологий предметной области}
\scnitem{Понятие объединенной онтологии предметной области, понятие предметной области и онтологии}
\scnitem{Отношения, заданные на множестве онтологий}

\end{scnreltovector}
\scnheader{Предметная область \textit{онтологий}}
\scnidtf{Предметная область теории \textit{онтологий}}
\scnidtf{Предметная область, объектами исследования которой являются \textit{онтологии}}
\scniselement{предметная область}
\begin{scnhaselementrole}{класс объектов исследования} онтология
\end{scnhaselementrole}
\begin{scnhaselementrolelist}{класс объектов исследования}

объединенная онтология;структурная спецификация предметной области;теоретико-множественная онтология предметной области;логическая онтология предметной области;логическая иерархия понятий предметной области;логическая иерархия высказываний предметной области;терминологическая онтология предметной области

\end{scnhaselementrolelist}
\begin{scnhaselementrolelist}{исследуемое отношение}

онтология*;используемые константы*;используемые утверждения*

\end{scnhaselementrolelist}
\scnsegmentheader{Что такое онтология}
\begin{scnsubstruct}
\scnheader{онтология}
\scnidtf{sc-онтология}
\scnidtf{онтология, представленная в SC-коде}
\scntext{note}{Поскольку термин ``\textit{онтология} в SC-коде соответствует множеству всевозможных онтологий, представленных в SC-коде, то для формальных онтологий, представленных на других языках, необходимо использовать sc-идентификатор, содержащий явное указание этих языков, например, ``owl-онтология.}\scnidtf{sc-текст онтологии}
\scnidtf{sc-модель онтологии}
\scnidtf{семантическая спецификация \textit{знаний}}
\scnidtf{семантическая спецификация любого знания, имеющего достаточно сложную структуру, любого целостного фрагмента базы знаний -- предметной области, метода решения сложных задач некоторого класса, описания истории некоторого вида деятельности, описания области выполнения некоторого множества действий (области решения задач), языка  представления методов решения задач и т.д.}
\scnidtftext{explanation}{\uline{семантическая} \textit{спецификация} некоторого достаточно информативного ресурса (\textit{знания})}
\scnsubset{спецификация}
\scntext{note}{Если \textit{спецификация} может специфицировать (описывать) любую \textit{сущность}, то \textit{отнология} специфицирует только различные \textit{знания}. При этом наиболее важными объектами такой спецификации являются \textit{предметные области}}\scnsubset{метазнание}
\scniselement{вид знаний}
\scntext{note}{\textit{онтологии} являются важнейшим \textit{видом знаний} (точнее, метазнаний), обеспечивающих семантическую систематизацию \textit{знаний}, хранимых в памяти \textit{интеллектуальных компьютерных систем} (в т.ч. \textit{ostis-систем}), и, соответственно, семантическую структуризацию \textit{баз знаний}}\scnidtf{важнейший вид \textit{метазнаний}, входящих в состав базы знаний}
\scnidtf{спецификация (уточнение) системы \textit{понятий}, используемых в соответствующем (специфицируемом) \textit{знании}}
\scntext{эпиграф}{Определив точно значения слов, вы избавите человечество от половины заблуждений}
\scnrelfrom{автор}{Рене Декарт}
\scntext{explanation}{\textit{онтология} включает в себя: \begin{scnitemize}
\item типологию специфицируемого \textit{знания};\item связи специфицируемого \textit{знания} с другими \textit{знаниями};\item спецификацию ключевых \textit{понятий}, используемых в специфицируемом \textit{знании}, а также ключевых экземпляров некоторых таких \textit{понятий}.\end{scnitemize}
}\scntext{explanation}{Основная \textit{цель} построения \textit{онтологии} -- семантическое уточнение (пояснение, а в идеале -- определение) такого семейства \textit{знаков}, используемых в заданном \textit{знании}, которых достаточно для понимания смысла всего специфицируемого \textit{знания}. Как выясняется, количество \textit{знаков}, смысл которых определяет смысл всего специфицируемого \textit{знания}, \uline{не является большим}.}\begin{scnsubdividing}

\scnitem{неформальная онтология}
\scnitem{формальная онтология}
\scnidtf{онтология, представленная на формальном языке}
\scnrelto{ключевой знак}{\cite{Loukashevich2011}}


\end{scnsubdividing}
\scnheader{формальная онтология}
\scnidtf{формальное описание \uline{денотационной семантики} (семантической интерпретации) специфицируемого знания}
\scntext{note}{Очевидно, что при отсутствии достаточно полных формальных онтологий невозможно обеспечить семантическую совместимость (интегрируемость) различных знаний, хранимых в базе знаний, а также приобретаемых извне.}\scnheader{онтология предметной области}
\scntext{note}{\textit{онтология} чаще всего трактуется как спецификация концептуализации (спецификация системы \textit{понятий}) заданной \textit{предметной области}. Здесь имеется в виду описание теоретико-множественных связей (прежде всего, классификации) используемых \textit{понятий}, а также описание различных закономерностей для сущностей, принадлежащих этим \textit{понятиям}. Тем не менее, важными видами спецификации \textit{предметной области} являются также: \begin{scnitemize}
\item описание связей специфицируемой \textit{предметной области} с другими \textit{предметными областями};\item описание терминологии специфицируемой \textit{предметной области}.\end{scnitemize}
}\scntext{note}{\textit{онтологию предметной области} можно трактовать, с одной стороны, как \textit{семантическую окрестность} соответствующей \textit{предметной области}, с другой стороны, как \textit{объединение} определённого вида \textit{семантических окрестностей} всех \textit{понятий}, используемых в рамках указанной \textit{предметной области}, а также, возможно, ключевых экземпляров указанных \textit{понятий}, если таковые экземпляры имеются}\scntext{explanation}{Каждая конкретная онтология заданного вида представляет собой семантическую окрестность соответствующей (специфицируемой) предметной области.Каждому \textit{виду онтологий} однозначно соответствует \textit{предметная область}, фрагментами которые являются конкретные \textit{онтологии} этого вида. Следовательно, каждому \textit{виду онтологий} соответствует свой специализированный sc-язык, обеспечивающий представление \textit{онтологий} этого вида.}\scnidtf{описание \textit{денотационной семантики} языка, определяемого (задаваемого) соответствующей (специфицируемой) \textit{предметной области}}
\scnidtf{информационная надстройка (метаинформация) над соответствующей (специфицируемой) \textit{предметной областью}, описывающая различные аспекты этой \textit{предметной области} как достаточно крупного, самодостаточного и семантически целостного фрагмента \textit{база знаний}}
\scnidtf{метаинформация (метазнание) о некоторой \textit{предметной области}}
\bigskip
\end{scnsubstruct}
\end{scnsubstruct}
\end{SCn}
