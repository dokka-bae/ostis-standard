\begin{SCn}
\scnsegmentheader{Уточнение понятия навыка, понятия класса методов и понятия модели решения задач}
\begin{scnsubstruct}
\scniselement{сегмент базы знаний}
\scnheader{навык}
\scnidtf{умение}
\scnidtf{объединение \textit{метода} с его исчерпывающей спецификацией -- \textit{полным представлением операционной семантики метода}}
\scnidtf{метод, интерпретация (выполнение, использование) которого полностью может быть осуществлено данной кибернетической системой, в памяти которой указанный метод хранится}
\scnidtf{метод, который данная кибернетическая система умеет (может) применять}
\scnidtf{метод + метод его интерпретации}
\scnidtf{умение решать соответствующий класс эквивалентных задач}
\scnidtf{метод плюс его операционная семантика, описывающая то, как интерпретируется (выполняется, реализуется) этот метод, и являющаяся одновременно операционной семантикой соответствующей модели решения задач}
\begin{scnsubdividing}
\scnitem{активный навык\scnidtf{самоинициирующийся навык}
}
\scnitem{пассивный навык}
\end{scnsubdividing}
\scntext{explanation}{\textit{Навыки} могут быть \textit{пассивными навыками}, то есть такими \textit{навыками}, применение которых должно явно инициироваться каким-либо агентом, либо \textit{активными навыками}, которые инициируются самостоятельно при возникновении соответствующей ситуации в базе знаний. Для этого в состав \textit{активного навыка} помимо \textit{метода} и его операционной семантики включается также \textit{sc-агент}, который реагирует на появление соответствующей ситуации в базе знаний и инициирует интерпретацию \textit{метода} данного \textit{навыка}.Такое разделение позволяет реализовать и комбинировать различные подходы к решению задач, в частности, \textit{пассивные навыки} можно рассматривать в качестве способа реализации концепции интеллектуального пакета программ.}\scnheader{класс методов}
\scnrelto{семейство подклассов}{метод}
\scnidtf{множество методов, для которых можно \uline{унифицировать} представление (спецификацию) этих методов}
\scnidtf{множество всевозможных методов решения задач, имеющих общий язык представления этих методов}
\scnidtf{множество всевозможных методов, представленных на данном языке}
\scnidtf{множество методов, для которых задан язык представления этих методов}
\scnhaselement{процедурный метод решения задач}
\scnsuperset{алгоритмический метод решения задач}
\scnhaselement{логический метод решения задач}
\scnsuperset{продукционный метод решения задач}
\scnsuperset{функциональный метод решения задач}
\scnhaselement{искусственная нейронная сеть}
\scnidtf{класс методов решения задач на основе искусственных нейронных сетей}
\scnhaselement{генетический алгоритм}
\end{scnsubstruct}
\end{SCn}