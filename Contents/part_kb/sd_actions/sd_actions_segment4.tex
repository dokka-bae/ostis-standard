\scnsegmentheader{Предметная область и онтология субъектно-объектных спецификаций воздействий}
\begin{scnsubstruct}
\begin{scnrelfromlist}{соавтор}
\scnitem{Гордей А.Н.}
\scnitem{Никифоров С.А.}
\scnitem{Бобёр Е.С.}
\scnitem{Святощик М.И.}
\end{scnrelfromlist}
\scniselement{предметная область и онтология}
\scnheader{индивид}
\scnidtftext{часто используемый sc-идентификатор}{субъект}
\scnrelfrom{источник}{\cite{Hardzei2005}}
\scnheader{участник воздействия\scnrolesign}
\scnidtf{участник акции\scnrolesign}
\scniselement{ролевое отношение}
\scnrelfrom{первый домен}{индивид}
\scnrelfrom{второй домен}{воздействие}
\scntext{explanation}{\textit{участник акции\scnrolesign} -- это ролевое отношение, которое связывает акцию с участвующим в ней индивидом.}\scnrelfrom{источник}{\cite{Hardzei2021}}
\scnrelfrom{источник}{\cite{Fillmore1977}}
\scnrelfrom{источник}{\cite{Fillmore1982}}
\begin{scnsubdividing}
\scnitem{субъект\scnrolesign\\\scntext{explanation}{\textit{субъект\scnrolesign} -- инициатор акции.}\begin{scnsubdividing}
\scnitem{инициатор\scnrolesign\\}
\scnitem{вдохновитель\scnrolesign\\}
\scnitem{распространитель\scnrolesign\\}
\scnitem{вершитель\scnrolesign\\\scntext{explanation}{\textit{вершитель\scnrolesign} завершает акцию производством из объекта продукта.}}
\end{scnsubdividing}
}
\scnitem{инструмент\scnrolesign\\\scntext{explanation}{\textit{инструмент\scnrolesign} -- исполнитель акции.}\begin{scnsubdividing}
\scnitem{активатор\scnrolesign\\\scntext{explanation}{\textit{активатор\scnrolesign} непосредственно воздействует на медиатор.}}
\scnitem{супрессор\scnrolesign\\\scntext{explanation}{\textit{супрессор\scnrolesign} подавляет сопротивление медиатора.}}
\scnitem{усилитель\scnrolesign\\\scntext{explanation}{\textit{усилитель\scnrolesign} наращивает воздействие на медиатор.}}
\scnitem{преобразователь\scnrolesign\\\scntext{explanation}{\textit{преобразователь\scnrolesign} преобразует медиатор в инструмент.}}
\end{scnsubdividing}
}
\scnitem{медиатор\scnrolesign\\\scntext{explanation}{\textit{медиатор\scnrolesign} -- посредник акции.}\begin{scnsubdividing}
\scnitem{ориентир\scnrolesign\\}
\scnitem{локус\scnrolesign\\\scntext{explanation}{\textit{локус\scnrolesign} частично или полностью окружает объект и тем самым локализует его в пространстве.}}
\scnitem{транспортёр\scnrolesign\\\scntext{explanation}{\textit{транспортёр\scnrolesign} перемещает объект.}}
\scnitem{адаптер\scnrolesign\\\scntext{explanation}{\textit{адаптер\scnrolesign} приспосабливает  инструмент к воздействию на объект.}}
\scnitem{материал\scnrolesign\\\scntext{explanation}{\textit{материал\scnrolesign} используется в качестве объекта-сырья для производства продукта.}}
\scnitem{макет\scnrolesign\\\scntext{explanation}{\textit{макет\scnrolesign} является исходным образцом для производства из объекта продукта.}}
\scnitem{фиксатор\scnrolesign\\\scntext{explanation}{\textit{фиксатор\scnrolesign} превращает переменный локус объекта в постоянный.}}
\scnitem{ресурс\scnrolesign\\\scntext{explanation}{\textit{ресурс\scnrolesign} питает инструмент.}}
\scnitem{стимул\scnrolesign\\\scntext{explanation}{\textit{стимул\scnrolesign} проявляет параметр объекта.}}
\scnitem{регулятор\scnrolesign\\\scntext{explanation}{\textit{регулятор\scnrolesign} служит инструкцией в производстве из объекта продукта.}}
\scnitem{хронотоп\scnrolesign\\\scntext{explanation}{\textit{хронотоп\scnrolesign} локализует объект во времени.}}
\scnitem{источник\scnrolesign\\\scntext{explanation}{\textit{источник\scnrolesign} обеспечивает инструкциями инструмент.}}
\scnitem{индикатор\scnrolesign\\\scntext{explanation}{\textit{индикатор\scnrolesign} отображает параметр воздействия на объект или параметр продукта как результата воздействия на объект.}}
\end{scnsubdividing}
}
\scnitem{объект\scnrolesign\\\scntext{explanation}{\textit{объект\scnrolesign} -- реципиент акции.}\begin{scnsubdividing}
\scnitem{покрытие\scnrolesign\\\scntext{explanation}{\textit{покрытие\scnrolesign} -- внешняя изоляция оболочки индивида.}}
\scnitem{корпус\scnrolesign\\\scntext{explanation}{\textit{корпус\scnrolesign} -- оболочка индивида.}}
\scnitem{прослойка\scnrolesign\\\scntext{explanation}{\textit{прослойка\scnrolesign} -- внутренняя изоляция оболочки индивида.}}
\scnitem{сердцевина\scnrolesign\\\scntext{explanation}{\textit{сердцевина\scnrolesign} -- ядро индивида.}}
\end{scnsubdividing}
}
\scnitem{продукт\scnrolesign\\\scntext{explanation}{\textit{продукт\scnrolesign} -- результат воздействия субъекта на объект.}\begin{scnsubdividing}
\scnitem{заготовка\scnrolesign\\\scntext{explanation}{\textit{заготовка\scnrolesign} -- превращённый в сырьё объект.}}
\scnitem{полуфабрикат\scnrolesign\\\scntext{explanation}{\textit{полуфабрикат\scnrolesign} -- наполовину изготовленный из сырья продукт.}}
\scnitem{прототип\scnrolesign\\\scntext{explanation}{\textit{прототип\scnrolesign} -- опытный образец продукта.}}
\scnitem{изделие\scnrolesign\\\scntext{explanation}{\textit{изделие\scnrolesign} -- готовый продукт.}}
\end{scnsubdividing}
}
\end{scnsubdividing}
\scnheader{воздействие}
\scnidtf{акция}
\scnrelfrom{источник}{\cite{Hardzei2017}}
\scnrelfrom{разбиение}{\scnkeyword{Типология по характеру взаимодействия участников\scnsupergroupsign}
}
\begin{scneqtoset}
\scnitem{воздействие активизации\\\scntext{explanation}{\textit{воздействие активизации} -- воздействие, в ходе которого взаимодействие осуществляется между субъектом и инструментом.}}
\scnitem{воздействие эксплуатации\\\scntext{explanation}{\textit{воздействие эксплуатации} -- воздействие, в ходе которого взаимодействие осуществляется между инструментом и медиатором.}}
\scnitem{воздействие трансформации\\\scntext{explanation}{\textit{воздействие трансформации} -- воздействие, в ходе которого взаимодействие осуществляется между объектом и продуктом.}}
\scnitem{воздействие нормализации\\\scntext{explanation}{\textit{воздействие нормализации} -- воздействие, в ходе которого взаимодействие осуществляется между объектом и продуктом.}}
\end{scneqtoset}
\scnrelfrom{разбиение}{\scnkeyword{Типология воздействий по виду взаимодействующих подсистем\scnsupergroupsign}
}
\begin{scneqtoset}
\scnitem{воздействие среда-оболочка\\}
\scnitem{воздействие оболочка-ядро\\}
\scnitem{воздействие ядро-оболочка\\}
\scnitem{воздействие оболочка-среда}
\end{scneqtoset}
\scnheader{воздействие}
\scnrelfrom{разбиение}{\scnkeyword{Типология воздействий по фазам наращивания воздействия\scnsupergroupsign}
}
\begin{scneqtoset}
\scnitem{воздействие инициации\\\scntext{explanation}{\textit{воздействие инициации} -- воздействие, в ходе которого оно начинается в каждой подсистеме.}}
\scnitem{воздействие аккумуляции\\\scntext{explanation}{\textit{воздействие аккумуляции} -- воздействие, в ходе которого происходит его накапливание в каждой подсистеме.}}
\scnitem{воздействие амплификации\\\scntext{explanation}{\textit{воздействие амплификации} -- воздействие, в ходе которого происходит его усиление.}}
\scnitem{воздействие генерации\\\scntext{explanation}{\textit{воздействие генерации} -- воздействие, которое представляет собой переход в каждой подсистеме с одного уровня, например, среды-оболочки, на другой, например, оболочки-ядра.}}
\end{scneqtoset}
\scnheader{воздействие}
\scnrelfrom{разбиение}{\scnkeyword{Типология воздействий по виду инструмента\scnsupergroupsign}
}
\begin{scneqtoset}
\scnitem{физическое воздействие\\\scntext{explanation}{\textit{физическое воздействие} -- воздействие, в котором в роли инструмента выступает оболочка субъекта.}}
\scnitem{информационное воздействие\\\scntext{explanation}{\textit{информационное воздействие} -- воздействие, в котором в роли инструмента выступает среда субъекта.}}
\end{scneqtoset}
\scnheader{Рис. Таблица воздействий}
\scneqfile{\\\includegraphics{figures/sd_actions/macroproc_table.png}\\}
\scnrelfrom{источник}{\cite{Hardzei2017}}
\scntext{explanation}{На изображении представлена типология \textit{воздействий}. Любое воздействие характеризуется принадлежностью четырём классам, соответствующим признакам классификации. Заштрихованы \textit{физические воздействия}.}\scnheader{Специфицируемые классы воздействий}
\scnsuperset{\begin{scnset}
формование\\\begin{scnreltoset}{пересечение}
\scnitem{воздействие трансформации}
\scnitem{воздействие среда-оболочка}
\scnitem{воздействие генерации}
\scnitem{физическое воздействие}
\end{scnreltoset}
;притягивание\\\begin{scnreltoset}{пересечение}
\scnitem{воздействие активизации}
\scnitem{воздействие среда-оболочка}
\scnitem{воздействие инициации}
\scnitem{физическое воздействие}
\end{scnreltoset}
;выхолащивание\\\begin{scnreltoset}{пересечение}
\scnitem{воздействие трансформации}
\scnitem{воздействие ядро-оболочка}
\scnitem{воздействие генерации}
\scnitem{физическое воздействие}
\end{scnreltoset}
;аннигилирование\\\begin{scnreltoset}{пересечение}
\scnitem{воздействие трансформации}
\scnitem{воздействие оболочка-среда}
\scnitem{воздействие генерации}
\scnitem{физическое воздействие}
\end{scnreltoset}
;введение\\\begin{scnreltoset}{пересечение}
\scnitem{воздействие эксплуатации}
\scnitem{воздействие оболочка-ядро}
\scnitem{воздействие инициации}
\scnitem{физическое воздействие}
\end{scnreltoset}
;распускание\\\begin{scnreltoset}{пересечение}
\scnitem{воздействие трансформации}
\scnitem{воздействие оболочка-среда}
\scnitem{воздействие амплификации}
\scnitem{физическое воздействие}
\end{scnreltoset}
;разжимание\\\begin{scnreltoset}{пересечение}
\scnitem{воздействие трансформации}
\scnitem{воздействие ядро-оболочка}
\scnitem{воздействие амплификации}
\scnitem{физическое воздействие}
\end{scnreltoset}
;разъединение\\\begin{scnreltoset}{пересечение}
\scnitem{воздействие эксплуатации}
\scnitem{воздействие ядро-оболочка}
\scnitem{воздействие генерации}
\scnitem{физическое воздействие}
\end{scnreltoset}

\end{scnset}
}
\scnheader{Пример sc.g-текста, описывающего спецификацию воздействия}
\scneq{\scnfileimage[20em]{figures/sd_actions/tapaz_description_example.png}}
\scniselement{sc.g-текст}
\scntext{explanation}{Представленный фрагмент базы знаний содержит декомпозицию воздействия во времени, указание принадлежности данного декомпозируемого воздействия и полученных в результате данной декомпозиции воздействий определенному их классу из приведенной выше классификации, а также указание участников данных акций.}\scntext{explanation}{Представленный фрагмент базы знаний можно протранслировать в следующий текст естественного языка: <<Некто принимает молоко, затем окисляет молоко, а именно: нормализует молоко до 15-процентной жирности, затем очищает молоко, затем пастеризует молоко, затем охлаждает молоко до определённой температуры, затем вносит закваску в молоко, затем сквашивает молоко, затем режет сгусток, затем подогревает сгусток, затем обрабатывает сгусток, затем отделяет сыворотку, затем охлаждает сгусток и, в итоге, производит творог>>.}\scnheader{субъект}
\scnidtftext{часто используемый sc-идентификатор}{индивид}
\scnidtf{активная сущность}
\scnidtf{сущность, способная самостоятельно выполнять некоторые виды действий}
\scnidtf{агент деятельности}
\scnsuperset{Собственное Я}
\scnsuperset{внутренний субъект ostis-системы}
\scnsuperset{внешний субъект ostis-системы, с которым осуществляется взаимодействие}
\scnsuperset{внешний субъект ostis-системы, с которым взаимодействие не происходит}
\scnheader{внутренний субъект ostis-системы}
\scnidtf{субъект, входящий в состав той \textit{ostis-системы, в базе знаний} которой он описывается}
\scnsuperset{sc-агент}
\scntext{explanation}{Под \textit{внутренним субъектом ostis-системы} понимается такой \textit{субъект}, который выполняет некоторые \textit{действия} в \uline{той же памяти}, в которой хранится его знак.\newlineК числу \textit{внутренних субъектов ostis-системы} относятся входящие в нее \textit{sc-агенты}, частные sc-машины, целые интеллектуальные подсистемы.}\scnheader{внешний субъект ostis-системы, с которым осуществляется взаимодействие}
\scntext{explanation}{К числу \textit{внешних субъектов ostis-системы, с которыми осуществляется взаимодействие}, относятся конечные пользователи \textit{ostis-системы}, ее разработчики, а также другие компьютерные системы(причем не только интеллектуальные).}\scnheader{субъект действия\scnrolesign}
\scnsubset{субъект\scnrolesign}
\scnidtf{сущность, воздействующая на некоторую другую сущность в процессе заданного действия\scnrolesign}
\scnidtf{сущность, создающая \textit{причину} изменений другой сущности (объекта действия)\scnrolesign}
\scnidtf{быть субъектом данного действия\scnrolesign}
\scnsuperset{субъект неосознанного воздействия\scnrolesign}
\scnsuperset{субъект осознанного воздействия\scnrolesign}
\scnidtf{субъект целенаправленного, активного воздействия\scnrolesign}
\scnheader{исполнитель*}
\scntext{explanation}{Связки отношения \textit{исполнитель*} связывают \textit{sc-элементы}, обозначающие \textit{действие} и \textit{sc-элементы}, обозначающие \textit{субъекта}, который предположительно будет осуществлять, осуществляет или осуществлял выполнение указанного \textit{действия}. Данное отношение может быть использовано при назначении конкретного исполнителя для проектной задачи по развитию баз знаний.В случае, когда заранее неизвестно, какой именно \textit{субъект*} будет исполнителем данного \textit{действия}, отношение \textit{исполнитель*} может отсутствовать в первоначальной формулировке \textit{задачи} и добавляться позже, уже непосредственно при исполнении.Когда действие выполняется (является \textit{настоящей сущностью}) или уже выполнено (является \textit{прошлой сущностью}), то исполнитель этого действия в каждый момент времени уже определён. Но когда действие только инициировано, тогда важно знать:\begin{enumerate}
\item кто \uline{хочет} выполнить это действие и насколько важно для него стать исполнителем данного действия;\item кто \uline{может} выполнить данное действие и каков уровень его квалификации и опыта;\item кто и кому поручает выполнить это действие и каков уровень ответственности за невыполнение (приказ, заказ, официальный договор, просьба...)\end{enumerate}
При этом следует помнить, что связь отношения \textit{исполнитель*} в данном случае также является временной прогнозируемой сущностью.Первым компонентом связок отношений \textit{исполнитель*} является знак \textit{действия}, вторым -- знак \textit{субъекта} исполнителя}\scnheader{объект воздействия\scnrolesign}
\scnsubset{объект\scnrolesign}
\scnidtf{сущность, на которую осуществляется воздействие в рамках заданного действия\scnrolesign}
\scnidtf{сущность, являющаяся в рамках заданного действия исходным условием (аргументом), необходимым для выполнения этого действия\scnrolesign}
\scntext{note}{Для разных действий количество объектов действий может быть различным.}\scntext{note}{Поскольку действие является процессом и, соответственно, представляет собой \textit{динамическую структуру}, то и знак \textit{субъекта действия\scnrolesign}, и знак \textit{объекта действия\scnrolesign} являются элементами данной структуры. В связи с этим можно рассматривать отношения \textit{субъект действия\scnrolesign} и \textit{объект действия\scnrolesign} как \textit{ролевые отношения}. Данный факт не  запрещает вводить аналогичные \textit{неролевые отношения}, однако это нецелесообразно.}\scnheader{продукт\scnrolesign}
\scnidtf{быть продуктом заданного действия\scnrolesign}
\scnsubset{продукт*}
\scnsubset{результат*}
\scnidtf{сухой}
\end{scnsubstruct}