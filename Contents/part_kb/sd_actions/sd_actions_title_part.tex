\begin{SCn}
\scniselement{раздел}
\scniselement{предметная область и онтология}
\begin{scnreltovector}{конкатенация сегментов}
\scnitem{Уточнение понятия воздействия и понятия действия. Типология воздействий и действий}
\scnitem{Уточнение понятия задачи. Типология задач}
\scnitem{Уточнение семейства параметров и отношений, заданных на множестве воздействий, действий и задач}
\scnitem{Предметная область и онтология субъектно-объектных спецификаций воздействий}
\scnitem{Уточнение понятий плана сложного действия, класса задач, метода}
\scnitem{ Уточнение понятия навыка, понятия класса методов и понятия модели решения задач}
\scnitem{Уточнение понятия деятельности, понятия вида деятельности и понятия технологии}
\end{scnreltovector}
\begin{scnhaselementrolelist}
\scnitem{исследуемый класс первичных объектов исследования}
\end{scnhaselementrolelist}
\bigskip\begin{scnhaselementrolelist}
\scnitem{исследуемый класс классов первичных объектов исследования}
\end{scnhaselementrolelist}
\begin{scnhaselementrolelist}
\scnitem{исследуемый класс классов}
\end{scnhaselementrolelist}
\scnsourcecommentpar{Здесь указаны классы классов, которые не являются классами классов \uline{первичных} объектов исследования}\bigskip\begin{scnhaselementrolelist}
\scnitem{исследуемое отношение, заданное на множестве первичных объектов исследования}
\end{scnhaselementrolelist}
\bigskip\begin{scnhaselementrolelist}
\scnitem{исследуемое отношение}
\end{scnhaselementrolelist}
\scnsourcecommentpar{Здесь указаны исследуемые отношения, которые заданы не на множестве первичных объектов исследования}\bigskip\begin{scnhaselementrolelist}
\scnitem{исследуемый класс структур, специфицирующих первичные объекты исследования}
\end{scnhaselementrolelist}
\bigskip\begin{scnhaselementrolelist}
\scnitem{исследуемый класс структур}
\end{scnhaselementrolelist}
\scnsourcecommentpar{Здесь указаны классы структур, не являющихся спецификациями первичных объектов исследования}\bigskip\begin{scnhaselementrolelist}
\scnitem{вводимое, но не исследуемое понятие}
\end{scnhaselementrolelist}
\scnsourcecommentpar{Здесь указаны понятия, исследуемые в предметной области (и соответствующей онтологии), которая является \uline{частной} по отношению к заданной и которая наследует все свойства заданной предметной области и онтологии}\bigskip\begin{scnhaselementrolelist}
\scnitem{используемое понятие, исследуемое в другой предметной области и онтологии}
\end{scnhaselementrolelist}
\scnheader{следует отличать*}
\begin{itemize}
    \item Класс действий
    \item Класс методов
    \item Вид деятельности
\end{itemize}
\scnsubset{семейство подклассов*}
\scntext{note}{Все сущности, принадлежащие рассмотренным \textit{понятиям}, требуют достаточно детальной \textit{спецификации}. При этом не следует путать сами сущности и их \textit{спецификации}. Так, например, не следует путать \textit{действие} и \textit{задачу}, которая специфицирует (уточняет) это \textit{действие}. Особое место среди указанных понятий занимает понятие \textit{метода}, т.к. каждый конкретный \textit{метод}, с одной стороны, является \textit{спецификацией} соответствующего \textit{класса действий}, а, с другой стороны, сам нуждается в \textit{спецификации}, которая уточняет либо \textit{декларативную семантику} этого \textit{метода} (т.е. обобщенную декларативную формулировку класса задач, решаемых с помощью этого \textit{метода}), либо \textit{операционную семантику} этого \textit{метода}, (т.е. множество \textit{методов}, обеспечивающих \textit{интерпретацию} данного специфицируемого \textit{метода}) и тем самым преобразует специфицируемый \textit{метод} в \textit{навык}.}\scnheader{следует отличать*}
\begin{scnhaselementvector}
\scnitem{первый домен*(спецификация*)\\\scnidtf{специфицируемая сущность}
\scnidtf{сущность, использование которой требует вполне определенной ее спецификации}
\scnsuperset{действие}
\scnsuperset{класс действий}
\scnsuperset{метод}
\scnsuperset{класс методов}
\scnsuperset{деятельность}
\scnsuperset{вид деятельности}
}
\scnitem{второй домен*(спецификация*)\\\scnidtf{спецификация}
\scnsuperset{задача}
\scnsuperset{декларативная формулировка задачи}
\scnidtf{семантическая формулировка задачи}
\scnsuperset{процедурная формулировка задачи}
\scnidtf{функциональная формулировка задачи}
\scnsuperset{план действия}
\scnidtf{план}
\scnidtf{план выполнения действия}
\scnsuperset{декларативная спецификация выполнения действий}
\scnidtf{иерархическая система подзадач}
\scnsuperset{протокол}
\scnsuperset{результативная часть протокола}
\scnsuperset{обобщенная декларативная формулировка класса задач}
\scnsuperset{метод}
\scnsuperset{декларативная семантика метода}
\scnsuperset{операционная семантика метода}
\scnsuperset{модель решения задач}
}
\end{scnhaselementvector}
\scntext{note}{При этом следует отличать:\begin{scnitemize}
\item спецификацию конкретного \textit{действия} (\textit{задачу}, \textit{план}, \textit{декларативную спецификацию выполнения действия}, \textit{протокол}, \textit{результативную часть протокола});\item спецификацию конкретной \textit{деятельности} (\textit{контекст}*, \textit{множество используемых методов}*);\item спецификацию \textit{класса действий} (\textit{обобщенную декларативную формулировку класса задач}, \textit{метод});\item спецификацию \textit{вида деятельности} (\textit{технологию});\item спецификацию \textit{метода} (\textit{декларативную семантику метода}, \textit{операционную семантику метода});\item спецификацию \textit{класса методов} (\textit{модель решения задач}).\end{scnitemize}
}\scnheader{следует отличать*}

\bigskip\begin{scnrelfromlist}{библиографический источник}
\scnitem{\cite{Martynov1984}\\\scnciteannotation{Martynov1984}}
\scnitem{\cite{Ikeda1998}\\\scnciteannotation{Ikeda1998}\scnrelfrom{ключевой знак}{онтология классов задач}
\scnidtf{задачная онтология}
\scnidtf{онтология классов задач, решаемых в данной предметной области}
}
\scnitem{\cite{Studer1996}\\\scnciteannotation{Studer1996}}
\scnitem{\cite{Benjamins1999}\\\scnciteannotation{Benjamins1999}}
\scnitem{\cite{Chandrasekaran1999}\\\scnciteannotation{Chandrasekaran1999}}
\scnitem{\cite{Chandrasekaran1998}\\\scnciteannotation{Chandrasekaran1998}}
\scnitem{\cite{Fensel1998Reuse}\\\scnciteannotation{Fensel1998Reuse}}
\scnitem{\cite{Kemke2001}\\\scnciteannotation{Kemke2001}}
\scnitem{\cite{Tu1995}\\\scnciteannotation{Tu1995}}
\scnitem{\cite{Trypuz2007}\\\scnciteannotation{Trypuz2007}}
\scnitem{\cite{Fang2019}\\\scnciteannotation{Fang2019}}
\scnitem{\cite{Fensel1997}}
\scnitem{\cite{McBride2021}\\\scnciteannotation{McBride2021}}
\scnitem{\cite{Crowther2020}\\\scnciteannotation{Crowther2020}}
\scnitem{\cite{McCann1998}\\\scnciteannotation{McCann1998}}
\scnitem{\cite{Yan2014}\\\scnciteannotation{Yan2014}}
\scnitem{\cite{Ansari2018}}
\scnitem{\cite{Crubezy2004}\\\scnciteannotation{Crubezy2004}}
\scnitem{\cite{Coelho1996}}
\end{scnrelfromlist}
\end{SCn}
