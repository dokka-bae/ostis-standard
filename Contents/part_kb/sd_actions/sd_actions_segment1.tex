\begin{SCn}
\scnsegmentheader{Уточнение понятия воздействия и понятия действия. Типология воздействий и действий}
\begin{scnsubstruct}
\scniselement{сегмент базы знаний}
\scnheader{воздействие}
\scnidtf{\textit{процесс} воздействия одной сущности (или некоторого множества \textit{сущностей}) на другую \textit{сущность} (или на некоторое множество других \textit{сущностей})}
\scnidtf{\textit{процесс}, в котором могут быть явно выделены хотя бы одна воздействующая сущность (\textit{субъект воздействия\scnrolesign}) и хотя бы одна \textit{сущность}, на которую осуществляется воздействие (\textit{субъект воздействия\scnrolesign})}
\scnheader{воздействие}
\scnsubset{процесс}
\scnidtf{динамическая структура}
\scntext{note}{Поскольку \textit{воздействия} являются частным видом \textit{процессов}, воздействиями наследуются все свойства \textit{процессов}. Смотрите Раздел \textit{Предметная область и онтология структур}). В частности, используются все \textit{параметры}, заданные на множестве \textit{процессов}, например, \textit{длительность*}, \textit{момент начала процесса*}, \textit{момент завершения процесса\scnsupergroupsign}}\scnheader{процесс}
\scnrelfrom{покрытие}{длительность\scnsupergroupsign\\\begin{scneqtoset}
\scnitem{краткосрочный процесс}
\scnitem{среднесрочный процесс}
\scnitem{долгосрочный процесс}
\scnitem{ перманентный процесс}
\end{scneqtoset}
\scntext{note}{Длительность различных процессов можно уточнять до любой необходимой точности, используя различные единицы измерения длительности (с точностью до секунд, минут, часов, дней, месяцев, лет, столетий и т.д.). Кроме того, можно ссылаться на процессы, длительность (время жизни) которых соизмерима с рассматриваемым процессом.}}
\scnheader{действие}
\scnsubset{воздействие}
\scnsubset{процесс}
\scnidtf{\textit{воздействие}, в котором \textit{воздействующая сущность\scnrolesign} осуществляет \textit{воздействие} осознанно}
\end{scnsubstruct}
\end{SCn}