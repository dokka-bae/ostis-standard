\begin{SCn}
\scnsectionheader{\currentname}
\begin{scnsubstruct}
\scnheader{Предметная область чисел и числовых структур}
\scniselement{предметная область}
\begin{scnhaselementrole}{класс объектов исследования}
число
\end{scnhaselementrole}
\begin{scnhaselementrolelist}{класс объектов исследования}

    натуральное число;целое число;рациональное число;иррациональное число;действительное число;комплексное число;отрицательное число;положительное число;арифметическое выражение;арифметическая операция;Число Пи;Нуль;Единица;Мнимая единица;числовая структура;система счисления;десятичная система счисления;двоичная система счисления;шестнадцатеричная система счисления; дробь; обыкновенная дробь; десятичная дробь; цифра; арабская цифра; римская цифра

\end{scnhaselementrolelist}
\begin{scnhaselementrolelist}{исследуемое отношение}

противоположные числа*;модуль*;сумма*;произведение*;возведение в степень*;больше*;равенство*;больше или равно*

\end{scnhaselementrolelist}
\scnheader{число}
\scnidtf{множество чисел}
\scnsubset{абстрактная терминальная сущность}
\scntext{explanation}{\textbf{\textit{число}} -- это основное понятие математики, используемое для количественной характеристики, сравнения, нумерации объектов и их частей. Письменными знаками для обозначения чисел служат \textit{цифры}.}\scnheader{цифра}
\scnidtf{множество цифр}
\scnsubset{внутренний файл ostis-системы}
\begin{scnrelfromlist}{включение}

\scnitem{арабская цифра}
\scnitem{римская цифра}

\end{scnrelfromlist}
\scntext{explanation}{\textbf{\textit{цифра}} - это множество файлов, обозначающих вхождение этой цифры во всевозможные записи чисел с помощью этой цифры.}\scnheader{натуральное число}
\scnidtf{множество натуральных чисел}
\scntext{explanation}{\textbf{\textit{натуральное число}} -- это подмножество множества \textit{целых чисел}, которые используются при счете предметов.}\scnsubset{целое число}
\scnheader{целое число}
\scnidtf{множество целых чисел}
\scntext{explanation}{\textbf{\textit{целое число}} -- это подмножество множества \textit{рациональных чисел}, получаемых объединением \textit{натуральных чисел} с множеством чисел, \textit{противоположных* натуральным} и \textit{нулём}.}\scnsubset{рациональное число}
\scnheader{рациональное число}
\scnidtf{множество рациональных чисел}
\scntext{explanation}{\textbf{\textit{рациональное число}} -- это число, представляемое \textit{обыкновенной дробью}, где числитель  \textit{целое число}, а знаменатель  \textit{натуральное число}.}\scnsubset{действительное число}
\scnheader{дробь}
\scnidtf{множество дробей}
\begin{scnrelfromlist}{включение}

\scnitem{обыкновенная дробь}
\scnitem{ десятичная дробь}

\end{scnrelfromlist}
\scntext{explanation}{\textbf{\textit{дробь}}  это число, состоящее из одной или нескольких равных частей (долей) единицы}\scnheader{обыкновення дробь}
\scnidtf{множество обыкновенных дробей}
\scnidtf{множество простых дробей}
\scntext{explanation}{\textbf{\textit{обыкновенная дробь}} - запись \textit{рационального числа} в виде $\displaystyle \pm \frac m$
\end{scnsubstruct}
\end{SCn}
